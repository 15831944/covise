
\begin{htmlonly}

\usepackage{html, htmllist}
\usepackage{longtable}

\bodytext{bgcolor="#ffffff" link="#0033cc" vlink="#0033cc"}

%%%==================================================	
%%%==================================================	

% #1  mark defined by \label
% #2  a linktext 
% #3  a html link 
\newcommand{covlink}[3]{\htmladdnormallink{#2}{#3} \latex{(\ref{#1})} }


\newenvironment{covimg}[4]%
{
 \begin{figure}[htp]
  \begin{center}
   \latexonly
      \includegraphics[scale=#4]{#1/pict/#2}
   \endlatexonly  
   \html{\htmladdimg[align="center"]{pict/#2.png}}
   \caption{#3}
  \end{center}
 \end{figure} 
}{} 

\newenvironment{covimg2}[3]%
{ 
 \begin{figure}[htp]
  \begin{center}
     \latexonly
       \includegraphics[scale=#3]{#1/pict/#2}   
     \endlatexonly
     \html{\htmladdimg[align="center"]{pict/#2.png}}
  \end{center}
 \end{figure} 
}{}

\definecolor{output}{rgb}{0.,0.,1.}
\definecolor{depend}{rgb}{1.,0.65,0.}
\definecolor{required}{rgb}{0.58,0.,0.83}
\definecolor{optional}{rgb}{0.,0.39,0.}

\newcommand{\addimage}[1] {\html{\htmladdimg{pict/#1.png}}}

\newcommand{\addpict}[4] {\latexonly
	     \begin{figure}[!htbp]
			  \begin{center}
   	 		  \includegraphics[scale=#1]{#2}
   	 		  \caption{#3}
		 		  \label{#4}
			  \end{center}
	 		\end{figure}
	     \endlatexonly}



\end{htmlonly}


%=============================================================
\startdocument
\label{Preface}
%=============================================================

\begin{Huge}{\bf Preface}\end{Huge}
%{\huge Preface}
\vspace{0.5cm}

This document is a short introduction to working with COVISE. It is primarily a 
tutorial for COVISE novices. It doesn't cover advanced topics such as the 
development of new application modules or the installation and configuration 
process. We assume that you have a running COVISE on your machine. For installation 
guide read the files {\it README} and {\it INSTALL.TXT} which come with your 
COVISE distribution. For developing new application modules read the 
{\it COVISE Programming Guide}.

\vspace{0.5cm}
COVISE is a Collaborative Visualization and Simulation Environment developed 
at the Computing Center of the University of Stuttgart. It is an
extendable distributed software environment to integrate supercomputer based 
simulations, postprocessing, and visualization functionality with
cooperative working in a seamless manner.

\vspace{0.5cm}
The tutorial contains the five chapters:

In {\it Chapter 1 Starting COVISE} you learn how to initiate a single user 
session and load a saved session. The functionality of some basic modules
(RWCovise, DomainSurface, CuttingSurface) is explained. 

{\it Chapter 2 Using the Inventor Renderer} gives a short introduction to the 
typical work with the Renderer. 

{\it Chapter 3 Working with Modules} covers module ports and module parameters. 

{\it Chapter 4 Analysis of 3D Data} describes the basic steps in analyzing
complex 3D data with the general COVISE modules. (This chapter has been 
reworked to make the user familiar with the 'Complex Modules'  
for building maps quicker and easier.) 

{\it Chapter 5 Advanced Topics} covers distributed computing and multi user 
sessions. 

\vspace{0.5cm}
This tutorial uses the following conventions: 

File names, path names, and environment variables are printed {\it italic}, for example 

{\it \$(COVISEDIR)/net/general/examples/ReadStar.net}

Program names and COVISE module names are printed {\bf bold}, for example the 
command to start COVISE: {\bf covise}, or the name of the module that
computes a cutting plane: {\bf CuttingSurface}. 

Output to the console window which COVISE produces is printed in fixed space font like 
\begin{verbatim}
Starting COVISE...
Please be patient... 
\end{verbatim}

\begin{longtable}{|l|}
\hline
Some sections are included in a frame, indicating troubleshooting information.\endhead
\hline
\end{longtable}
