
\begin{htmlonly}

\usepackage{html, htmllist}
\usepackage{longtable}

\bodytext{bgcolor="#ffffff" link="#0033cc" vlink="#0033cc"}

%%%==================================================	
%%%==================================================	

% #1  mark defined by \label
% #2  a linktext 
% #3  a html link 
\newcommand{covlink}[3]{\htmladdnormallink{#2}{#3} \latex{(\ref{#1})} }


\newenvironment{covimg}[4]%
{
 \begin{figure}[htp]
  \begin{center}
   \latexonly
      \includegraphics[scale=#4]{#1/pict/#2}
   \endlatexonly  
   \html{\htmladdimg[align="center"]{pict/#2.png}}
   \caption{#3}
  \end{center}
 \end{figure} 
}{} 

\newenvironment{covimg2}[3]%
{ 
 \begin{figure}[htp]
  \begin{center}
     \latexonly
       \includegraphics[scale=#3]{#1/pict/#2}   
     \endlatexonly
     \html{\htmladdimg[align="center"]{pict/#2.png}}
  \end{center}
 \end{figure} 
}{}

\definecolor{output}{rgb}{0.,0.,1.}
\definecolor{depend}{rgb}{1.,0.65,0.}
\definecolor{required}{rgb}{0.58,0.,0.83}
\definecolor{optional}{rgb}{0.,0.39,0.}

\newcommand{\addimage}[1] {\html{\htmladdimg{pict/#1.png}}}

\newcommand{\addpict}[4] {\latexonly
	     \begin{figure}[!htbp]
			  \begin{center}
   	 		  \includegraphics[scale=#1]{#2}
   	 		  \caption{#3}
		 		  \label{#4}
			  \end{center}
	 		\end{figure}
	     \endlatexonly}



\end{htmlonly}


%=============================================================
\startdocument
\chapter{Using the OpenInventor Renderer}
\label{Using_the_Renderer}
%=============================================================

\section{Introduction}

After having read this chapter you will be familiar with: 

\begin{itemize}
\item Renderer modules in general 
\item the user interface of the Renderer 
\item Mouse interaction modes
\end{itemize}


In the previous chapter you have read about some basic steps in working with COVISE. 
You have already learned how to load a prepared module network. In this chapter we 
introduce some functionalities of the OpenInventor {\bf Renderer} and their main usages. 
COVISE comprises other render modules as well but the OpenInventor Renderer can be seen
as the default desktop renderer. OpenInventor is a high level object oriented graphics 
toolkit developed by Silicon Graphics. It uses OpenGL, the 3D graphics standard, as its 
rendering interface. 

You can regard the {\bf Renderer} as a movable window to a virtual world. The displayed 
scenario contains objects which are typically visual representations of simulation data. 
Depending on the selected visualization method you can have different visual representations 
for the same dataset. This allows to explore the various aspects and characteristics of
a dataset and detect features otherwise not seen. In addition to the mapping of data to 
visual representations supportive geometries, such as bounding surfaces or reference 
geometries of machineries provided via CAD models can also be included in combination with 
the data visualizations. A typical example would be the visualization of a pressure 
distribution in a combustion chamber of a car engine. Such scenarios can be further 
examined with all the possibilities that the Renderer provides. 


\section{The Render Module}

\begin{covimg}{Using_the_Renderer}{RendererModule}{The Renderer Module Icon}{0.7}\end{covimg}
\begin{htmlonly}
Figure 2.1 : The Renderer Module Icon
\vspace{0.5cm}
\end{htmlonly}

At the end of a module network processing chain there is an output module, which in most 
cases will be a render module to visualize resulting data. Other output modules are write 
modules for saving results to disk or a special plot module for xy- charting. In the 
{\bf MapEditor} working area those output modules are typically placed at the bottom of the
map. Output modules normally have input ports only (blue rectangle on the left top of the 
render module in Figure 2.1). 

The module selection and the usage of ports to build module networks will be described 
in the following chapters

\section{The User Interface}

In Figure 2.2 you can see the OpenInventor Renderer as it will appear on your desktop 
after you have inserted the Renderer module in the module placing area.

The renderer window consists of three major parts: 

\begin{itemize}
\item the menubar
\item the information area
\item the viewer area.
\end{itemize}

The menubar will only be explained shortly. It provides functionalities such as saving 
the current scene to a file or printing the scene. You can also adjust viewing parameters 
or set material and light properties using the menus. The information area provides 
information about the current state of the renderer. Beside some text fields for 
collaboration status there are two list boxes. The ColorMaps list box contains all the 
colormaps used in the visualization pipeline. The GeometryObjects list box contains all the geometry objects currently displayed. 
The viewer area holds the main viewer widget where the objects are displayed. The viewer 
area contains also other user interface elements.

\begin{covimg}{Using_the_Renderer}{RendererUIF}{The Renderer User Interface}{0.7}\end{covimg}
\begin{htmlonly}
Figure 2.2: The Renderer User Interface
\vspace{0.5cm}
\end{htmlonly}

Figure 2.3 shows the toolbar at the right side of the viewer area in greater detail. One 
feature of the toolbar is the home position. It represents the special camera position and 
orientation of an initial view. By pressing the home position button in the toolbar the 
camera will switch to the position and orientation of the viewers home position. With the set
home position button you can set these values at any time.

\begin{covimg}{Using_the_Renderer}{Toolbar}{The Toolbar}{0.7}\end{covimg}
\begin{htmlonly}
Figure 2.3: The Toolbar
\vspace{0.5cm}
\end{htmlonly}

Another feature of the toolbar is the view all button. Pressing this button causes the 
{\bf Renderer} to adjust the camera position and orientation to view all objects. 

Pressing the seek mode button changes the cursor to the seek mode symbol. Within this mode 
you can click on any part of your geometry. This will set the camera position in a way, 
that the selected point will be in the center of the window with a view orientation 
orthogonal to the object. 

The button on the bottom of the toolbar selects a projection mode. By pressing this button 
you can switch between a perspective and an orthogonal projection. The default projection 
mode is the perspective one.

Arranged at the bottom of the viewer area are some user interface elements for rotating 
and zooming the camera. On the left side are two thumb wheels to rotate the camera about 
the x (Rotx) and the y (Roty) axis. 

On the right side is a slider (Zoom) and a thumb wheel (Dolly) to zoom the camera.


\clearpage
\section{Mouse Interaction Mode}

The OpenInventor Renderer knows two fundamental interaction modes of the mouse: 

\begin{covimg2}{Using_the_Renderer}{ViewMode}{1.0}\end{covimg2} 
The view mode transforms mouse interactions in viewpoint changes and the pick mode is used 
to change the properties of geometric objects. An active view mode is indicated by an 
active view mode button in the toolbar on the right side of the window. Also the cursor 
changes to a hand symbol. Within this mode you can rotate the whole scene about its center 
using the left mouse button. Move the mouse pointer over the geometry, press the button and 
move the mouse in that direction you want to rotate to. By releasing the button without 
moving the mouse simultaneously, the scene will be rendered from the new viewpoint. When 
you drag the mouse and release the button while you are moving, the scene will spin 
permanently until you click into to the viewer area. This continuation of the last 
transformation operation after mouse button release only works with rotations. Using 
the middle mouse button you can pan the scene. Using the combined left and middle button 
you can zoom the viewpoint in the same manner. Thus, dragging the mouse to the bottom of 
the window with the left and middle mouse button pressed, you will zoom into the scene. 
Dragging to the top will zoom out.

\begin{covimg2}{Using_the_Renderer}{PickMode}{1.0}\end{covimg2} 
The pick mode is indicated by an active pick button  in the toolbar and the cursor changes 
to an arrow. In this mode you can select individual geometries for further manipulations. 
By clicking with the left mouse button on a geometry object you select the geometry. The 
selection will be indicated by a bounding box (a red wireframed box) around that object. 
For the selection of multiple objects hold down the shift key while selecting the
different objects. To remove the selection of the object just select the object again. 
To remove the whole selection click on any free part of the viewer widget. This working 
method is just the same you already know from popular drawing programs. 

By clicking with the right mouse button into the viewer area in any of the two interaction 
modes a popup menu appears. This popup menu will not be further explained here. It provides 
many functionalities you find on the menubar too.




