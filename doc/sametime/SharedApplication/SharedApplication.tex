\begin{htmlonly}\begin{htmlonly}
\documentclass{covise}

\usepackage{html, htmllist}
\usepackage{color}
\usepackage{graphicx}
\usepackage{longtable}
\usepackage{palatino}
\usepackage{picins}
\usepackage[colorlinks,dvips]{hyperref}

	  
\bodytext{BGCOLOR=FFFFFF LINK=#0033cc VLINK=#0033cc}

\newcommand{\covlink}[3]%
{\html{\htmladdnormallink{#Architecture.tex1}{#3}}\latex{\hyperref[#1]{#2} (\ref{#1})}}


\newenvironment{covimg}[4]%
{ \html{\htmladdimg[ALIGN=CENTER]{#2.gif}}
 
 \latexonly
 \begin{figure}[htp]
  \begin{center}
   \includegraphics[scale=#4]{#1/#2}
   \caption{#3}
  \end{center}
 \end{figure}
 \endlatexonly
}

\newenvironment{covimg2}[3]%
{ \html{\htmladdimg[ALIGN=CENTER]{#2.gif}}
 
 \latexonly
 \begin{figure}[htp]
  \begin{center}
   \includegraphics[scale=#3]{#1/#2}
  \end{center}
 \end{figure}
 \endlatexonly
}

\definecolor{output}{rgb}{0.,0.,1.}
\definecolor{depend}{rgb}{1.,0.65,0.}
\definecolor{required}{rgb}{0.58,0.,0.83}
\definecolor{optional}{rgb}{0.,0.39,0.}

\end{htmlonly}


%=============================================================
\startdocument
\chapter{Shared Application}
\label{SharedApplication}
%=============================================================

%\section{S1}

You have two possibilities to communicate via your computer screen during a meeting:
\begin{itemize}
\item the Whiteboard
\item a Shared Application
\end{itemize}
\vspace{0.5cm}

A typical application to share with all meeting participants could be Word. You could e.g. look at
a word file containing the Agenda at the beginning of the meeting in order to discuss and modify
it if necessary, or build up a Summary or a ToDo-List file at the end of the meeting.

In order to make this application a Shared Application use the facilities provided for Screen
Sharing.

\vspace{0.5cm}
{\bf Screen Sharing}
\vspace{0.5cm}  

If you have permission to edit/share, you can share your screen with other participants and even
allow them to control your screen. Before sharing your screen, make sure that any confidential
information is secure.

\begin{enumerate}
\item Click on the screen-sharing button near the top of the Meeting Room.
\item Open the program(s) that you want to share e. g. Word.
\item Read the screen-sharing help in the Meeting Room to select one of the following options from
the screen-sharing toolbar:
\begin{itemize}
\item Share a program
\item Share My Entire Screen
\item Share Part of My Screen with a Frame
\end{itemize}
\item You can be sure that you are sharing when you see the StopSharing button in the upper right
of the shared area.
\item To stop sharing at any time, click Stop Sharing.
\end{enumerate}

\vspace{0.5cm}
{\bf Give Control to Others} - applies both to Shared Applications and Whiteboard
\vspace{0.5cm}  

You can allow other participants with permission to edit/share to control your screen. When you
give control to other participants, they can use their own computers to edit the information on
your shared screen.
\begin{enumerate}
\item Begin Sharing your screen.
\item Click Allow Control. Other participants can now control your screen. Do not use your
keyboard or mouse while another participant has control.
\item To confirm who controls your shared screen, click the Participants List details button at
the bottom of the Participant List. An arrow appears next to the name of the participant who
currently controls the shared screen.  
\item To remove contro from other participants, click Allow Control.
\end{enumerate}


%\section{S2}
%\section{S3}

Fig. 5. 1. below shows you - as an example - which help information is available for COVISE, and you could use your
application to display help for the COVISE session.

\begin{covimg}{SharedApplication}{shared}{Use Shared Application to show available Help}{0.7}\end{covimg}
\begin{htmlonly}
Figure 5.1: Use Shared Application to show available Help
\vspace{0.5cm}
\end{htmlonly}
