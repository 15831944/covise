\begin{htmlonly}\begin{htmlonly}
\documentclass{covise}

\usepackage{html, htmllist}
\usepackage{color}
\usepackage{graphicx}
\usepackage{longtable}
\usepackage{palatino}
\usepackage{picins}
\usepackage[colorlinks,dvips]{hyperref}

	  
\bodytext{BGCOLOR=FFFFFF LINK=#0033cc VLINK=#0033cc}

\newcommand{\covlink}[3]%
{\html{\htmladdnormallink{#Architecture.tex1}{#3}}\latex{\hyperref[#1]{#2} (\ref{#1})}}


\newenvironment{covimg}[4]%
{ \html{\htmladdimg[ALIGN=CENTER]{#2.gif}}
 
 \latexonly
 \begin{figure}[htp]
  \begin{center}
   \includegraphics[scale=#4]{#1/#2}
   \caption{#3}
  \end{center}
 \end{figure}
 \endlatexonly
}

\newenvironment{covimg2}[3]%
{ \html{\htmladdimg[ALIGN=CENTER]{#2.gif}}
 
 \latexonly
 \begin{figure}[htp]
  \begin{center}
   \includegraphics[scale=#3]{#1/#2}
  \end{center}
 \end{figure}
 \endlatexonly
}

\definecolor{output}{rgb}{0.,0.,1.}
\definecolor{depend}{rgb}{1.,0.65,0.}
\definecolor{required}{rgb}{0.58,0.,0.83}
\definecolor{optional}{rgb}{0.,0.39,0.}

\end{htmlonly}


%=============================================================
\startdocument
\chapter{Usage Hints}
\label{Usage Hints}
%=============================================================

As a first step you will find below the short User's Guide that developed as part 
of the EU project ENScube and that can be used as a kind of checklist.

As a second step you get a reference where to find additional and more detailed user information.

\section{Simplified User Guide}

\begin{Large}{\bf Prepare your meeting}\end{Large}

{\bf Define a detailed agenda:}

\begin{itemize}
\item Date, time and duration
\item List of participants
\item Meeting room
\item Topics to be discussed / supporting documents / who provides the documents ?
\item Chairperson
\end{itemize}

{\bf Transfer the documents and data:}
\vspace{0.5cm}


\begin{Large}{\bf Join the meeting}\end{Large}

{\bf Open your desktop on the PC}

\begin{itemize}
\item Ctrl + Alt + Del
\item Username
\item Password
\end{itemize}

{\bf Join the meeting}

\begin{itemize}
\item Display the meeting place portal:	click the icon (lower right)
\item Username
\item Password
\item Room
\end{itemize}
\vspace{0.5cm}


\begin{Large}{\bf Can you hear me ? - AUDIO}\end{Large}

\begin{itemize}
\item Check carefully that the audio quality is good for each collaborator. Set the microphones and speakers/headsets level.
\end{itemize}

\vspace{0.5cm}

\begin{Large}{\bf Can you see me ?	 - VIDEO}\end{Large}

\begin{itemize}
\item Check your camera is correctly focused
\end{itemize}

{\bf See your collaborators}

\begin{itemize} 
\item Switch to collaborators video:	select ??? in the menu "users" below the video. 
\item Confirm using the audio that you can see your collaborator; comment on the quality of the video.
\end{itemize}
\vspace{0.5cm}

\begin{Large}{\bf Application Sharing}\end{Large}

{\bf Prepare the document}

{\bf Share the document}

\begin{itemize}
\item Ask your collaborator what they can see
\end{itemize}

{\bf Allow control to collaborators}

{\bf Take control}

\begin{itemize}
\item Ask for control and wait for the agreement of the chairperson
\end{itemize}

{\bf Switch to a new document}

{\bf Recommendations:}

\begin{itemize}	
\item Normally, the first part of the meeting, once the Audio/Video is correctly set, 
is to check collaboratively the Agenda: the Agenda is shared by the chairperson.
\end{itemize} 

\begin{Large}{\bf Whiteboard}\end{Large}

{\bf Recommendations:}

\begin{itemize}	
\item use different colors, one for each partner
\item wait your collaborator is done before adding something
\item the chairperson organize the use of the whiteboard: who can take hand ...
\end{itemize}
\vspace{0.5cm}

\begin{Large}{\bf Collaborative visualization}\end{Large}

{\bf Start the collaborative visualization}

{\bf Basic manipulations (Renderer)}

\begin{longtable}{|p{5cm}|p{8.5cm}|}
\hline
        Translate the view &  {\it Click and drag middle mouse} \\
\hline
	Rotate the view 
	&  {\it Rotx toggle OR Roty toggle OR click and drag left mouse} \\
\hline
	Zoom in/out the view & {\it Dolly toggle}  \\
\hline
        Zoom on one detail &   \\
\hline
	View all objects &   \\
\hline
	Retrieve the original view &   \\														
\hline
	 Pointer (with machine name)\newline
	 ({\bf Telepointer}) 
	 &  {\it Press Shift on your keyboard and move your mouse} \\	
\hline
\end{longtable}


{\bf Exchange control (Master/Slave)}

{\bf Stop the collaborative visualization}


{\bf Recommendations:}

\begin{itemize}	
\item Stop your application sharing to free bandwidth
\end{itemize}

\vspace{0.5cm}

\begin{Large}{\bf Close the meeting}\end{Large}
\vspace{0.5cm}

\begin{Large}{\bf Exchange the documents}\end{Large}
\vspace{0.5cm}

\begin{Large}{\bf Procedures to solve technical difficulties}\end{Large}
\vspace{0.5cm}


\section{Sametime User's Guide}

\begin{htmlonly}N'S�\end{htmlonly}\latexonly N'S$^3$\endlatexonly (Conference
Room Interface) uses Sametime from IBM Lotus, and the
information provided in this document is based on the Sametime documentation. If you 
have problems/questions beyond the scope of this document you can refer to the IBM
Lotus website: \begin{verbatim}/http: //www.lotus.com/\end{verbatim} 

You will get there following documents
\begin{itemize}
\item Lotus Sametime - Quick Start Guide
\item Lotus Sametime - User's Guide
\end{itemize}

\vspace{0.5cm}

\begin{itemize}
\item Use \begin{verbatim}Products > Sametime > Demos & Trials > Quick Start Guide
\end{verbatim} to get Quick Start Guide or User's Guide as .pdf file   
\end{itemize}

\begin{itemize}
\item Use \begin{verbatim}Products > Sametime > Demos & Trials > Demo Site 
\end{verbatim} to get Quick Start Guide or User's Guide online
\end{itemize}


%\section{S3}


%\begin{covimg}{UsageHints}{imagename}{caption}{0.7}\end{covimg}
%\begin{htmlonly}
%Figure x.x: caption for html
%\vspace{0.5cm}
%\end{htmlonly}
