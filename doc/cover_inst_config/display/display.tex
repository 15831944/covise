\begin{htmlonly}

\usepackage{html, htmllist}
\usepackage{longtable}

\bodytext{bgcolor="#ffffff" link="#0033cc" vlink="#0033cc"}

%%%==================================================	
%%%==================================================	

% #1  mark defined by \label
% #2  a linktext 
% #3  a html link 
\newcommand{covlink}[3]{\htmladdnormallink{#2}{#3} \latex{(\ref{#1})} }


\newenvironment{covimg}[4]%
{
 \begin{figure}[htp]
  \begin{center}
   \latexonly
      \includegraphics[scale=#4]{#1/pict/#2}
   \endlatexonly  
   \html{\htmladdimg[align="center"]{pict/#2.png}}
   \caption{#3}
  \end{center}
 \end{figure} 
}{} 

\newenvironment{covimg2}[3]%
{ 
 \begin{figure}[htp]
  \begin{center}
     \latexonly
       \includegraphics[scale=#3]{#1/pict/#2}   
     \endlatexonly
     \html{\htmladdimg[align="center"]{pict/#2.png}}
  \end{center}
 \end{figure} 
}{}

\definecolor{output}{rgb}{0.,0.,1.}
\definecolor{depend}{rgb}{1.,0.65,0.}
\definecolor{required}{rgb}{0.58,0.,0.83}
\definecolor{optional}{rgb}{0.,0.39,0.}

\newcommand{\addimage}[1] {\html{\htmladdimg{pict/#1.png}}}

\newcommand{\addpict}[4] {\latexonly
	     \begin{figure}[!htbp]
			  \begin{center}
   	 		  \includegraphics[scale=#1]{#2}
   	 		  \caption{#3}
		 		  \label{#4}
			  \end{center}
	 		\end{figure}
	     \endlatexonly}



\end{htmlonly}

%================================================================================
%================================================================================


%================================================================================
\startdocument
\chapter{Graphics Board and Display}
\label{label_chapter_display}

VR Environments today come in various flavours  First you have to differentiate between head mounted displays (HMDs) and 
projection based virtual environments. Both types are supported by OpenCOVER.
For stereo you need either a graphics board which supports active stereo video
formats, such as NVIDIA Quadro boards, a graphics board with two video outputs, a computer with
two graphics boards or two computers connected to each other via a LAN.
Some display devices such as Autostereoscopic displays can be driven by a single graphics output by using interlaced
stereo modes. Some of those are also supported by OpenCOVER.
For driving more than one display device like in tiled displays or multiple screen projection environments, either a cluster of computers, 
multiple graphics boards in one computer or any combination of those are nedded.
The following sections explains which combinations are supported by OpenCOVER
and how they are configured.


\section{Virtual Environment Types with Fixed Screens}
\label{label_section_vetypes}

OpenCOVER supports all types of Virtual Environments (VEs) with spatially
fixed  planar screens like a CAVE, Powerwall, Workbench or Holobench.
The VEs differ in the number of screens, the number of people they are suited for
and the level of immersion.

A Powerwall is a semi-immersive system with one big flat translucent canvas. 
The image is projected from behind, so that people don't stand in the optical path
of the projector.
The resolution can be enhanced by composing the final image from several graphics
boards, video outputs or several computers. A Powerwall is well suited
for presentations for large groups. As the user is not completely surrounded by screens
the immersion is lost when he looks too far to the left or right, up or down.

A Workbench is a semi-immersive system with a single translucent canvas. 
The workbench dimensions are similar to a drawing board and often the screen 
can be tilted back.
The advantage of a workbench is, that
the user can reach every point on the screen. It is suitable for small groups 
(3-4 persons).
The stereo impression and therefore the immersion gets lost when objects 
in the scene are cut by the frame of the workbench.

\begin{covimg}{display}{workbench}{RUS Workbench}{0.4}\end{covimg}

The Visenso Cykloop is a mobile one-screen backprojection system
with a screen size of 160 x 120 cm.

\begin{covimg}{display}{cykloop}{Vircinity Cycloop}{0.4}\end{covimg}

A CAVE (TM) is a system where the walls form a cube. The first CAVE was 
built at EVL. There are CAVEs with 4 screens (left, front, right and bottom), with 5 screens, (additionally the ceiling)
and 6 screens. The more screens the CAVE has the better is the immersion. The dimension of
one cube side is typically 3000 mm x 3000 mm. A CUBE is suited for smaller number of persons up to six or eight.

\begin{covimg}{display}{cave}{RUS Cube}{0.4}\end{covimg}

A Holobench has two screens, a front and a bottom screen.
The bottom screen is on the height of a table and the dimensions are like the 
ones of a workbench.
A Holospace also has a front and a bottom screen, but the bottom screen
is on the floor and the dimensions are like a CAVE.

Tiled displays are monoscopic or autostereoscopic highres displays composed out of an array of flat pannel displays.

% TODO: missing file
%\begin{covimg}{display}{TiledDisplay}{Calit Varrier}{0.4}\end{covimg}

\section{Stereo Modes}
\label{label_section_stereo_modes}

To improve the immersion in the VE, COVER supports stereo. Stereo means
that two images are presented to the user, one computed from the
view of the left eye, and one computetd from the view of the right eye.
Stereo can be turned on or off by setting the Stereo value to "ON" or "OFF". The default is "OFF"!

\small \begin{verbatim}
<Stereo vaue="ON" />
\end{verbatim} \normalsize

The separation between the two eyes can be specified with separaton attribute in the Stereo entry. The actual distance between the eyes differs from human to human but
the default of 60 mm is good in most cases. If you want to disable stereo for taking pictures or a movie, you can temorarily disable the stereo separation in
the 3D GUI under View options or set this value to 0.  

\small \begin{verbatim}
<Stereo separation="60" />
\end{verbatim} \normalsize

Special glasses ensure that each eye sees only the appropriate image. 
As in natural viewing, the brain forms a 3D image from the two images
from two different perspectives.

There are several methods for stereo viewing: red-green stereo, interlaced stereo and most important: active and passive stereo.

\subsection{Active Stereo}
		
In active stereo, the graphics card of the computer needs to be able
to switch to a video mode, where the left and the right image are stored
in different areas of the video memory. Every video frame it
switches between the left and the right image. This mode is called QuadBuffer mode. Special
glasses called shutter glasses which can make the glass opaque are synchronised
with the graphics card trough an infrared emitter. When the graphics card
displays the left image, the right glass is set to opaque and vice versa.

\begin{covimg}{display}{shutterglasses}{Shutter Glasses and Infrared
Emitter}{0.7}\end{covimg}

This stereo mode is supported among others on NVIDIA Quadro boards and AMD/ATI FireGL boards as well as most SGI Systems.
In order to enable QuadBuffer Stereo, you have to modify the X-Configuration according to your driver's documentation.
On WindowsXP and Vista, you have to open your graphicsdriver controll pannel, go to 3D options and enable stereo.
On Linux systems, you have to modify your X configuration, typically located in /etc/X11/xorg.conf
for ATI boards you have to add the following to your xorg.conv:
\begin{samepage}
\small \begin{verbatim}
Option "Stereo" "on"
\end{verbatim} \normalsize
\end{samepage}

for NVIDIA boards you have to select an appropriate Stereo mode similar to the following line, see
/usr/share/doc/NVIDIA\_GLX-1.0/README.txt

\begin{samepage}
\small \begin{verbatim}
Option "Stereo" "3"
\end{verbatim} \normalsize
\end{samepage}
After restaring our X server, you can verify that your graphics board is in stereo mode by running "glxinfo".
In the output, there is a column called "st ro" (for stereo). Some of the lines have to contain a "y" in that column

\begin{samepage}
\small \begin{verbatim}
   visual  x  bf lv rg d st colorbuffer ax dp st accumbuffer  ms  cav
 id dep cl sp sz l  ci b ro  r  g  b  a bf th cl  r  g  b  a ns b eat
----------------------------------------------------------------------
0x21 24 tc  0 32  0 r  y  .  8  8  8  0  4 24  8 16 16 16 16  0 0 None
0x22 24 tc  0 32  0 r  y  y  8  8  8  0  4 24  8 16 16 16 16  0 0 None
0x23 24 tc  0 32  0 r  y  .  8  8  8  8  4 24  8 16 16 16 16  0 0 None
\end{verbatim} \normalsize
\end{samepage}

On SGI Maximum Impact and Infinite Reality Systems it is 
possible to switch between a mono and a stereo video format on the fly 
as longs as the "managed area" remains the same.

If you want to
switch the monitor into stereo only when COVER is running you can
specify the setmon command with the command attributes to the Stereo and Mono entry.

Example:

\begin{samepage}
\small \begin{verbatim}
<Stereo command="/usr/gfx/setmon -n 1024x768_96s" />
<Mono command="/usr/gfx/setmon -n 1280x1024_76" />	
\end{verbatim} \normalsize
\end{samepage}

With setmon only standard video formats can be loaded. Infinite 
Reality Systems support also custom video formats which can be loaded
with the command ircombine.
If the video mode should not be changed (for example because the correct mode is
already running all the time), set command to and empty string. (this is the default)

Finally, you have to select a stereo visual for rendering. you can do that with the stereo attribute in the appropriate Window Entry.

\begin{samepage}
\small \begin{verbatim}
<Window index="0" width="1280"height="1024" decoration="false" stereo="true" />	
\end{verbatim} \normalsize
\end{samepage}


\subsection{Passive Stereo}

In passive stereo the left and the right image are presented at the
same time, but at two different video outputs which are connected to
two projectors. Each projector has a polarised lense. 
Special stereo glasses with polarised glasses ensure that each eye
sees the appropriate image.

The application
typically opens two windows or one window with two viewports and
draws the left and right image into the separate windows/viewports.
Another possibility is that two different computers create the left
and the right image.

	  
\begin{covimg}{display}{polarizedglasses}{Polarized Glasses}{0.7}\end{covimg}

Passive stereo can be created either with a graphics board with two
video outputs or with two computers.

\paragraph{Passive stereo with two video outputs}
Most of today's graphics boards support two video
outputs (dual-head). On nvidia cards it is called TwinView. 
If the window manager supports the Xinerama mode, you can also configure
only one large window with two viewports aka channels. If not, you need to configure
two windows. The preferred choice is one large window with two
channels, because it needs less ressources (for example textures need
to be loaded to the graphics board only once).
On Linux, you can enable the second output of your graphics card by modifying you xorg.conf, see the following example:
note that you have to disable XineramaInfo, otherwise you can't open a single window which spans both screens.

\begin{samepage}
\small \begin{verbatim}
  Option      "TwinView" "true"
  Option      "ConnectedMonitor" "CRT-0, DFP-1"
  Option      "SecondMonitorHorizSync" "30-168"
  Option      "SecondMonitorVertRefresh" "50-60"
  Option      "MetaModes" "1280x1024,1400x1050x60.00"
  Option      "IgnoreEDID" "true"
  Option      "NoTwinViewXineramaInfo" "true"
\end{verbatim} \normalsize
\end{samepage}


On SGI system you need the "multichannel option", this means a graphics
board with more than one video output. SGI Infinite Reality systems
can have 2-8 video outputs.


\paragraph{Passive stereo with two computers}
COVER can also be configured to use two computers (nodes) to create one
(passive) stereo image. The computers need to be synchronised
either via a serial cable or via ethernet. 
A powerwall or a cave can be driven by any number of nodes. Typical combinations are one node per screen with a passive stereo dual head configuration per
node or one node per screen and per eye. The latter is the configuration with highest rendering performance.
 
2*n or 2*n+1 computers. In the 2n+1 mode one computer
can be used as controlling computer, where COVISE is started and
where the Mapeditor is running, the other 2*n computers are then used 
only for rendering. You can also have COVER on the controlling computer.
In case that the controlling computer doesn't have a COVER, you
have to start the tracker through the Vircinity Device Server (see chapter
Tracking and Input Devices). The tracking system is connected
only to the masterPC or the controlling computer. the tracking
information is sent to the other computers via the network.

This multi-PC mode is configured in covise.config in the section
MultiPC. Here you indicate which computer is the master and which are
the slaves. Each computer needs a screen, window and channel configuration
for mono mode. As one half of the computers need to draw a left image
and the other alf a right image, this has to be indicated:

\small 
\begin{verbatim}
MultiPC: <masterPC> <slavePC1> <slavePC2> <...>
{
    SyncMode          <TCP SERIAL>
    SyncProcess       <APP DRAW>
    SerialDevice      <device>
    Master            <masterPC>
    MasterInterface   <masterPC>
    Host0             <slavePC1>    
    Host1             <....>
    ...
    numSlaves         <n>
}

COVERConfig: <masterPC>
{
    TRACKING_SYSTEM             <tracking system>
    STEREO                      off
    MONO_VIEW                   <LEFT RIGHT MIDDLE>
}

COVERConfig: <slavePC>
{
    TRACKING_SYSTEM             NONE
    STEREO                      off
    MONO_VIEW                   <LEFT RIGHT MIDDLE>
}

\end{verbatim}
\normalsize



\subsection{Memory consumption in active stereo mode} 
The active stereo modes (also called quadbuffer stereo) need four framebuffers, 
front and back (for doublebuffering), left and right (for stereo). 

The minimum amount of frame buffer memory you can allocate per pixel is 256 bits (small). The
amount of memory you can allocate per pixel can be doubled (512 or medium) or quadrupled 
(1024 or large).

During configuring the framebuffer with ircombine you select the frame buffer depth.
When you choose deepest the maximum pixeldepth for your configuration is selected.

With the program 'findvis' you can then check which framebuffer configurations (visuals)
are possible with the selected pixeldepth. For example, the IR2 pipe at RUS, configured for medium pixel depth,
offers the following configuration:
\small 
\begin{verbatim}
0x6f, RGBA 12/12/12/12, db, stereo, Z 23, S 1, samples 4
\end{verbatim}
\normalsize

COVER requests a framebuffer configuration with RGBA, doublebuffer, stereo and multisampling.
If it can't get such a visual, it selects one, which matches some of the criteria, for
example multisampling, but not stereo.

In case you don't get a visual with multisampling and stereo, you can tell COVER
to forgo multisampling:
\small 
\begin{verbatim}
COVERConfig
{
    ...
    ANTIALIAS OFF
    ...
}
\end{verbatim}
\normalsize

On SGI IR systems on each raster manager you have 80 GB of video memory. Here some typical
workbench configurations and the respective framebuffer calculations:

\begin{description}

\item [1024x768, framebuffer medium]
1024*768*512 = 50.33 GB. This combination of resolution and framebuffer depth is
 possible with one RM.

\item [2048x768, framebuffer medium]
2048*768*512 = 100.66 GB. With a managed area of 2048x768 you could configure 
two video outputs, each with 1028x768.
As this combination of resolution and framebuffer depth needs
more than 80 GB, you would need a second RM.

\end{description}

For COVISE and COVER it is quite useful, to have two video outputs, a monitor where
you run the mapeditor and a Workbench, CAVE or Powerwall where you run COVER.

\section{Loading Video Formats on SGI systems}
On SGI systems different video formats can be loaded. All systems besides
IR support only a fixed number of formats. Stereo Formats
are available on SGI Impact (1024x768\_96s), VPro (1280x1024\_100s), RealityEngine
(1024x768\_96s) and
InfiniteReality graphics boards. On IR Systems custom video formats
can be created using the video format compiler. Generally,
video format files reside in the directory 
\begin{verbatim}
/usr/gfx/ucode/<boardname>/vof
\end{verbatim}
and have the extension .vfo.

Custom video formats have to be loaded using the program /usr/gfx/ircombine,
while the predefined formats can be loaded with /usr/gfx/setmon. Look at the
manpages of setmon for details.

On IR systems you can have more than one video output per pipe. This can be
used to drive more than 1 projector for example for a powerwall system,
or a cave, or for passive stereo. The two video otputs are configured
with ircombine. Select 2 channels, apply the appropriate video format
on each channel and save this a combination file (extension .cmb).
The frequencies of the video formats in one combination need to be either
the same or multiples. The resolution doesn't need to be the same. Look
at the manpages of ircombine for details.

\section{Multipipe Synchronisation on SGI IR systems}
\label{label_section_multipe_sync}

If you drive your VE with more than one pipe (this is possible with
SGI IR systems only), the pipes have to be synchronised. For
synchronising the vertical retrace of the pipes, you have to genlock
the pipes with the genlock cable. When you create the video combination
you have to indicate, which pipe is the master (synchronisation intern)
and which one is the slave (synchronisation extern).

For synchronising the buffer swap
there are two different possibilities: hardware and software. For the
hardware solution you have to connect the swap ready cable.

The buffer swap synchronisation mode needs to be indicated in the section
COVERConfig. The keyword is PIPE\_LOCKING and the possible modes are called
CHANNEL (hardware), or WINDOW (software). 

\section{Configuration}
\label{label_section_configuration}
COVER reads the file covise.config to configure the number of
graphics boards, windows, viewports etc..
The following sections explain how to congigure COVER for a your VE.

\subsection{Pipe configuration}
The entry NUM\_PIPES in the section COVERConfig defines how many graphics 
boards are used for the application. 
\small
\small \begin{verbatim}
COVERConfig
{
    ...
    NUM_PIPES	<number of graphics boards used for COVER>
    ...
}
\end{verbatim}
\normalsize
In the section PipeConfig for each pipe used in the application
the appropriate hardware pipe is defined.
\begin{samepage}
\small
\small \begin{verbatim}
PipeConfig
{
    <index> <screen> <:server.display>
    <index> <screen> <:server.display>
    ...
}
\end{verbatim}
\end{samepage}
\normalsize
In multipipe operation the pipes have to be synchronized. See section 3.1.4
for details on pipe synchronisation on SGI Onyx systems.

\subsection{Screen configuration}
 
The entry NUM\_SCREENS in the section COVERConfig defines the number of 
phsical screens/walls in this VE. A workbench has one screen, a holobench 
two screens, a CAVE four two six screens etc.
\small
\begin{verbatim}
COVERConfig
{	
    ...
    NUM_SCREENS	<number of screens>
    ...
}
\end{verbatim}
\normalsize
The section ScreenConfig defines the size, position and orientation
of each screen in relation to the world coordinate system. In a CAVE
the recommended origin of the wolrd coordinate system is in the
center of the cube, in a workbench or powerwall system the best choice
for the origin is the middle of the screen.
\small
\begin{verbatim}
ScreenConfig
{
    <index> <name> <width  [mm]> <height[mm]> <origin [mm]> <euler angles [degree]>
    <index> <name> <width  [mm]> <height[mm]> <origin [mm]> <euler angles [degree]>
    ...
}
\end{verbatim}
\normalsize
\subsection{Window Configuration}

The entry NUM\_WINDOWS defines the number of windows. Imagine a 
holobench: if the computer has only one graphics board but two video
outputs you can define one pipe, one window and two channels, if the computer
has two graphics board you can define 2 pipes and 2 windows.
\small
\begin{verbatim}
COVERConfig
{
    ...
    NUM_WINDOWS	<number of windows>
    ...
}
\end{verbatim}
\normalsize
In the section WindowConfig the location of the window (left bottom corner) 
on the managed area and the size is defined:

\small \begin{verbatim}
WindowConfig
{
    <index> <name> <pipe index> <origin [pixels]> <size [pixels]>
    <index> <name> <pipe index> <origin [pixels]> <size [pixels]>
    ...
}
\end{verbatim} \normalsize


\subsection{Channel Configuration}

The number of channels (= viewports) is the same as the number of screens.
In the section ChannelConfig the left, bottom, right and top of the 
viewport in fractions between 0 and 1 or the left bottom width and height
in pixels in relation to the the window is defined. 

\small \begin{verbatim}
ChannelConfig
{
    <index> <name> <window index> <left> <bottom> <right> <top>
    <index> <name> <window index> <left> <bottom> <right> <top>
    ...
}
\end{verbatim} \normalsize
\small \begin{verbatim}
ChannelConfig
{
    <index> <name> <window index> <left> <bottom> <width> <height>
    <index> <name> <window index> <left> <bottom> <width> <height>
    ...
}
\end{verbatim} \normalsize




\newpage
\section{Example Configurations}
\label{label_section_examples}

In the covise directory you find several example configuration
\begin{itemize}
\item covise.config.cave-astereo-4pipes-motionstar: 4-sided cave, active stereo, two graphics pipes
\item covise.config.cave-astereo-2pipes-motionstar:  4-sided active, stereo cave, four graphics pipes
\item covise.config.workbench-astereo-polhemus: Tilted Workbench, active stereo, 1 pipe
\item covise.config.cycloop-pstereo-1window-fob: Cykloop, passive stereo, 1 pipe, 1 window, 2 viewports
\item covise.config.cycloop-pstereo-2window-fob: Cykloop, passive stereo, 1 pipe, 2 window
\item covise.config.cycloop-2cluster-vrc: Cykloop, passive stereo, cluster of 2 computers
\item covise.config.cycloop-2+1cluster-vrc: Cykloop, passive stereo, cluster of 3 computers
 
\end{itemize}

\subsection{4-sided CAVE with active stereo}
\begin{covimg}{display}{ruscube_sketch}{Sketch of the RUS CUBE}{1.0}\end{covimg}

The images for the four projectors of the 4-sided CAVE in this example are generated 
by an SGI onyx with active stereo. The video formats are custom formats. 
The aspect ratio of the resolution 
matches the one of the wall dimensions. The left, front and right wall
have height=2800mm, width=2500mm, the floor dimesion is 2800mm x 2800mm.
The video format for the walls has 992x992 pixels with a frequency of 
114 Hz, the floor has 1024x992 pixels at 114 Hz.

If only two pipes are used for the cave, the managed area on the
wall-wall-pipe is 992x992, on the wall-bottom pipe 1024x992.
If all four pipes are used for the cave, the managed area resolution is
equal to the window resolution.


\subsection{Tilted workbench with active stereo}

The workbench is operated by an SGI Onyx with one pipe.
We configured a 
combination of two video channels each with the format 1024\_768\_96s. 
The managed area is 2048x768 with medium pixel depth. 
To the first video output we connected a normal monitor, to the second 
one the projector. Therefore the COVER window has its origin at [1024 0]
(see section WindowConfig).
 
The dimension of the screen is 1700 mm x  1270 mm, and the screen is 
tilted 40 degrees.

\begin{covimg}{display}{rusworkbench_sketch}{Sketch of the RUS Workbench}{0.9}\end{covimg}


\subsection{Cykloop with passive stereo}
\begin{covimg}{display}{cykloop_sketch}{Sketch of the Vircinity Cycloop}{1.0}\end{covimg}

The cycloop has two projectors, the passive stereo images can be either generated by one
computer with two video channels or by two synchronised computers. In case of one
computer the configuration of one large window with two viewports is  preferable, because
textures are loaded only once.
