\begin{htmlonly}
\documentclass{covise}

\usepackage{html, htmllist}
\usepackage{color}
\usepackage{graphicx}
\usepackage{longtable}
\usepackage{palatino}
\usepackage{picins}
\usepackage[colorlinks,dvips]{hyperref}
	  
\bodytext{TEXT=#000000 BGCOLOR=#FFFFFF LINK=#0033CC VLINK=#0033CC}

% #1  mark defined by \label
% #2  a linktext 
% #3  a html link 
\newenvironment{covlink}[3]%
{\html{\htmladdnormallink{#1}{#3}}\latex{\hyperref[#1]{#2} (\ref{#1})}%
}

\newenvironment{covimg}[4]%
{ \html{\htmladdimg[ALIGN=CENTER]{#2.gif}}
 
 \latexonly
 \begin{figure}[!Hhtp]
  \begin{center}
   \includegraphics[scale=#4]{#1/#2}
   \caption{#3}
  \end{center}
 \end{figure}
 \endlatexonly
}



\definecolor{output}{rgb}{0.,0.,1.}
\definecolor{depend}{rgb}{1.,0.65,0.}
\definecolor{required}{rgb}{0.58,0.,0.83}
\definecolor{optional}{rgb}{0.,0.39,0.}

\end{htmlonly}

%================================================================================
%================================================================================


%================================================================================
%\startdocument

\chapter{Einf\"uhrung}
Die Rei\ss festigkeit von Tissue h\"angt von sehr vielen Einfl\"ussen ab, zum
Beispiel vom Material des Tissue, Abnutzungsgrad der Pr\"agewalze, Druck beim
Pr\"agen und Pr\"agemuster. Oft werden Festigkeitsverluste erst
festgestellt, wenn die Walze schon gepr\"agt und das Tissue im 
Einsatz ist. Das Ziel des SCA E-Type Projekts ist es, durch Simulation 
den Verlust an Festigkeit schon w\"ahrend des Entwurfs zu analysieren.
Dabei sollten interaktiv neue Designs getestet werden und die Parameter
beim Pr\"agevorgang variiert werden k\"onnen.

Bisher wird das Muster vom Designer im PDF-Format zum Walzenhersteller
geschickt, der das Muster in eine CAD-Zeichnung umsetzt. 
Wenn sich das Muster als Walze nicht fertigen l\"asst, wird es an den Designer
zur\"uckgeschickt und ge\"andert. Aufgrund der CAD-Zeichnung wird die Walze 
gefr\"ast und damit Tissue gepr\"agt. Spezialisten pr\"ufen die Haptik des Tissue. 
Danach wird das Tissue in Serie produziert. Durch die Abnutzung der Walze
\"andern sich mit der Zeit die Eigenschaften des Tissue. 

SCA hat in diesem Projekt zusammen mit der Universit\"at der Bundeswehr
Hamburg und VirCinity einen virtuellen
Prototyp f\"ur gepr\"agtes Tissue entwickelt, mit dem man sowohl das Design
ver\"andern  als auch die Auswirkung des Designs auf die physikalischen
Eigenschaften analysieren kann. Durch die Integration in die Virtual
Reality Umgebung COVER kann man das Tissue r\"aumlich und beinahe fotorealistisch
anschauen. Mit einem dreidimensionalen Eingabeger\"at kann das Tissue
intuitiv bewegt werden und das Muster direkt ge\"andert werden. Eine Simulation
der Pr\"agung mit dem Programm LS-Dyna liefert die plastische Verformung des
Tissue nach dem Pr\"agen. Auf das verformte Tissue wird dann eine 
Zugfestigkeitssimulation  mit Ansys angeschlossen und mit COVISE visualisiert.
Die Festigkeitswerte werden in der Virtual Reality Umgebung farblich
angezeigt und der maximale Verlust als Zahl ausgegeben (noch nicht implementiert). 
Durch die Integration
aller Schritte in COVISE und die Bedienung \"uber die Virtual Reality
Schnittstelle muss der Benutzer kein Berechnungsspezialist sein, sondern
lediglich mit dem Prozess Tissue-Pr\"agung vertraut sein.

Im Kapitel 2 werden die 
Proze\ss schritte Design, Pr\"agungssimulation, Rei\ss festigkeitssimulation und 
Visualisierung beschrieben und der Programmablauf erl\"autert.\newline 

In Kapitel 3 wird die Virtual Reality Benutzerschnittstelle vorgestellt.\newline

Der Anhang gibt eine Einf\"uhrung in das f\"ur die Pr\"agesimulation verwendete
Verformungsmodell (gek\"urzte und \"ubersetzte Fassung eines internen 
Berichts von Prof. Dr.-Ing. R. Lammering et al., Universit\"at der Bundeswehr
Hamburg).


 
