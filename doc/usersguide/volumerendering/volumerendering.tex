
\begin{htmlonly}


\usepackage{html, htmllist}
\usepackage{longtable}

\bodytext{bgcolor="#ffffff" link="#0033cc" vlink="#0033cc"}

%%%==================================================	
%%%==================================================	

% #1  mark defined by \label
% #2  a linktext 
% #3  a html link 
\newcommand{covlink}[3]{\htmladdnormallink{#2}{#3} \latex{(\ref{#1})} }


\newenvironment{covimg}[4]%
{
 \begin{figure}[htp]
  \begin{center}
   \latexonly
      \includegraphics[scale=#4]{#1/pict/#2}
   \endlatexonly  
   \html{\htmladdimg[align="center"]{pict/#2.png}}
   \caption{#3}
  \end{center}
 \end{figure} 
}{} 

\newenvironment{covimg2}[3]%
{ 
 \begin{figure}[htp]
  \begin{center}
     \latexonly
       \includegraphics[scale=#3]{#1/pict/#2}   
     \endlatexonly
     \html{\htmladdimg[align="center"]{pict/#2.png}}
  \end{center}
 \end{figure} 
}{}

\definecolor{output}{rgb}{0.,0.,1.}
\definecolor{depend}{rgb}{1.,0.65,0.}
\definecolor{required}{rgb}{0.58,0.,0.83}
\definecolor{optional}{rgb}{0.,0.39,0.}

\newcommand{\addimage}[1] {\html{\htmladdimg{pict/#1.png}}}

\newcommand{\addpict}[4] {\latexonly
	     \begin{figure}[!htbp]
			  \begin{center}
   	 		  \includegraphics[scale=#1]{#2}
   	 		  \caption{#3}
		 		  \label{#4}
			  \end{center}
	 		\end{figure}
	     \endlatexonly}



\newcommand{\mapeditor}{\textbf{Map Editor}}
\newcommand{\covise}{\textbf{COVISE}}

    
\end{htmlonly}

%=============================================================
%=============================================================


%=============================================================
\startdocument
\chapter{Volume Rendering in COVISE}
\label{Volume_Rendering}
%=============================================================

For volumetric scalar data, in addition to cutting planes and iso-surfaces, COVISE offers a direct volume 
rendering method based on texture hardware. This technique displays entire volume datasets. 
Transfer functions are used to determine the visual appearance of the datasets. The volume rendering
functionality of COVISE was originally developed as part of the project VIRVO (Virtual Reality Volume Rendering)
at HLRS.

{\bf Please note:}

The {\bf Volume Rendering} function has been provided as a {\bf preversion} together with COVISE version 5.2.
You gain additional functionality but you might encounter minor problems.

\section{Volume Rendering Basics}

This chapter will give some basic information about volume rendering in COVISE. It will describe what
types of volume data can be processed and how the user can display them.

\subsection{Transfer Functions}

Since scalar volumetric data represent a solid 3D block of data values, the user needs a way to look 
inside of the block. A simple way to look inside is to clip the block along a plane, a more sophisticated 
method is to assign opacities to the data values. Opacity is the opposite of
transparency: the higher the opacity, the lower the transparency. The assignment of opacity values to 
data values is defined by a transfer function. In addition to the transfer function for opacity, there is a 
transfer function that assigns colors to the scalar values. In Figure 5.1, both transfer functions are
depicted: the opacity function is drawn as a line, the color function is represented as a color gradient.
On the desktop, a ColorEdit module serves as a transfer function editor,
the VR renderer COVER has its own built-in editor which comes with the Volume plugin and is displayed
as soon as volume data is loaded.

\begin{covimg}{volumerendering}{transferfunction}{Opacity and color transfer functions}{0.7}\end{covimg}
\begin{htmlonly}
Figure 5.1: Opacity and color transfer functions
\vspace{0.5cm}
\end{htmlonly}

\subsection{Rendering Technique}

Depending on the available graphics hardware, either 2D or 3D textures are used for displaying the volume data. 
If only 2D textures are supported, three sets of textures have to be stored in texture memory, one for each 
principal axis. For 3D textures the volume data only needs to be stored once. For technical reasons, on SGI 
machines each voxel occupies at least two bytes of texture memory, even if only 8 bits per voxel are stored.

The volumes are displayed by drawing them slice by slice (see Figure 5.2). 
The number of slices drawn determines the rendering speed: 
the faster the graphics hardware, the more slices can be drawn and the higher the image quality. The 
slices are always oriented in a way that their normal vectors point towards the user, so that the user never 
looks in-between slices. The number of slices that can be drawn at interactive rates depends on
the size of the volume object on screen. This is due to the pixel fill rate being the limiting factor.

\begin{covimg}{volumerendering}{slicing}{Slicing approach for texture based volume rendering}{0.7}\end{covimg}
\begin{htmlonly}
Figure 5.2: Slicing approach for texture based volume rendering
\vspace{0.5cm}
\end{htmlonly}

\section{COVISE Modules}

In order to work with volumetric data, a dataset that is compatible to the volume rendering subsystem needs 
to be created. Compatible data must be located on a cartesian grid, which means that the coordinate axes must 
be perpendicular to each other, and the data values must be distributed equally on each coordinate axis. 
There can be different sample distances on each coordinate axis, but not within an axis. If the
source data is not on a cartesian grid, it has to be resampled with the appropriate COVISE modules. 

The total size of texture memory required for rendering can be computed by multiplying the number of voxels 
in each dimension with one another and with the number of byes per voxel. For example, a 16 bit per voxel dataset 
with 256 x 256 x 256 voxels requires 256x256x256x2 bytes = 32 megabytes of texture memory. If the data does not 
fit entirely into texture memory, it either has to be swapped in and out, which is time consuming, or it does 
not load at all. In the latter case, a white volume dataset is displayed.

Volume data in COVISE is internally represented as one of the following data types:

\begin{itemize}
\item 8 bit per voxel scalar data
\item 16 bit per voxel scalar data (usually only the most significant 12 bits can be displayed by the graphics hardware) 
\item 24 bit per voxel RGB data: 8 bit are stored for each color component 
\item 32 bit per voxel RGB+scalar data: the scalar value is used as a look-up into the opacity transfer function 
and for rendering, the color components are multiplied by the resulting value Module ReadVolume
\end{itemize}

In COVISE, Volume data can either be computed at runtime, or it can be read from disk using the 
module ReadVolume. This module reads standardized VIRVO volume files, as well as sequences of 2D slice images. 
Volume files can be created by the module WriteVolume.

\subsection{Module ReadVolume}

Figure 5.3 shows a typical COVISE network to read a volume file and display it in the renderer. 
It accepts several volume data types, and it can load a series of 2D images and merge 
them to a volume dataset. Files types are distinguished exclusively by the suffixes of their file names.

\begin{covimg}{volumerendering}{readvolume-net}{Map with ReadVolume to read volume datasets from disk}{0.7}\end{covimg}
\begin{htmlonly}
Figure 5.3: Map with ReadVolume to read volume datasets from disk
\vspace{0.5cm}
\end{htmlonly}

The following file types are supported by the module ReadVolume:

\begin{table}
\begin{tabular}{|c|l|}
\hline 
{\bf File Extension} & {\bf Description} \\
rvf & Raw Volume File \\ %
xvf & Extended Volume File \\ %
avf & ASCII Volume File \\ %
tif, tiff & 3D TIF File (2D TIFF not supported) \\ %
dat & Raw volume data (no header) - automatic format detection \\ %
rgb & RGB image file (SGI 8 bit grayscale only) \\ %
pgm & Portable Graymap file (P5 binary only) \\ %
ppm & Portable Pixmap file (P6 binary only) \\ 
\hline
\end{tabular}
\end{table}

Figure 5.4 shows the ReadVolume Preferences window. The source file name must be set at the FilePath
entry. If CustomSize is unchecked, the size entries are ignored and default values or the values
from the respective volume file are used. Otherwise, the volume size will be
set as entered in VolumeWidth, -Height, and -Depth.

\begin{covimg}{volumerendering}{readvolume-pref}{ReadVolume Preferences window}{0.7}\end{covimg}
\begin{htmlonly}
Figure 5.4: ReadVolume Preferences window
\vspace{0.5cm}
\end{htmlonly}

ReadVolume can create a volume from a number of 2D slice images (RGB, PGM, or PPM files). 
To do so, the slice images have to be numbered ascendingly, 
for instance IMAGE001.RGB, IMAGE002.RGB, IMAGE003.RGB, etc. The first file name has
to be entered as the source path in the ReadVolume Preferences window. 
Then, on execution, the module loads all slice images and creates a volume dataset from them. 

\subsection{Module WriteVolume}

Figure 5.5 shows an example COVISE network to write volume data from GenDat. 

\begin{covimg}{volumerendering}{writevolume-net}{Map with WriteVolume to write volume datasets to disk}{0.7}\end{covimg}
\begin{htmlonly}
Figure 5.5: Map with WriteVolume to write volume datasets to disk
\vspace{0.5cm}
\end{htmlonly}

The following file types are supported by WriteVolume:

\begin{table}
\begin{tabular}{|c|l|}
\hline {\bf File Extension} & {\bf Description} \\
rvf & Raw Volume File \\ %
xvf & Extended Volume File \\ %
dat & Raw volume data (no header) - automatic format detection \\ %
pgm, ppm & Density or RGB images (depending on volume data type) \\ 
\hline
\end{tabular}
\end{table}

Figure 5.6 shows the preferences window of the module WriteVolume. The FileName entry expects the destination file name.
If OverwriteExisting is checked, the destination file will be overwritten, if it previously existed.
The file type and data format can be selected with the respective choice menus.
MinimumValue und MaximumValue allow to constrain the stored data range. All values that are smaller
or equal to MinimumValue will become zero, values greater or equal to MaximumValue will become
255 or 65535, depending on the data format (8 or 16 bit per voxel). The remaining values are 
distributed linearly inbetween.

\begin{covimg}{volumerendering}{writevolume-pref}{WriteVolume Preferences window}{0.7}\end{covimg}
\begin{htmlonly}
Figure 5.6: WriteVolume Preferences window
\vspace{0.5cm}
\end{htmlonly}


\section{Desktop Renderer}

COVISE's desktop renderer displays volume data after they had been classified with the Color Editor module. 
A simple module layout can be created with the GenDat module (see Figure 5.7) as a uniform grid generator, 
when volume rendering compatible parameters (see Figure 5.8) are used. 
Both a uniform grid and scalar data are needed as data sources for volume data. 
The Color Editor (see Figure 5.9) acts as a transfer function editor. 
In order to get a volume display with semi-transparencies, the Transparency checkbox must be checked. 
The Color Editor module converts incoming scalar values to RGBA tuples, which are then passed on to the Collect module. 
The Collect module combines grid and data value information and feeds them into the renderer (see Figure 5.10). 

\begin{covimg}{volumerendering}{gendat-net}{Simple volume rendering map with GenDat}{0.7}\end{covimg}
\begin{htmlonly}
Figure 5.7: Simple volume rendering map with GenDat
\vspace{0.5cm}
\end{htmlonly}

\begin{covimg}{volumerendering}{gendat-parameters_small}{GenDat parameters suitable for volume rendering}{0.7}\end{covimg}
\begin{htmlonly}
Figure 5.8: GenDat parameters suitable for volume rendering
\vspace{0.5cm}
\end{htmlonly}

\clearpage

\begin{covimg}{volumerendering}{gendat-coloredit}{ColorEdit's color editor window}{0.7}\end{covimg}
\begin{htmlonly}
Figure 5.9: ColorEdit's color editor window
\vspace{0.5cm}
\end{htmlonly}

\begin{covimg}{volumerendering}{gendat-output}{Volume rendering output of simple GenDat application}{0.7}\end{covimg}
\begin{htmlonly}
Figure 5.10: Volume rendering output of simple GenDat application 
\vspace{0.5cm}
\end{htmlonly}

In the renderer window, the volume object is just another COVISE data object. If both volume data 
and traditional data are displayed, occlusion artifacts may occur. 
For this case, the renderer menu offers several types of transparency sorting. 

The desktop renderer offers a special draw style for volume data: while the data is rotated with the mouse,
the volume is drawn in a lower quality to speed up the drawing process, and when the mouse button is
released, the volume is drawn in regular quality. 
This draw mode ("move low volume") can be enabled in the pop-up menu which appears 
when the right mouse button is pressed in the renderer window.

The quality of the static volume display can be set in the renderer's 
Preferences window (sampling rate, see encircled area in Figure 5.11). 
The Preferences window can be accessed from the renderer window's pop-up.

\begin{covimg}{volumerendering}{inventor-pref}{Desktop renderer's Preferences for volume quality}{0.7}\end{covimg}
\begin{htmlonly}
Figure 5.11: Desktop renderer's Preferences for volume quality
\vspace{0.5cm}
\end{htmlonly}

\clearpage

\section{VR Renderer}

In order to work with volume data in the VR renderer COVER, the volume plugin must be loaded by adding the following line 
to the COVERConfig section of the covise.config file:

\begin{verbatim}
MODULE VolumePlugin
\end{verbatim}

COVER's volume rendering capabilities can be accessed by the Volume menu (see Figure 5.12), 
which appears in the COVER main menu (see Figure 5.13) when the volume plugin was 
successfully loaded at startup. Its topmost entry is Files, which opens another 
menu with a selection of volume files. The file selection can be defined in the covise.config file in the VolumeFiles 
scope. Each line represents a file entry, consisting of a file path and a display name. These
files can be created with the COVISE module WriteVolume, the supported file types are the same as
for ReadVolume.

\begin{covimg}{volumerendering}{volume-menu}{COVER Volume menu with Animation sub-menu}{0.7}\end{covimg}
\begin{htmlonly}
Figure 5.12: COVER Volume menu with Animation sub-menu
\vspace{0.5cm}
\end{htmlonly}

\begin{covimg}{volumerendering}{cover-menu}{The COVER main menu}{0.7}\end{covimg}
\begin{htmlonly}
Figure 5.13: The COVER main menu
\vspace{0.5cm}
\end{htmlonly}

\clearpage

The Probe Mode checkbox toggles a mode in which only a cubic sub-volume is displayed (see Figure 5.16), 
which can be dragged around with the left mouse button and the size of which can be adjusted by twisting the mouse. 
Due to a smaller displayed region, a higher image quality is gained in the sub-volume.

A clipping plane can be enabled by the Clipping Plane entry. By default the plane clips off the data on one 
side of the plane (see Figure 5.14), but it displays an opaque plane at the clipping location if Opaque Clipping 
is enabled (see Figure 5.15). To move the clipping plane, it has to be turned off and on again. The Clipping Plane
checkbox enables or disables both clipping plane modes, the Opaque Clipping checkbox defines only the clipping type used.

With the Frame Rate slider, the Volume menu allows to set the rendering speed, which in turn affects display quality. 

\begin{covimg}{volumerendering}{lambda-clip}{Lambda dataset with clipping plane}{0.7}\end{covimg}
\begin{htmlonly}
Figure 5.14: Lambda dataset with clipping plane
\vspace{0.5cm}
\end{htmlonly}

\begin{covimg}{volumerendering}{lambda-oclip}{Lambda dataset with opaque clipping plane}{0.7}\end{covimg}
\begin{htmlonly}
Figure 5.15: Lambda dataset with opaque clipping plane 
\vspace{0.5cm}
\end{htmlonly}

\begin{covimg}{volumerendering}{lambda-probe}{Lambda dataset in probe mode}{0.7}\end{covimg}
\begin{htmlonly}
Figure 5.16: Lambda dataset in probe mode
\vspace{0.5cm}
\end{htmlonly}

\clearpage

Another menu entry toggles a boundary box (see Figure 5.17) around the dataset.
Yet another entry toggles data value interpolation (see Figure 5.18). By default,
three-linear interpolation is used (if supported by the graphics hardware), when the interpolation
is off, nearest neighbor interpolation is used for the volume display.

\begin{covimg}{volumerendering}{lambda-boundaries}{Lambda dataset with wireframe boundary box}{0.7}\end{covimg}
\begin{htmlonly}
Figure 5.17: Lambda dataset with wireframe boundary box
\vspace{0.5cm}
\end{htmlonly}

\begin{covimg}{volumerendering}{lambda-interpol}{Lambda dataset without trilinear interpolation}{0.7}\end{covimg}
\begin{htmlonly}
Figure 5.18: Lambda dataset without trilinear interpolation
\vspace{0.5cm}
\end{htmlonly}

When the Discrete Colors knob is set to zero, the Discrete Colors mode is turned off, and continuous
color gradients are used for the transfer function. If the knob is set to a non-zero value, only as many different 
colors are used for the color transfer function as selected.

The Animation menu can be used to control the display of time dependent datasets. If only a single time step
is loaded, this menu has no effect. When Animate is checked, the time steps are cycled at the value
selected by Speed. When the Speed slider is in the middle of its range, the speed is zero. The animation runs backwards
if the slider is in the left half of its range. Step Forward and Backward can be used to switch to the next or 
previous time step respectively. The Frame slider can be used to directly access a specific time step.

When the Save Volume menu item is clicked on, the currently loaded volume dataset, together with the 
current transfer function, is stored to the file "virvo-saved.xvf", which is located in the directory which
was current when COVER was started. The file can be read with ReadVolume, but only without the transfer function.
When it is loaded directly from COVER's Volume Plugin by entering it into the list of files for the Files menu entry 
(in covise.config), the saved transfer function will be restored.

As soon as a volume is loaded from the menu, the transfer function editor window pops up (see Figure 5.19). 
The transfer function 
for opacity can be combined by a number of different elements: a tent function, a ramp, and an alpha blank. 
In the transfer function window, the elements can be accessed by Pins, which are represented as vertical lines. 
These lines can be moved horizontally. For each scalar value, the maximum value of the alpha function's 
components define the current transfer function. Alpha blanks dominate over all other alpha Pins, they set the 
alpha value to transparency no matter what other elements are located at the same position. 

\begin{covimg}{volumerendering}{tfe-commented}{Transfer function editor in three different interaction states}{0.7}\end{covimg}
\begin{htmlonly}
Figure 5.19: Transfer function editor in three different interaction states
\vspace{0.5cm}
\end{htmlonly}

Since in some cases the volume data display in virtual environments may be poor (see Figure 5.20) due 
to large image sizes, 
stereo projection and multiple screens, a high quality rendering mode (see  Figure 5.21) was implemented. 
The user can switch 
to that display mode by clicking a mouse button while the mouse is located above the user's head. Then the volume 
is displayed with full detail, but the frame rate usually drops. No interaction with menus is
possible, until the user leaves this mode with another mouse click at an arbitrary position.

\begin{covimg}{volumerendering}{lambda-lowq}{Low quality Lambda dataset}{0.7}\end{covimg}
\begin{htmlonly}
Figure 5.20: Low quality Lambda dataset
\vspace{0.5cm}
\end{htmlonly}

\begin{covimg}{volumerendering}{lambda-hiq}{High quality Lambda dataset}{0.7}\end{covimg}
\begin{htmlonly}
Figure 5.21: High quality Lambda dataset
\vspace{0.5cm}
\end{htmlonly}

\clearpage

\section{Volume file types}

{\bf DAT: Pure volume data file}

This file type stores raw volume data without a header. The data can contain 1,2,3, or 4 byte per voxel.
When loading a file of this type, the program tries to find the volume dimensions automatically. If this
doesn't work, you can help by adding the volume size to the file name prefix, 
for instance "cthead256x256x64.dat" for a 256 x 256 x 64 voxels dataset. 

The order of voxels in the file is: start at top-left-front, go right first, then down, then back 
(just like the order of letters in a book).
All bytes of each voxel are stored consecutively, beginning with the most
significant byte for 8 and 16 bit per voxel files, or in RGB(A) order for 24 and 32 bit per voxel files.
DAT files can only store one time step.

{\bf RVF: Raw volume file}

This format can easily be created by hand from any voxel data array on disk 
by adding the appropriate header: 3 x 2 Bytes (big endian) for the volume's width, height, 
and depth in voxels. This can be done with a hex editor, for example. 
The header of a 256x128x127 volume would be (hex values): 01 00 00 80 00 7F 
The volume data can only have 8 bit per voxel in RVF files, and only one time step can be stored. 
The data order is the same as in DAT files.

{\bf XVF: Extended volume file}

This format can store more information than DAT and RVF, but it is still easy enough to describe and to 
create manually. XVF files can store multiple volume datasets (time steps) in one file, 
and the storage of a random number of transfer functions. 8 to 32 bit can be stored per voxel.
In order to create an XVF file manually, it is important to know that the byte order of integer
values is big endian (most significant first), floating point values are stored in big endian
mode and 4 byte IEEE standard. In this standard, the hexadecimal representation of 1.0 is: 3F 80 00 00. 
Here is the XVF header specification:

{\bf XVF Header:}



\begin{longtable}{|p{2.5cm}|p{3.5cm}|p{8cm}|}
\hline
   {\bf Length} & {\bf Data Type} & {\bf Description} \endhead
\hline\hline
	9 bytes & char &  file ID string: "VIRVO-XVF" \\
\hline
	2 bytes & unsigned short & offset to beginning of data area, from top of file [bytes] \\
\hline
	2 x 4 bytes & unsigned int &  width and height of volume [voxels] \\
\hline
	4 bytes & unsigned int &  number of slices per time step \\
\hline
	4 bytes & unsigned int &  number of frames in volume animation (time steps) \\
\hline
	1 byte & unsigned char &  bits per voxel (supported values: 8, 16, 24, 32) \\	 
\hline
	3 x 4 bytes & float & real world voxel size (width, height, depth) [mm] \\
\hline
	4 bytes & float & length of a time step in the volume animation [seconds] \\
\hline
  2 x 4 bytes & float & physical data range covered by voxel data (minimum, maximum) \\
\hline
  3 x 4 bytes & float & real world location of volume center (x,y,z) [mm] \\
\hline
  1 byte & unsigned char & compression type (0=none, not supported yet) \\
\hline
	2 bytes & unsigned short &  number of transfer functions \\
\hline
	2 bytes & unsigned short &  type of transfer functions: \newline
                                    0 = 4 x 256 Byte for RGBA channels (deprecated) \newline
                                    1 = list of control pins \\

\hline
\end{longtable} 

{\bf Data area:}
 
The data starts at the position "offset to beginning of data area" (see table). 
The voxel order is similar to DAT and RVF files, all bytes of each voxel are stored consecutively. 
If multiple time steps are stored, they follow one by one, with no separator inbetween.

{\bf Transfer functions:}

The transfer functions are stored at the end of the file, right after the data area.
Transfer functions should not be added to the file manually, this should only be done from within COVISE (currently this
is only supported by the "save volume" function in COVER). Therefore, the format of 
transfer functions will not be described here. To create a volume file manually, it is 
sufficient to set the number of transfer functions to zero in the header.

{\bf AVF: ASCII volume file}

AVF files are ASCII representations of volume data. They consist of a header and a data section:

{\bf Header:} 

In the header, several lines give information about the data format. Each line consists of an 
identifier and a value, separated by whitespace.
Each line can contain one identifier and one value. This file format cannot store transfer functions. 
Anywhere in the file, comments starting with '\#' are allowed. This comments out all the rest of the line.

The following abbreviations are used: 
\begin{verbatim}
<int> = integer values
<float> = floating point values
<OPT1|OPT2|OPT3> = list of options
\end{verbatim}

The following lines are required:

\begin{longtable}{|l|l|}
\hline
  WIDTH [int] & width of the volume [voxels]  \\
\hline
  HEIGHT [int] & height of the volume [voxels]  \\ 
\hline
  SLICES [int] &  number of slices in the volume [voxels]  \\
\hline
\end{longtable} 

The following lines are optional. If they are missing, the respective default values are used:

\begin{longtable}{|p{3.5cm}|p{10.5cm}|}
\hline
     	FRAMES [int]   	& number of data sets contained in the file (default: 1) \\                
\hline
 	MIN [float]	& minimum data value, smaller values are constrained to this value (default: 0.0) \\		
\hline
	MAX [float]	& maximum data value, larger values are constrained to this value (default: 1.0) \\
\hline
	FORMAT [SCALAR8/16 or RGB(A)] 
	& voxel data format (default: SCALAR8): \newline 
          SCALAR8 = scalar values, to be stored as 8 bit integers \newline 
          SCALAR16 = scalar values, to be stored as 16 bit integers \newline 
          RGB = color values, consisting of a red, a green, and a blue color component,
          stored as 3x8 bit \newline
          RGBA = color values, consisting of a red, a green, a blue, and an opacity
          (alpha) value, stored as 4x8 bit \\
\hline
	XDIST [float]	& the sample distance in x direction (width) [mm] (default: 1.0) \\
\hline
	YDIST [float]	& the sample distance in y direction (height) [mm] (default: 1.0) \\
\hline
	ZDIST [float]	& the sample distance in z direction (slices) [mm] (default: 1.0) \\
\hline
	TIME [float]	& the length of each time step for transient data [s] (default: 1.0) \\
\hline
\end{longtable}

{\bf Data area:} 

The data area begins right after the header. The voxel data values are listed, separated by whitespace 
and/or end-of-line markers. Both floating point and integer values are accepted. The voxel order is similar to DAT, 
RVF, and XVF files. All elements of each voxel are stored consecutively.

Sample file:
\begin{verbatim}
WIDTH 4
HEIGHT 3
SLICES 2
FRAMES 1
MIN 0.0
MAX 1.0
FORMAT SCALAR8 # 8 bit data
XDIST 1.0
YDIST 1.0
ZDIST 1.0
TIME 1.0
0.9 0.9 0.9 0.9
0.9 0.2 0.3 0.9
0.9 0.2 0.4 0.9
0.8 0.8 0.8 0.8
0.8 0.1 0.1 0.8
0.8 0.0 0.0 0.8 
\end{verbatim}

\section{Acknowledgments}

The lambda function dataset used in some images is courtesy of Ulrich Rist, IAG, University of Stuttgart.





