\begin{htmlonly}
\documentclass{covise}

\usepackage{html, htmllist}
\usepackage{color}
\usepackage{graphicx}
\usepackage{longtable}
\usepackage{palatino}
\usepackage{picins}
\usepackage[colorlinks,dvips]{hyperref}
	  
\bodytext{TEXT=#000000 BGCOLOR=#FFFFFF LINK=#0033CC VLINK=#0033CC}

% #1  mark defined by \label
% #2  a linktext 
% #3  a html link 
\newenvironment{covlink}[3]%
{\html{\htmladdnormallink{#1}{#3}}\latex{\hyperref[#1]{#2} (\ref{#1})}%
}

\newenvironment{covimg}[4]%
{ \html{\htmladdimg[ALIGN=CENTER]{#2.gif}}
 
 \latexonly
 \begin{figure}[!Hhtp]
  \begin{center}
   \includegraphics[scale=#4]{#1/#2}
   \caption{#3}
  \end{center}
 \end{figure}
 \endlatexonly
}



\definecolor{output}{rgb}{0.,0.,1.}
\definecolor{depend}{rgb}{1.,0.65,0.}
\definecolor{required}{rgb}{0.58,0.,0.83}
\definecolor{optional}{rgb}{0.,0.39,0.}

\end{htmlonly}

%================================================================================
%================================================================================


%================================================================================
%\startdocument

\chapter{Introduction}

The tensile strength of tissue depends on a lot of influences, e.g. on the material 
of the tissue, the wear of the press cylinder, the embossing pressure, and the 
embossing pattern. Frequently loss of tensile strength is detected after the cylinder 
has been manufactured and the tissue is already in use. The SCA E-Type project is 
targeted on analyzing this loss of tensile strength by simulation already during 
design. The intention is to interactively test new designs and to be able to vary 
parameters for embossing.

Until now the designer sends his pattern as a .pdf-file to the manufacturer of the 
cylinder, who transforms this pattern into a CAD drawing. Based on this CAD drawing 
the cylinder is milled, and tissue is embossed using this cylinder. Specialists examine 
the haptics of the tissue, and then mass production of the tissue starts. Due to the 
wear of the cylinder the characteristics of the tissue change over time.

In this project, SCA together with the University of the Federal Armed Forces Hamburg 
and VirCinity have developed a virtual 
prototype for embossed tissue; with this prototype it will be possible to modify the 
design as well as to analyse how the design influences the physical properties. 
Due to the integration in the Virtual Reality environment COVER the user gets a threedimensional and 
almost photorealistic view of the tissue. With a 3D input device he can move the 
tissue intuitively, and change the pattern directly. Simulation of the embossing using 
the LS-Dyna program gives him the  plastic deformation of the tissue after embossing. 
Then he applies, using Ansys, a simulated traction to the tissue, visualizes the results with COVISE. The values for tensile strength are 
shown in color in the Virtual Reality environment, and the maximum loss will be 
shown as number (only partly implemented). The integration of all steps in COVISE and the Virtual Reality user interface 
does not require a specialist for computation, but a professional for the process of tissue embossing.

Chapter 2 describes the process steps for design, simulation of embossing, simulation 
of traction, and visualization, and  explains the flow of control.

Chapter 3 introduces the Virtual Reality interface.

The Appendix provides an introduction into the deformation model used to simulate
embossing (Short version of an internal report by Prof.Dr.-Ing. R. Lammering et
al., University of the Federal Armed Forces Hamburg).


 
