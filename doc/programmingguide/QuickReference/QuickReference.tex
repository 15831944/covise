
\begin{htmlonly}

\usepackage{html, htmllist}
\usepackage{longtable}

\bodytext{bgcolor="#ffffff" link="#0033cc" vlink="#0033cc"}

%%%==================================================	
%%%==================================================	

% #1  mark defined by \label
% #2  a linktext 
% #3  a html link 
\newcommand{covlink}[3]{\htmladdnormallink{#2}{#3} \latex{(\ref{#1})} }


\newenvironment{covimg}[4]%
{
 \begin{figure}[htp]
  \begin{center}
   \latexonly
      \includegraphics[scale=#4]{#1/pict/#2}
   \endlatexonly  
   \html{\htmladdimg[align="center"]{pict/#2.png}}
   \caption{#3}
  \end{center}
 \end{figure} 
}{} 

\newenvironment{covimg2}[3]%
{ 
 \begin{figure}[htp]
  \begin{center}
     \latexonly
       \includegraphics[scale=#3]{#1/pict/#2}   
     \endlatexonly
     \html{\htmladdimg[align="center"]{pict/#2.png}}
  \end{center}
 \end{figure} 
}{}

\definecolor{output}{rgb}{0.,0.,1.}
\definecolor{depend}{rgb}{1.,0.65,0.}
\definecolor{required}{rgb}{0.58,0.,0.83}
\definecolor{optional}{rgb}{0.,0.39,0.}

\newcommand{\addimage}[1] {\html{\htmladdimg{pict/#1.png}}}

\newcommand{\addpict}[4] {\latexonly
	     \begin{figure}[!htbp]
			  \begin{center}
   	 		  \includegraphics[scale=#1]{#2}
   	 		  \caption{#3}
		 		  \label{#4}
			  \end{center}
	 		\end{figure}
	     \endlatexonly}



\end{htmlonly}




%=============================================================
%=============================================================


%=============================================================
\startdocument
\chapter{Quick Reference}
\label{QuickReference}
%=============================================================


\section{coModule}

{\bf Base class for all module programming}

\begin{verbatim}
coModule(const char *description);
virtual ~coModule();
 
virtual void start(int argc, char *argv[]);
virtual int compute(const char*);
virtual void param(const char *paramName);
virtual void sockData(int sockNo);
virtual void quit(void*);
virtual void postInst();
virtual void mainLoop();

coBooleanParam *addBooleanParam(const char *name, const char *desc);
coFileBrowserParam *addFileBrowserParam(const char *name, const char *desc);
coChoiceParam *addChoiceParam(const char *name, const char *desc);
coFloatParam *addFloatParam(const char *name, 
                                        const char *desc);
coFloatSliderParam *addFloatSliderParam(const char *name, 
                                        const char *desc);
coFloatVectorParam *addFloatVectorParam(const char *name, 
                                        const char *desc);
coInt32Param *addInt32Param(const char *name, 
                                    const char *desc);
coIntSliderParam *addIntSliderParam(const char *name, 
                                    const char *desc);
coIntVectorParam *addInt32VectorParam(const char *name, 
                                    const char *desc);
coStringParam *addStringParam(const char *name, const char *desc);
coInputPort *addInPort(const char *name, const char *types, 
                                      const char *desc);
coOutputPort *addOutputPort(const char *name, const char *types, 
                                        const char *desc);

coChoiceParam *paraSwitch(const char *name, const char *desc);
int paraEndSwitch();
int paraCase(const char *name);
int paraEndCase();

static void sendError(const char *format, \dots);
static void sendWarning(const char *format, \dots);
static void sendInfo(const char *format, \dots);

void selfExec();

void addSocket(int socket); 
void removeSocket(int socket); 
\end{verbatim}
 

\section{coSimpleModule}

{\bf This class overloads coModule to automatically recurse through 
coDoSet hierarchies.}

\begin{verbatim}
class coSimpleModule : public coModule

void setComputeTimesteps(const int v); 
void setComputeMultiblock(const int v); 
void copyAttributes(coDistributedObject *tgt, coDistributedObject *src); 
void setCopyAttributes(const int v);
\end{verbatim}
 

\section{coSimLib}

{\bf This class overloads coModule to automatically recurse through coDoSet 
hierarchies.}

\begin{verbatim}
class coSimLib : public coModule

coSimLib(const char *moduleName);
int setTargetHost(const char *hostname);
int setLocalHost(const char *hostname);
int setUserArg(int num, const char *data);
int startSim();
int serverMode();
int isConnected();
int recvData(void *buffer, size_t length);  
int sendData(const void *buffer, size_t length);
int coParallelInit(int numParts, int numPorts);
int coParallelPort(const char *portName, int isCellData);
int setParaMap(int isCell,int isFortran, int nodeNo, int length, 
               int32 *nodeMap);
\end{verbatim}


\section{Simlib client}

{\bf This is the client-side connectivity for a coSimLib module: These are 
C, not C++ routines, with additional FORTRAN77 language binding, which have
the same parameters but 6-char names. The file coSimClient.o from the 
covise/\$ARCH/bin  directory must be linked to the simulation.}

\begin{verbatim}
int coNotConnected()                                      F77: CONOCO

/* Logic send/receive calls ******/
 
int coGetParaSlider(const char *name,                     F77: COGPSL
                    float *min, float *max, float *val);
int coGetParaFloatScalar(const char *name, float *data);  F77: COGPFL
int coGetParaIntScalar(const char *name, int *data);F77:  F77: COGPIN
int coGetParaChoice(const char *name, int *data);         F77: COGPCH
int coGetParaBool(const char *name, int *data);           F77: COGPBO
int coGetParaText(const char *name, char *data);          F77: COGPTX
int coGetParaFile(const char *name, int *data);           F77: COGPFI
int coSend1Data(const char *portName,                     F77: COSU1D
                int numElem, float *data);
int coSend3Data(const char *portName,                     F77: COSU3D
                int numElem, float *data);
int coExecModule();                                       F77: COEXEC
int coFinished();                                         F77: COFINI
int coParallelInit(int numParts, int numPorts);           F77: COPAIN
int coParallelPort(const char *portname, int isCellData); F77: COPAPO
int coParallelCellMap(int node, int numCells,             F77: COPACM
                      const int *localToGlobal);
int coParallelVertexMap(int node, int numCells,           F77: COPAVM
                        const int *localToGlobal);
int coParallelNode(int node);                             F77: COPANO
int sendData(const void *buffer, size_t length);          F77: COSEND 
int recvData(void *buffer, size_t length);                F77: CORECV
int getVerboseLevel();
                                                          F77: COVERB
\end{verbatim}


\section{coUifElem}

{\bf Base class for all ports, Parameters and switch groups nodes}

\begin{verbatim}
enum Kind { SWITCH, PARAM, INPORT, OUTPORT };
virtual void hide();
virtual void show();
virtual Kind kind();
virtual const char *getName();
\end{verbatim}


\section{coPort}

{\bf Base class for In-, Out- and Parameter ports}

\begin{verbatim}
class coPort : public coUifElem

virtual const char *getDesc() const;
\end{verbatim}

 

\section{coInputPort}

{\bf Input data port}

\begin{verbatim}
class coInputPort : public coPort

void setRequired(int isRequired);
coDistributedObject *getCurrentObject();
\end{verbatim}
 

\section{coOutputPort}

{\bf Output data port}

\begin{verbatim}
class coOutputPort : public coPort
 
void setRequired(int isRequired);
void setObj(coDistributedObject *obj);
coDistributedObject *getObj();
const char *getObjName();
\end{verbatim}


\section{coUifPara} 

{\bf Base class for all parameter ports}

\begin{verbatim}
class coUifPara : public coPort

void setImmediate(int isImmediate);
void setActive(int isActive);
virtual void hide();
virtual void show();
void enable();
void disable();
int isActive() const;
\end{verbatim}

 

\section{coBooleanParam} 

{\bf Boolean value parameter}

\begin{verbatim}
class coBooleanParam: public coUifPara
 
int setValue(int value);
int getValue() const;
\end{verbatim}

 

\section{coFileBrowserParam}

{\bf File browser parameter}

\begin{verbatim}
class coFileBrowserParam : public coUifPara
 
int setValue(const char *path, const char *value);
const char *getValue() const;
\end{verbatim}

 

\section{coChoiceParam}

{\bf Parameter to choose values from a list}

\begin{verbatim}
class coChoiceParam : public coUifPara

int setValue(int numChoices, const char * const* choice, 
             int actChoice);

int updateValue(int numChoices, const char * const* choice, 
             int actChoice);
	     
int setValue(int actChoice);
int getValue() const;
\end{verbatim}


\section{coFloatParam}

{\bf single float parameter}

\begin{verbatim}
class coFloatParam : public coUifPara

int setValue(float val);
float getValue() const;
\end{verbatim}
 

\section{coFloatSliderParam}

{\bf float slider parameter}

\begin{verbatim}
class coFloatSliderParam : public coUifPara
 
int setValue(float min, float max, float value);
int setMin(float min);
int setMax(float max);
int setValue(float value);
void getValue(float &min, float &max, float &value) const;
float getMin() const;
float getMax() const;
float getValue() const;
\end{verbatim}
 

\section{coFloatVectorParam}

{\bf Multiple float parameters}

\begin{verbatim}
class coFloatVectorParam : public coUifPara
 
int setValue(int pos, float data);
int setValue(int size, const float *data);
int setValue(float data0, float data1, float data2);
float getValue(int pos) const;
int getValue(float &data0, float &data1, float &data2) const;
\end{verbatim}
 

\section{coInt32Param}

{\bf single integer parameter}

\begin{verbatim}
class coInt32Param : public coUifPara

int setValue(int val);
int getValue() const;
\end{verbatim}
 

\section{coIntSliderParam}

{\bf Integer slider parameter}

\begin{verbatim}
class coIntSliderParam : public coUifPara

int setValue(int min, int max, int value);
int setMin(int min);
int setMax(int max);
int setValue(int value);
 
void getValue(int &min, int &max, int &value) const;
int getMin() const;
int getMax() const;
int getValue() const;
\end{verbatim}


\section{coIntVectorParam}

{\bf Parameter for multiple integers}

\begin{verbatim}
class coIntVectorParam : public coUifPara
 
int setValue(int pos, int data);
int setValue(int size, const int *data);
int setValue(int data0, int data1, int data2);
int getValue(int pos) const;
\end{verbatim}


\section{coStringParam}

{\bf Implements string parameters}

\begin{verbatim}
class coStringParam : public coUifPara

int setValue(const char *val);
const char *getValue() const;
\end{verbatim}
 

\section{coDistributedObject}

{\bf Base class for all data objects}
 
\begin{verbatim}
void setAttribute(const char *, const char *);
void setAttributes(int, const char **, const char **);
const char *getAttribute(const char *);
int getAllAttributes(const char ***name,
                     const char ***content);
const char *getName() { return name; }
const char *getType() const;
int isType(const char *reqType);
char objectOk() const;
char *getName();
char *getType();
\end{verbatim}


\section{coDoUniformGrid}

\begin{verbatim} 
coDoUniformGrid(const char *name, int x, int y, int z,
               float xmin, float xmax, float ymin,
               float ymax, float zmin, float zmax);
void getGridSize(int *x, int *y, int *z);    
void getPointCoordinates(int i, float *x_c,
                           int j, float *y_c, 
                           int k, float *z_c);
void getDelta(float *dx, float *dy, float *dz);
void getMinMax(float *xmin, float *xmax,
                 float *ymin, float *ymax,
                 float *zmin, float *zmax);   
\end{verbatim}
 

\section{coDoRectilinearGrid}

\begin{verbatim} 
coDoRectilinearGrid(const char *name, int x, int y, int z);
coDoRectilinearGrid(const char *name, int x, int y, int z,
                   float *xc, float *yc, float *zc);
void getGridSize(int *x, int *y, int *z);
void getPointCoordinates(int i, float *x_c,
                           int j, float *y_c, 
                           int k, float *z_c);
void getAddresses(float **x_c, float **y_c, float **z_c);
\end{verbatim}
         

\section{coDoStructuredGrid}

\begin{verbatim} 
coDoStructuredGrid(const char *name, int x, int y, int z);
coDoStructuredGrid(const char *name, int x, int y, int z,
                     float *xc, float *yc, float *zc);
void getGridSize(int *x, int *y, int *z);
void getPointcoordinates(int i, float *x_c,
                           int j, float *y_c, 
                           int k, float *z_c);
void getAddresses(float **x_c, float **y_c, float **z_c);
\end{verbatim}
 

\section{coDoUnstructuredGrid} 

\begin{verbatim} 
coDoUnstructuredGrid(const char *name, 
                    int nelem, int nconn, int ncoord, int ht);
coDoUnstructuredGrid(const char *name, 
                    int nelem, int nconn, int ncoord, 
                    int *el, int *cl, 
                    float *xc, float *yc, float *zc, 
                    int *tl);
void getGridSize(int *e, int *c, int *p);
void getAddresses(int **elem, int **conn, 
                  float **x_c, float **y_c, float **z_c);
void getTypeList(int **l);
int hasTypeList();
void getNeighborList(int *n, int **l, int **li);
\end{verbatim}


\section{coDoFloat}
 
\begin{verbatim}
coDoFloat(const char *n, int no, float *s);
coDoFloat(const char *n, int no);
 
int getNumPoints();
void getPointValue(int no, float *s);
void getAddress(float **data);
\end{verbatim}
 

\section{coDoVec2}
 
\begin{verbatim}
coDoVec2(const char *n, int no, float *s, float *t);
coDoVec2(const char *n, int no);

int getNumPoints();    // returns gridsize
void getPointValue(int no, float *s, float *t);
void getAddresses(float **s_d, float **t_d);
\end{verbatim}
 

\section{coDoVec3}
 
\begin{verbatim}
coDoVec3(const char *n, int no,
                        float *xc, float *yc, float *zc);
coDoVec3(const char *n, int no);
 
int getNumPoints();
void getPointValue(int no, float *s);
void getAddresses(float **u_v, float **v_v, float **w_v);      
\end{verbatim}
          

\section{coDoTensor}
 
\begin{verbatim}
coDoTensor(const char *n, int no,
                        float *data, TensorType ttype);
coDoTensor(const char *n, int no, TensorType ttype);
 
int getNumPoints();
coDoTensor::TensorType getTensorType();
void getPointValue(int no, float *s);
void getAddress(float **value);      
\end{verbatim}
          

\section{coDoRGBA}
 
\begin{verbatim}
coDoRGBA(const char *n, int no, int *pc);
coDoRGBA(const char *n, int no);
 
int getNumElements();    // returns gridsize
void getElementValue(int no, int *s);
void getAddress(int **pc);                 
 
int setFloatRGBA(int pos, float r, float g, float b, float a = 1.0);
int setIntRGBA(int pos, int r, int g, int b, int a = 255);
int getFloatRGBA(int pos, float *r, float *g, float *b, float *a);
int getIntRGBA(int pos, int *r, int *g, int *b, int *a);
\end{verbatim}
 

\section{coDoGeometry}

\begin{verbatim} 
coDoGeometry(const char *n, coDistributedObject *geo);
 
void setColor(int cattr, coDistributedObject *c);
void setNormal(int nattr, coDistributedObject *n);
void setTexture(int tattr, coDistributedObject *t) ;
      
coDistributedObject *getGeometry();
coDistributedObject *getColors();
coDistributedObject *getNormals();
coDistributedObject *getTexture();
 
int getGeometryType();
int getColorAttributes();
int getNormalAttributes();
int getTextureAttributes();
 
void setColorAttributes(int cattr);
void setNormalAttributes(int nattr);
void setTextureAttributes(int nattr);
\end{verbatim}


\section{coDoPoints} 

\begin{verbatim}
coDoPoints(char *n, int no);
coDoPoints(char *n, int no,
          float *x, float *y, float *z);
 
int getNumPoints();
void getPointCoordinates(int no, float *xc, float *yc, float *zc);
void getAddresses(float **x_c, float **y_c, float **z_c);
\end{verbatim}


\section{coDoLines}

\begin{verbatim}
coDoLines(char *n, int no_p, int no_v, int no_l);
coDoLines(char *n, 
         int no_p, float *x_c, float *y_c, float *z_c, 
         int no_v, int *v_l, int no_l, int *l_l);
 
int getNumLines();
int getNumVertices();
int getNumPoints();
void getAddresses(float **x_c, float **y_c, float **z_c, 
                  int **v_l, int **l_l);
\end{verbatim}


\section{coDoPolygons}

\begin{verbatim}
coDoPolygons(char *n, int no_p, int no_v, int no_l);
coDoPolygons(char *n, 
            int no_p, float *x_c, float *y_c, float *z_c, 
            int no_v, int *v_l, int no_pol, int *pol_l);
 
int getNumPolygons();
int getNumVertices();
int getNumPoints();
void getNeighborList(int *n, int **l, int **li);      
void getAddresses(float **x_c, float **y_c, float **z_c, 
                  int **v_l, int **l_l);
\end{verbatim}


\section{coDoTriangleStrips}

\begin{verbatim}
coDoTriangleStrips(char *n, int no_p, int no_v, int no_l);
coDoTriangleStrips(char *n, 
                  int no_p,float *x_c,float *y_c, float *z_c, 
                  int no_v, int *v_l, int no_pol, int *pol_l);
 
int getNumStrips();
int getNumVertices();
int getNumPoints();
void getAddresses(float **x_c, float **y_c, float **z_c, 
                  int **v_l, int **l_l) ;
\end{verbatim}


\section{coDoTexture}
 
\begin{verbatim}
coDoTexture(const char *name,coDoPixelImage* texture,
           int border, int components, int level,
           int numVertices, int* vertexIndex,
           int numCoord, float** coords)
  
coDoPixelImage* getBuffer();
int            getBorder();
int            getComponents();
int            getLevel();
int            getWidth();
int            getHeight();
int*           getVertices();
int            getNumCoordinates();
float**        getCoordinates();
\end{verbatim}


\section{coDoPixelImage}

\begin{verbatim}
coDoPixelImage(const char *name, int width, int height, 
              unsigned format, short pixelsize, const char **buffer);
coDoPixelImage(const char *name, int width, int height, 
              unsigned format, short pixelsize, const char *buffer);
coDoPixelImage(const char *name, int width, int height, 
              unsigned format, short pixelsize);
 
int getWidth();
int getHeight();
short getPixelsize();
unsigned getFormat();
char* getPixels(); 
char& operator() (int x, int y);
char& operator[] (int i);
\end{verbatim}


\section{coDoText}
 
\begin{verbatim}
coDoText(const char *n, int s);
coDoText(const char *n, int s, const char *d);
int getTextLength();
void getAddress(char **base);
\end{verbatim}


\section{coDoIntArr}

\begin{verbatim} 
coDoIntArr(const char *objName, int numDim, const int *dimArray,
          const int *initdata=NULL);
int getNumDimensions();
int getDimension(int i);
int getSize();
int *getAddress();
void getAddress(int **res);
\end{verbatim}


\section{coDoSet}

\begin{verbatim} 
coDoSet(const char *n, coDistributedObject * const *elem);
coDistributedObject * const *getAllElements(int *no);
coDistributedObject *getElement(int no);
\end{verbatim}


\section{coCoviseConfig}

{\bf This class makes the COVISE config-file accessible for the module.}

\begin{verbatim}
#include <config/CoviseConfig.h>

static const char *coCoviseConfig::getEntry(const char *entry);
static const char *coCoviseConfig::getEntry(const char *variable, const char *entry);
static int coCoviseConfig::getInt(const char *entry, int defaultValue, bool *exists=NULL);
static int coCoviseConfig::getInt(const char *variable, const char *entry, int defaultValue, bool *exists=NULL);
static long coCoviseConfig::getLong(const char *entry, long defaultValue, bool *exists=NULL);
static long coCoviseConfig::getLong(const char *variable, const char *entry, long defaultValue, bool *exists=NULL);
static float coCoviseConfig::getFloat(const char *entry, long defaultValue, float *exists=NULL);
static float coCoviseConfig::getFloat(const char *variable, const char *entry, float defaultValue, bool *exists=NULL);
static bool coCoviseConfig::isOn(const char *entry, bool defaultValue, float *exists=NULL);
static bool coCoviseConfig::isOn(const char *variable, const char *entry, bool defaultValue, bool *exists=NULL);
static char *coCoviseConfig::getScopeEntry(const char *scope, char *name);
static char **coCoviseConfig::getScopeEntries(const char *scope);
static char **coCoviseConfig::getScopeEntries(const char *scope, const char *name);
\end{verbatim}
