
\begin{htmlonly}

\usepackage{html, htmllist}
\usepackage{longtable}

\bodytext{bgcolor="#ffffff" link="#0033cc" vlink="#0033cc"}

%%%==================================================	
%%%==================================================	

% #1  mark defined by \label
% #2  a linktext 
% #3  a html link 
\newcommand{covlink}[3]{\htmladdnormallink{#2}{#3} \latex{(\ref{#1})} }


\newenvironment{covimg}[4]%
{
 \begin{figure}[htp]
  \begin{center}
   \latexonly
      \includegraphics[scale=#4]{#1/pict/#2}
   \endlatexonly  
   \html{\htmladdimg[align="center"]{pict/#2.png}}
   \caption{#3}
  \end{center}
 \end{figure} 
}{} 

\newenvironment{covimg2}[3]%
{ 
 \begin{figure}[htp]
  \begin{center}
     \latexonly
       \includegraphics[scale=#3]{#1/pict/#2}   
     \endlatexonly
     \html{\htmladdimg[align="center"]{pict/#2.png}}
  \end{center}
 \end{figure} 
}{}

\definecolor{output}{rgb}{0.,0.,1.}
\definecolor{depend}{rgb}{1.,0.65,0.}
\definecolor{required}{rgb}{0.58,0.,0.83}
\definecolor{optional}{rgb}{0.,0.39,0.}

\newcommand{\addimage}[1] {\html{\htmladdimg{pict/#1.png}}}

\newcommand{\addpict}[4] {\latexonly
	     \begin{figure}[!htbp]
			  \begin{center}
   	 		  \includegraphics[scale=#1]{#2}
   	 		  \caption{#3}
		 		  \label{#4}
			  \end{center}
	 		\end{figure}
	     \endlatexonly}



\end{htmlonly}



%=============================================================
\startdocument
\chapter{Basics}
\label{Basics}
%=============================================================
\index{Basics}


\section{Introduction}

This documentation describes how to integrate new application modules into COVISE.
There is also a section describing how to write plugins for the VR renderer.
The programming functionality to interface and communicate with COVISE is explained 
in detail.

Currently, each COVISE module is realized as an operating system process.
Future versions may allow to create modules which are modeled as subroutines or
threads in a single process environment.

\section{Prerequisites}
\latexonly
\index{Prerequisites}
\endlatexonly

To create new COVISE modules, the COVISE development distribution with all the 
necessary header files and libraries has to be installed on the target platform. 
For UNIX and similar systems, this includes also a set of scripts for Bourne
Shell compatible shells
to set up your compile environment, especially the environment variables necessary
for compiling COVISE.
Start a bash by typing \texttt{bash} (if this is not your default shell), change to the
COVISE top level directory and type \texttt{source .covise.sh}.
If your current working directory is not the COVISE top level directory
and if your home directory does not contain the covise top level directory then you have to pass the
location of this directory as a parameter to this script.

When compiling modules, COVISE automatically creates different directories for 
platform-specific parts like object files and executables: object files are generated 
in the objects\_\$ARCHSUFFIX subdirectory of your source (\$COVISEDIR) directory, while binaries are put in 
\$COVISEDIR/\$ARCHSUFFIX/bin, where \$ARCHSUFFIX is a platform identifier.
Refer to
\$COVISEDIR/README-ARCHSUFFIX.txt for a full list. Here are some examples:

\begin{longtable}{|p{2.5cm}|p{12cm}|}
\hline
   {\bf ARCH} & {\bf Platform}  \endhead
\hline\hline
   rhel6 & RedHat Enterprise Linux 6.x or compatible (CentOS, Scientific Linux) \\
\hline
	win32 &  Microsoft Windows using the Visual Studio 2003 compiler \\
\hline
	lion &  Apple Mac OS X 10.7 (x86\_64)\\
\hline
\end{longtable}
{\bf Table 1: Architecture suffixes}
\index{Architecture suffix}

If you want to put your enhancements into a different directory, you can set
the environment variable \$COVISEDESTDIR to a directory where the added
binaries should go. You should also add this directory to your \$COVISE\_PATH
environment variable.

If you are going to develop COVISE modules, you might want to switch on ``developer
features'' in the Mapeditor settings dialog. This will enable you to restart a
module from its context menu or add an option to start a module in a debugger
to the context menu of modules in the module library.

There are following requirements before starting to program:

\begin{itemize}

   \item A \texttt{CMakeLists.txt} file \index{CMake} used by CMake to generate a Makefile \index{Makefile} or
compile instructions for other build environments: examples are available with the example modules, 
which reside in \verb+$COVISEDIR/src/application/Examples+.
You can also use the template in \verb+$COVISEDIR/src/template/module+ as a
starting point.

\item A directory for the source code of the module is required.

\item The {\tt Hello} module in {\tt \$COVISEDIR/src/application/Examples/Hello} defines 
the complete basic structure of a module. It is recommended to copy this module source 
code as the base for own developments.

\end{itemize}

For users of the make build system having sourced the .covise.sh setup file,
compiling the module is as simple as typing \texttt{make} in the module source directory.

COVISE has an object-oriented architecture requiring static initialization of 
multiple classes. For proper initialization, the main body of each module has to be 
written in C++. Nevertheless it is possible to integrate FORTRAN and C routines into 
an application module. 

Each module is executed as a separate operating system process.
Modules communicate in general 
using TCP/IP socket connections.
This applies within a machine as well as between 
machines.
Data exchange is handled differently. Within a machine pointers to shared 
memory are used  to avoid copying of data objects. Between machines data objects are 
exchanged via TCP/IP socket connections. All this is handled transparently  for  
the application programmer.
The COVISE API (Application Programming Interface) is 
accessible through several libraries, which have to be linked to each new module. 
This happens automatically, if you specify \texttt{covise\_add\_module} as done
in the CMakeLists.txt for the module template.
\index{cmake!template}

The main COVISE library is called {\tt libcoCore.so}
({\tt .sl} on HP, \texttt{.dll}/\texttt{.lib} on Windows, \texttt{.dylib} on Mac OS X).
It is used for data management, message communication, starting processes, etc.
\index{Libraries!coCore}

The Application library, called {\tt libcoAppl.so}, contains the basic functionality to 
write application programs. It hides the details of packing and receiving COVISE 
messages and provides a framework for structuring application modules.
\index{Libraries!coAppl}

The library {\tt libcoApi.so} builds on these libraries and provides the
Application Programming Interface, which makes programming of applications
easier and less error-prone.
Whenever possible, programmers are recommended to use the higher level API
functions instead of the direct COVISE calls provided with the Application library.
\index{Libraries!coApi}

\section{Data flow model}
\latexonly
\index{Data flow}
\endlatexonly

COVISE sessions are organized as a collection of modules, which are connected in a 
strictly unidirectional data flow network.

A typical data flow network is shown in the next figure.  Data and control flows from top to 
bottom. Loops are not allowed, nevertheless, there are possibilities to send feedback 
messages from later to earlier modules in the processing chain.

\begin{covimg}{Basics}{Module_and_Dataflow_network}{Module and Dataflow network}{0.7}\
\end{covimg}
\begin{htmlonly}
Figure: Module and Dataflow network
\vspace*{1cm}
\end{htmlonly}

Each module in the pipeline communicates with the central controller
(the process named \texttt{covise}) and a local 
data-manager (named \texttt{crb}) by sending or receiving control messages via a TCP/IP socket. 
Data flow between modules is only a visual metaphor. Within a machine data objects are 
stored in shared memory segments. They are accessible from different modules by mapping
their storage into the address space of the respective module processes. Between 
machines, objects are transferred by COVISE request brokers (CRBs), including necessary 
format conversions. This is transparent for the modules accessing the data objects. 
Both data and control communication are completely hidden within the
COVISE libraries.

\section{Execution sequence and module states}
\latexonly
\index{Execution sequence} 
\index{module states}
\endlatexonly

The sequence of states a COVISE module can have is shown in the following figure.

\begin{covimg}{Basics}{Module_execution_scheme}{Module execution scheme}{0.7}\
\end{covimg}
\begin{htmlonly}
Figure: Module execution scheme
\vspace*{1cm}
\end{htmlonly}

The start-up of a module \index{coModule!start-up} is divided into two parts: The Constructor creates the module 
layout, which is sent to the COVISE controller using the module start message.
In order to avoid timeout conditions during running the constructor,
time consuming initialization operations should be put into an additional user subroutine
which is called after establishing the COVISE connectivity to allow user-supplied start-up routines to 
be performed without timeout problems. After the start-up sequence, COVISE enters a 
main event loop, which is only left for the module shut-down. In the main loop, COVISE 
can receive different events, which are then handled by one of the following event 
handlers:

\begin{itemize}

\item {\it param}: Every change of a module parameter submits a message 
to the module. The \texttt{param} subroutine reacts on these change messages.

\item {\it compute}: COVISE tells the module to read its input data and create output data. 
Only within this routine data objects in shared memory can be read or new data objects 
in shared memory can be created.

\item {\it sockData}: The user can register open network sockets to be monitored. Whenever 
data on one of these sockets arrives, sockData is called.

\end{itemize}

It is important to note that several commands are only allowed when COVISE is in the 
appropriate state. Due to performance reasons these states are not generally checked, 
thus calling functions during illegal states may result in module crashes.

