
\begin{htmlonly}

\usepackage{html, htmllist}
\usepackage{longtable}

\bodytext{bgcolor="#ffffff" link="#0033cc" vlink="#0033cc"}

%%%==================================================	
%%%==================================================	

% #1  mark defined by \label
% #2  a linktext 
% #3  a html link 
\newcommand{covlink}[3]{\htmladdnormallink{#2}{#3} \latex{(\ref{#1})} }


\newenvironment{covimg}[4]%
{
 \begin{figure}[htp]
  \begin{center}
   \latexonly
      \includegraphics[scale=#4]{#1/pict/#2}
   \endlatexonly  
   \html{\htmladdimg[align="center"]{pict/#2.png}}
   \caption{#3}
  \end{center}
 \end{figure} 
}{} 

\newenvironment{covimg2}[3]%
{ 
 \begin{figure}[htp]
  \begin{center}
     \latexonly
       \includegraphics[scale=#3]{#1/pict/#2}   
     \endlatexonly
     \html{\htmladdimg[align="center"]{pict/#2.png}}
  \end{center}
 \end{figure} 
}{}

\definecolor{output}{rgb}{0.,0.,1.}
\definecolor{depend}{rgb}{1.,0.65,0.}
\definecolor{required}{rgb}{0.58,0.,0.83}
\definecolor{optional}{rgb}{0.,0.39,0.}

\newcommand{\addimage}[1] {\html{\htmladdimg{pict/#1.png}}}

\newcommand{\addpict}[4] {\latexonly
	     \begin{figure}[!htbp]
			  \begin{center}
   	 		  \includegraphics[scale=#1]{#2}
   	 		  \caption{#3}
		 		  \label{#4}
			  \end{center}
	 		\end{figure}
	     \endlatexonly}



\end{htmlonly}



%=============================================================
\startdocument
\chapter{OpenCOVER Plugin Programming }
\label{COVERPlugin}
%=============================================================

\index{COVER|see{OpenCOVER}}



The virtual reality Renderer module OpenCOVER \index{OpenCOVER!introduction}is a COVISE
module with support for virtual reality (VR) input devices such as Polhemus FASTRAK,
Ascension Motionstar or A.R.T. tracking devices and backprojection displays with
stereo projection like CAVE or workbench.
OpenCOVER can also be started independently of COVISE and be used as a 
virtual reality viewer for 3D geometry. 

The functionality of OpenCOVER can be extended through \emph{plugins}. 
A plugin \index{Plugin Interface!Introduction}is a 
dynamic library with a certain interface. During execution,
OpenCOVER executes the functions provided by each such library.

Plugins can either be used to implement interaction with other COVISE modules 
(also called "feedback"), for example for steering a simulation module, or for 
extending the functionality of OpenCOVER as a VR viewer.

The plugin programmer needs to derive from the class \code{coVRPlugin} and reimplement
some of its virtual methods as well as its constructor and destructor.
These virtual methods are called at the appropriate times while OpenCOVER is running.
This class has to be advertised to OpenCOVER with the macro \code{COVERPLUGIN}.

A simple skeleton for an OpenCOVER plugin is available in
\code{covise/src/template/plugin}.


\section{Background on Dynamic Libraries}
\latexonly
\index{OpenCOVER!Dynamic Libraries}
\endlatexonly

In UNIX systems \code{dlopen} makes a dynamic library available to a running process. 
\code{dlsym} returns the address of a symbol in the dynamic library (returns a
pointer to a function in the dynamic library), \code{dlerror} returns diagnostic 
information and \code{dlclose} closes the dynamic library. 
The class \code{coVRDynLib} encapsulates this in a platform independent way.

If OpenCOVER notices that it has to open a plugin it creates an instance of the 
class \code{coVRPlugin}. The class \code{coVRPlugin} handles a plugin in OpenCOVER. It
has virtual methods which correspond to the functions in a plugin, e.~g.\ \code{preFrame}, 
\code{postFrame}, \code{addObject}, \code{addInteractor}, \code{removeObject},~\ldots
Then OpenCOVER opens the dynamic library with 

\begin{verbatim}
      handle=dlopen(name, mode)
\end{verbatim}

and extracts the function with the name \code{coVRPluginInit} with 

\begin{verbatim}
      dlsym(handle, "coVRPluginInit") 
\end{verbatim}

which creates a new instance of \code{coVRPlugin}.
All this is handled by the \code{COVERPLUGIN} macro.

During execution of OpenCOVER for each loaded plugin the methods are called at the 
appropriate time, e.~g.\ \code{coVRPlugin::preFrame} is executed before OpenCOVER 
calls \code{osgViewer::Viewer::frame} and for each plugin \code{coVRPlugin::postFrame}
is executed after.


\section{OpenSceneGraph}
\latexonly
\index{OpenCOVER!OpenSceneGraph}
\endlatexonly

OpenCOVER uses the scene graph library OpenSceneGraph. The plugin programmer should be 
familiar with the basic concepts of OpenSceneGraph such as the scene graph with its different
types of nodes, the rendering loop and its main concepts. 
You can consult the OpenSceneGraph Quickstart Guide by Paul Martz
(available as hardcopya or a free download from \url{http://www.lulu.com/content/767629})
and the OpenSceneGraph Doxygen documentation (available online at
\url{http://www.openscenegraph.org/projects/osg/wiki/Support/ReferenceGuides})
for more information.

OpenCOVER can be configured to use multiple threads for rendering.
%for multi-pipe application. The default multiprocessing 
%model is PFMP\_APP\_CULL\_DRAW. For single pipe applications the
%multiprocessing model can be set to PFMP\_APPCULLDRAW in the section CoverConfig 
%in the file covise.config (you find more information in the
%online help about OpenCOVER). 

%TODO
The scene graph of OpenCOVER which is relevant to the plugin programmer looks like 
this:

\begin{covimg}{COVERPlugin}
     {Scene_Graph}{OpenCOVER scene graph}{0.7}\end{covimg}
\begin{htmlonly}
Figure: OpenCOVER scene graph
\end{htmlonly}
\vspace*{1cm}

The matrix of handTransform comes from the 3D input device of the hand e.~g.\ the 
Polhemus Stylus or the Ascension 6DOF Mouse. Under handTransform the interactor 
geometry for the hand is located. This is for e.~g.\ laser sword or the 
magnifying glass. For the action "xform" the matrix of the handTransform is applied 
also to the objectXformTransform, for the action "scale" only the x component of the 
position is used to compute a scale matrix and is applied to the objectScaleTransform. 
COVISE geometry and other 3D geometry (in IV, OBJ etc. format) is appended 
under the group node \code{cover->getObjectsRoot()}.


\section{OpenCOVER as a COVISE module}
\latexonly
\index{OpenCOVER!OpenCOVER as a COVISE module}
\endlatexonly

As a COVISE module, OpenCOVER loads the COVISE plugin and uses the library \emph{coAppl} for interfacing to COVISE. 
This library is similar to the newer \emph{coApi} (which is
used by non-renderer modules) and provides functions to open a socket connection 
to the controller and data manager and to create COVISE data objects. OpenCOVER can 
also be used stand-alone. Depending on the number of command line parameters the program 
decides if a connection to COVISE is needed. 

As OpenSceneGraph has its own main loop, OpenCOVER checks for COVISE "ADD OBJECT" 
or "REMOVE OBJECT" messages from COVISE every frame. If it receives an object,
OpenCOVER creates the appropriate OpenSceneGraph objects and appends them to the scene 
graph under the node \code{cover->getObjectsRoot()}. If feedback information was
appended to the object (see \ref{COVER:module}), OpenCOVER loads the appropriate
plugin automatically.

Section \ref{COVER:covise} describes the methods your plugin can implement for
interfacing with COVISE.


\section{OpenCOVER as a VR viewer for 3D geometry}

If OpenCOVER is started outside of COVISE, it interprets the argument as a file which 
should be loaded. If you want to load a plugin, this has to be specified
in your COVISE configuration in the section \code{COVER.Plugin}.
A minimal configuration file to enable OpenCOVER's Vrml97 plugin would look like this:
\begin{verbatim}
<?xml version="1.0"?>

<COCONFIG version="1" >
  <GLOBAL>
    <COVER>
      <Plugin>
        <Vrml97 value="on" />
      </Plugin>
    </COVER>
  </GLOBAL>
</COCONFIG>
\end{verbatim}


\section{The Plugin Interface}

During execution OpenCOVER calls the functions provided by the plugin at certain 
points in the rendering loop. The functions are declared in \code{cover/coVRPlugin.h}. 

None of these methods have to implemented, all are optional.
However, you should make sure that all resources acquired by your plugin are released
in its desctructor.

\subsection{Plugin Life Cycle}

The following methods are called during the different phases of the plugin life time.

\index{Plugin Interface!coVRPlugin()}
\index{Plugin Interface!constructor}
\begin{longtable}{|p{4cm}|p{2.5cm}|p{7cm}|}
\hline
\multicolumn{3}{|p{13.5cm}|}
{\bf \code{coVRPlugin()}}\\
\hline
{Description:}  
  & \multicolumn{2}{|p{9.5cm}|}
  {The plugin is constructed when it is loaded.
   There are four methods to load a plugin:
   \begin{enumerate}
   \item The plugin is specified in the configuration file.
      Then it is initialized in OpenCOVER before the main loop starts.
      In this case, the OpenCOVER scene graph is not yet fully initialized.
   \item OpenCOVER gets a COVISE object with had a \code{coFeedback} object applied.
   \item The user requests that a plugin be loaded from  Tablet User Interface.
   \item A plugin starts another plugin.
   \end{enumerate}
   %The implementation of this function is required!
   The plugin constructor should be used to do early plugin initialization that does
   not depend on OpenCOVER data structures.
  } \\
\hline
\end{longtable}


\index{Plugin Interface!init()}
\begin{longtable}{|p{4cm}|p{2.5cm}|p{7cm}|}
\hline
\multicolumn{3}{|p{13.5cm}|}
{\bf \code{bool init()}}\\
\hline
{Description:}  
  & \multicolumn{2}{|p{9.5cm}|}
  {This method is called when the plugin is loaded and when OpenCOVER has
   completed start-up, all its data structures are established.
   This is the place where you can insert objects into the scene graph,
   add new menu entries, \ldots
  } \\
\hline
\end{longtable}


\index{Plugin Interface!destroy()}
\begin{longtable}{|p{4cm}|p{2.5cm}|p{7cm}|}
\hline
\multicolumn{3}{|p{13.5cm}|}
{\bf \code{bool destroy()}}\\
\hline
{Description:}  
  & \multicolumn{2}{|p{9.5cm}|}
  {This method is called to announce that OpenCOVER wants to remove your plugin.
   Reimplement it in order to do early clean-up work.
   Return \code{false} to prevent your module from being removed.
  } \\
\hline
\end{longtable}


\index{Plugin Interface!~coVRPlugin()}
\index{Plugin Interface!destructor}
\begin{longtable}{|p{4cm}|p{2.5cm}|p{7cm}|}
\hline
\multicolumn{3}{|p{13.5cm}|}
{\bf \code{\~{}coVRPlugin()}}\\
\hline
{Description:} & \multicolumn{2}{|p{9.5cm}|}
        {This function is called before the plugin is removed from OpenCOVER.\newline
	 It is important that you remove all your menu entries,
         remove your objects from the scene graph and release all resources
         acquired by your plugin.
        }\\
\hline
\end{longtable}

\subsection{COVISE Interface}
\label{COVER:covise}

The following functions deal with COVISE data objects. A plugin can be 
informed if OpenCOVER gets \code{ADD\_OBJECT} or \code{DELETE\_OBJECT} messages:

Within OpenCOVER all COVISE data objects, i.~e.\ objects derived from
\code{coDistributedObject}, are represented as \code{RenderObjects}.
These are replicated between all nodes if OpenCOVER runs on a render cluster for e.~g.\ a CAVE.
All data and attributes are copied to the corresponding fields of the \code{RenderObject}.

\index{Plugin Interface!addObject()} 
\begin{longtable}{|p{4cm}|p{2.5cm}|p{7cm}|}
\hline
\multicolumn{3}{|p{13.5cm}|}
{\bf \code{void addObject(RenderObject *baseObj,
 RenderObject *geomObj,\newline
 RenderObject *normObj, RenderObject *colorObj, RenderObject *texObj,\newline
 const char *parentName,\newline 
 int numCol, int colorBinding, int colorPacking,
 float *r, float *g, float *b, int *packedCol,\newline 
 int numNormals, int normalBinding,
 float *xn, float *yn, float *zn, float transparency)}}\\
\hline
{Description:}  
           & \multicolumn{2}{|p{9.5cm}|}
	   {If implemented, this function is called whenever OpenCOVER 
	   receives a COVISE object.} \\
\hline
\multicolumn{1}{|r|}{IN:} & \code{baseObj} 
                          & \
 {The container object, i.~e.\ the representation of the \code{coDoSet} object the newly
    added object is part of.
 }\\
\hline
\multicolumn{1}{|r|}{IN:} & \code{geomObj} 
                          & \
 {The geometry container object consisting of a geometry object, a color object, 
 a normal object and a texture object.\newline
 geomContainer is \code{NULL} if there is no container object but only a geometry object 
 (e.~g.\ polygons without colors)}\\
\hline
\multicolumn{1}{|r|}{IN:} & \code{normObj} 
                          & \
			  {The representation of the associated normals data.}\\
\hline
\multicolumn{1}{|r|}{IN:} & \code{colorObj} 
                          & \
			  {The representation of the associated colors data.}\\
\hline
\multicolumn{1}{|r|}{IN:} & \code{texObj} 
                          & \
			  {The representation of the associated texture data.}\\
\hline
\multicolumn{1}{|r|}{IN:} & \code{parentName} 
                          & \
 {Name of the parent COVISE object, it is only set for set elements,
  for other data objects it is \code{NULL}.}\\
\hline
\multicolumn{1}{|r|}{IN:} & \code{numCol} 
                          & \
			  {Number of colors}\\
\hline
\multicolumn{1}{|r|}{IN:} & \code{colorBinding} 
                          & \
  {Color binding type: \newline
   PER\_PRIMITIVE, PER\_FACE, PER\_VERTEX, NONE}\\
\hline
\multicolumn{1}{|r|}{IN:} & \code{colorPacking} 
                          & \
			  {Non-zero if colors are packed colors
                                        instead of RGB values.}\\
\hline
\multicolumn{1}{|r|}{IN:} & \code{r} 
                          & \
			  {Red color values.}\\
\hline
\multicolumn{1}{|r|}{IN:} & \code{g} 
                          & \
			  {Green color values.}\\
\hline
\multicolumn{1}{|r|}{IN:} & \code{b} 
                          & \
			  {Blue color values.}\\
\hline
\multicolumn{1}{|r|}{IN:} & \code{packedCol:} 
                          & \
			  {Packed RGBA colors.}\\
\hline
\multicolumn{1}{|r|}{IN:} & \code{numNormals} 
                          & \
			  {Number of normals.}\\
\hline
\multicolumn{1}{|r|}{IN:} & \code{normalBinding} 
                          & \
  {Normal binding type: \newline
  PER\_PRIMITIVE, PER\_FACE, PER\_VERTEX, NONE}\\
\hline
\multicolumn{1}{|r|}{IN:} & \code{xn} 
                          & \
			  {List of normals x values.}\\
\hline
\multicolumn{1}{|r|}{IN:} & \code{yn} 
                          & \
			  {List of normal y values.}\\
\hline
\multicolumn{1}{|r|}{IN:} & \code{zn} 
                          & \
			  {List of normal z values.}\\
\hline
\multicolumn{1}{|r|}{IN:} & \code{transparency}
                          & \
			  {Transparency value.}\\
\hline
\end{longtable}

\index{Plugin Interface!removeObject}
\begin{longtable}{|p{4cm}|p{2.5cm}|p{7cm}|}
\hline
\multicolumn{3}{|p{13.5cm}|}
{\bf \code{void removeObject(const char *objName, bool replace)}}\\
\hline
{Description:}  
           & \multicolumn{2}{|p{9.5cm}|}
	   {If implemented, this function is called whenever OpenCOVER
            receives a DELETE\_OBJECT message.} \\
\hline
\multicolumn{1}{|r|}{IN:} & \code{objName} 
                          & \
	  {The name of the COVISE object to delete.}\\
\hline
\multicolumn{1}{|r|}{IN:} & \code{replace} 
                          & \
	  {\code{true}, if the COVISE object is replaced (this happens if the user 
	  a module executes again), \code{false} if it really has to be deleted 
	  (this happens, if the user removes a module or the connection between 
	  the module and OpenCOVER).}\endhead
\hline
\end{longtable}


\index{Plugin Interface!newInteractor}
\begin{longtable}{|p{4cm}|p{2.5cm}|p{7cm}|}
\hline
\multicolumn{3}{|p{13.5cm}|}
{\bf \code{void newInteractor(RenderObject *container,
coInteractor *feedback)}}\\
\hline
{Description:}  
           & \multicolumn{2}{|p{9.5cm}|}
	   {If implemented, this function is called whenever 
	   OpenCOVER receives a COVISE object with feedback 
	   information appended. } \\
\hline
\multicolumn{1}{|r|}{IN:} & \code{container} 
                          & \
  {The geometry container object consisting of a geometry object, 
  a color object, a normal object and a texture object.\newline
  \code{container} is \code{NULL} if there is no container object but only a 
  geometry object (e.~g.\ polygons without colors).}\\
\hline
\multicolumn{1}{|r|}{IN:} & \code{feedback} 
                          & \
	{The class \code{coInteractor} handles feedback in OpenCOVER.}\endhead
\hline
\end{longtable}

\index{Plugin Interface!coviseError}
\begin{longtable}{|p{4cm}|p{2.5cm}|p{7cm}|}
\hline
\multicolumn{3}{|p{13.5cm}|}
{\bf \code{void coviseError(const char *errorMsg)}}\\
\hline
{Description:}  
           & \multicolumn{2}{|p{9.5cm}|}
	   {If implemented, this function is called whenever OpenCOVER
      receives an error message. Error messages are created
		by modules with \code{sendError}. They are displayed in the Map Editor 
		message window.}\\
\hline
\multicolumn{1}{|r|}{IN:} & \code{errorMsg} 
                          & \
	  {The COVISE error message.}\endhead
\hline
\end{longtable}


\subsection{Scene Graph Management}

These functions are called when OpenCOVER appends or removes a node in the 
scene graph:

\index{Plugin Interface!addNode}
\begin{longtable}{|p{4cm}|p{2.5cm}|p{7cm}|}
\hline
\multicolumn{3}{|p{13.5cm}|}
{\bf \code{void addNode(osg::Node *node, RenderObject *obj)}}\\
\hline
{Description:}  
           & \multicolumn{2}{|p{9.5cm}|}
	   {If implemented, this function is called when OpenCOVER adds a new COVISE 
	    object to the scenegraph or if other plugins insert a node into the 
	    scene graph.
           }\\
\hline
\multicolumn{1}{|r|}{IN:} & \code{node} 
                          & \
			  {A pointer to the scene graph node.}\\
\hline
\multicolumn{1}{|r|}{IN:} & \code{obj} 
                          & \
			  {A pointer to the \code{RenderObject} generated from the COVISE object,\newline
                           \code{NULL} if not corresponding to a COVISE object}\endhead
\hline
\end{longtable}


\index{Plugin Interface!removeNode}
\begin{longtable}{|p{4cm}|p{2.5cm}|p{7cm}|}
\hline
\multicolumn{3}{|p{13.5cm}|}
{\bf \code{void removeNode(osg::Node *node)}}\\
\hline
{Description:}  
           & \multicolumn{2}{|p{9.5cm}|}
	   {If implemented, this function is called if a node
            is removed from the scene graph.
	   }\\
\hline
\multicolumn{1}{|r|}{IN:} & \code{node} 
                          & \
			  {A pointer to the scene graph node.}\\
\hline
\end{longtable}


\subsection{Render Loop}

These functions are called at certain points during the execution of OpenCOVER.

\index{Plugin Interface!preFrame}
\begin{longtable}{|p{4cm}|p{10cm}|}
\hline
\multicolumn{2}{|p{9.5cm}|}
{\bf \code{void preFrame()}}\\
\hline
{Description:}  
           & 
	   {If implemented, this function is called from the main thread immediately
              before \code{osgViewer::Viewer::frame()} is called.} \endhead
\hline
\end{longtable}

\index{Plugin Interface!postFrame}
\begin{longtable}{|p{4cm}|p{10cm}|}
\hline
\multicolumn{2}{|p{9.5cm}|}
{\bf \code{void postFrame()}}\\
\hline
{Description:}  
           & 
	   {If implemented, this function is called from the main thread immediately
              after \code{osgViewer::Viewer::frame()} was called.} \endhead
\hline
\end{longtable}


\index{Plugin Interface!preSwapBuffers}
\begin{longtable}{|p{4cm}|p{10cm}|}
\hline
\multicolumn{2}{|p{9.5cm}|}
{\bf \code{void preSwapBuffers()}}\\
\hline
{Description:}  
           & 
	   {This method is called by the draw thread immediately before GL swap buffers.}
	   \endhead
\hline
\end{longtable}


\subsection{Other Events}

All animations in OpenCOVER are synchronised. If a plugin needs to synchronise
its interaction to the global animation, it should implement 
\code{setTimestep} to become informed whenever OpenCOVER displays another timestep.

\index{Plugin Interface!setTimestep}
\begin{longtable}{|p{4cm}|p{2.5cm}|p{7cm}|}
\hline
\multicolumn{3}{|p{8cm}|}
{\bf \code{void setTimestep(int timestep)}}\\
\hline
{Description:}  
           & \multicolumn{2}{|p{9.5cm}|}
	   {If implemented, this function is called when OpenCOVER switches to
		a new timestep.} \\
\hline
\multicolumn{1}{|r|}{IN:} & \code{timestep} 
                          & \
			  {The number of the current timestep.}\endhead
\hline
\end{longtable}


Plugins can be notified when the user presses keys on its keyboard.

\index{Plugin Interface!key}
\begin{longtable}{|p{4cm}|p{2.5cm}|p{7cm}|}
\hline
\multicolumn{3}{|p{13.5cm}|}
{\bf \code{void key(int type, int keySym, int modifiers)}}\\
\hline
{Description:}  
           & \multicolumn{2}{|p{9.5cm}|}
	   {If implemented, this function is called when a key is pressed
		or released.} \\
\hline
\multicolumn{1}{|r|}{IN:} & \code{type} 
                          & \
			  {Type of event: \code{osgGA::GUIEventAdapter::KEYDOWN} or
                          \code{KEYUP}.}\\
\hline
\multicolumn{1}{|r|}{IN:} & \code{keySym} 
                          & \
			  {The key symbol, see \code{osgGA/GUIEventAdapter}.}\\
\hline			  	
\multicolumn{1}{|r|}{IN:} & \code{modifiers} 
                          & \
			  {The state of the Shift and other modifier keys,
                           also see \code{osgGA/GUIEventAdapter}.}\endhead			  		  
\hline
\end{longtable}

Plugins can send messages to each other, see \ref{COVER:messages}.

\index{Plugin Interface!message}
\begin{longtable}{|p{4cm}|p{2.5cm}|p{7cm}|}
\hline
\multicolumn{3}{|p{13.5cm}|}
{\bf \code{void message(int type, int len, const void *buf)}}\\
\hline
{Description:}  
           & \multicolumn{2}{|p{9.5cm}|}
	   {If implemented, this function is called when 
	   messages from other plugins arrive.} \\
\hline
\multicolumn{1}{|r|}{IN:} & {type} 
                          & \
			  {Integer representing the message type.}\\
\hline
\multicolumn{1}{|r|}{IN:} & {len} 
                          & \
			  {Message length.}\\
\hline
\multicolumn{1}{|r|}{IN:} & {buf} 
                          & \
			  {A pointer to the message.}\endhead
\hline
\end{longtable}



\section{Accessing OpenCOVER from a plugin}

The class \code{coVRPluginSupport} gives the plugin programmer access to the OpenCOVER scene
graph if he wants to add own geometry, or needs access to the menu to add new 
menu items, \ldots
This class is a singleton, the static instance of \code{coVRPluginSupport} named \code{cover} is
declared in \code{cover/coVRPluginSupport.h}.

Example: 
\begin{verbatim}
   osg::Group *s = cover.getScene();
\end{verbatim}

\subsection{Access to the scene graph and its transformations}
\index{coVRPluginSupport!Access to the scene graph}

With the following methods you can access parts of the OpenCOVER scene graph (see 
Figure 12: OpenCOVER Scene Graph) or get/set transformations in the osg::MatrixTransform nodes.

\begin{longtable}{|p{4cm}|p{10cm}|}
\hline
\multicolumn{2}{|p{9.5cm}|}
{\bf osg::Group* getScene();}\\
\hline
{Description:}  
           & 
	   {Returns a pointer to the scene node} \\
\hline
{Return value:}  
    & 
    {pointer to the scene node} \endhead
\hline
\end{longtable}


\begin{longtable}{|p{4cm}|p{10cm}|}
\hline
\multicolumn{2}{|p{9.5cm}|}
{\bf osg::ClipNode* getObjectsRoot();}\\
\hline
{Description:}  
           & 
	   {Returns a pointer to the group node objectsRoot} \\
\hline
{Return value:}  
    & 
    {pointer to the group node} \endhead
\hline
\end{longtable}


\begin{longtable}{|p{4cm}|p{10cm}|}
\hline
\multicolumn{2}{|p{9.5cm}|}
{\bf osg::MatrixTransform * getPointer();}\\
\hline
{Description:}  
           & 
	   {Returns a pointer to the transform node for the hand which contains
	    the transformation matrix of the 3D input device.} \\
\hline
{Return value:}  
    & 
    {pointer to the transform node for the hand/pointer device} \endhead
\hline
\end{longtable}


\begin{longtable}{|p{4cm}|p{10cm}|}
\hline
\multicolumn{2}{|p{9.5cm}|}
{\bf osg::MatrixTransform * getObjectsXform();}\\
\hline
{Description:}  
           & 
	   {Returns a pointer to the objectsXform node.} \\
\hline
{Return value:}  
    & 
    {pointer to the transform node objectsXform} \endhead
\hline
\end{longtable}


\begin{longtable}{|p{4cm}|p{10cm}|}
\hline
\multicolumn{2}{|p{9.5cm}|}
{\bf osg::MatrixTransform * getObjectsScale();}\\
\hline
{Description:}  
           & 
	   {Returns a pointer to the objectsXform node.} \\
\hline
{Return value:}  
    & 
    {pointer to the transform node objectsXform} \endhead
\hline
\end{longtable}



\begin{longtable}{|p{4cm}|p{10cm}|}
\hline
\multicolumn{2}{|p{9.5cm}|}
{\bf osg::Matrix \&getBaseMat();}\\
\hline
{Description:}  
           & 
	   {transformation matrix from object coordinates 
	   to world coordinates \newline
	   (objectScaleMat*objectsXformMat)} \\
\hline
{Return value:}  
    & 
    {transformation matrix} \endhead
\hline
\end{longtable}


\begin{longtable}{|p{4cm}|p{10cm}|}
\hline
\multicolumn{2}{|p{9.5cm}|}
{\bf osg::Matrix \&getInvBaseMat();}\\
\hline
{Description:}  
           & 
	   {transformation from world coordinates to 
	   object coordinates \newline
	   \code{1 / (objectScaleMat*objectsXformMat)}\newline
           use this cached value instead of computing it on your own} \\
\hline
{Return value:}  
    & 
    {transformation matrix} \endhead
\hline
\end{longtable}



\begin{longtable}{|p{4cm}|p{10cm}|}
\hline
\multicolumn{2}{|p{9.5cm}|}
{\bf osg::Matrix \&getPointerMat();}\\
\hline
{Description:}  
           & 
	   {get matrix of handTransform \newline
	    (same as \code{getPointer()->getMatrix()}}) \\
\hline
{Return value:}  
    & 
    {transformation matrix} \endhead
\hline
\end{longtable}


\begin{longtable}{|p{4cm}|p{10cm}|}
\hline
\multicolumn{2}{|p{9.5cm}|}
{\bf osg::Matrix \&getXformMat();}\\
\hline
{Description:}  
           & 
	   {get the matrix of objectsXform \newline
	   (same as \code{getObjectXforms()->getMatrix()})} \\	      
\hline
{Return value:}  
    & 
    {transformation matrix} \endhead
\hline
\end{longtable}


\begin{longtable}{|p{4cm}|p{2.5cm}|p{7cm}|}
\hline
\multicolumn{3}{|p{13.5cm}|}
{\bf void setXformMat(osg::Matrix \&transformMatrix);}\\
\hline
{Description:}  
           & \multicolumn{2}{|p{9.5cm}|}
	   {set the matrix of objectsXform\newline
	   (same as \code{getObjectXform()->setMatrix()})} \\
\hline
\multicolumn{1}{|r|}{IN:} & {transformMatrix} 
                          & \
			  {transformation matrix}\endhead
\hline			  
\end{longtable}


\begin{longtable}{|p{4cm}|p{10cm}|}
\hline
\multicolumn{2}{|p{9.5cm}|}
{\bf float getScale();}\\
\hline
{Description:}  
           & 
	   {get the scale factor of the scale node, scale factor 
	   is the same for all directions} \\
\hline
{Return value:}  
    & 
    {scale factor} \endhead
\hline
\end{longtable}


\begin{longtable}{|p{4cm}|p{2.5cm}|p{7cm}|}
\hline
\multicolumn{3}{|p{13.5cm}|}
{\bf void setScale(float s);}\\
\hline
{Description:}  
           & \multicolumn{2}{|p{9.5cm}|}
	   {set the scale factor of the matrix of the scale node} \\
\hline
\multicolumn{1}{|r|}{IN:} & {s} 
                          & {scale factor}\endhead
\hline
\end{longtable}


\begin{longtable}{|p{4cm}|p{10cm}|}
\hline
\multicolumn{2}{|p{9.5cm}|}
{\bf float getSceneSize();}\\
\hline
{Description:}  
           & 
	   {get the scene size defined in covise.config} \\
\hline
{Return value:}  
    & 
    {the scene size in [mm]} \endhead
\hline
\end{longtable}


\subsection{Access to the camera}
\index{coVRPluginSupport!Access to the camera}

You get the viewing matrix with:

\begin{longtable}{|p{4cm}|p{10cm}|}
\hline
\multicolumn{2}{|p{9.5cm}|}
{\bf osg::Matrix \&getViewerMat();}\\
\hline
{Description:}  
           & 
	   {get the transformation matrix of the viewer. 
	    The matrix contains the position and orientation of the 
	    users head. If headtracking is on, the viewer matrix changes 
	    every frame.} \\
\hline
{Return value:}  
           & 
           {transformation matrix} \endhead
\hline
\end{longtable}

\subsection{Loading 3D Geometry}
\index{coVRPluginSupport!Loading 3D Geometry}

The following methods simplify the loading of files into the plugin:
 
\begin{longtable}{|p{4cm}|p{2.5cm}|p{7cm}|}
\hline
\multicolumn{3}{|p{13.5cm}|}
{\bf const char *getname(const char *file);}\\
\hline
{Description:}  
           & \multicolumn{2}{|p{9.5cm}|}
	   {get the full name for the file (with absolute path)} \\
\hline
\multicolumn{1}{|r|}{IN:} & {file} 
                          & {short file name}\\
\hline
{Return value:}  
    & \multicolumn{2}{|p{9.5cm}|}{full name} \endhead
\hline
\end{longtable}
 
 
\begin{longtable}{|p{4cm}|p{2.5cm}|p{7cm}|}
\hline
\multicolumn{3}{|p{13.5cm}|}
{\bf osg::Node *loadFile(const char *file);}\\
\hline
{Description:}  
           & \multicolumn{2}{|p{9.5cm}|}
	   {load a file in any format supported by OpenSceneGraph or a VRML97 file} \\
\hline
\multicolumn{1}{|r|}{IN:} & {file} 
                          & {file name}\\
\hline
{Return value:}  
    & \multicolumn{2}{|p{9.5cm}|}{the root node of the scene graph created for this file} \endhead
\hline
\end{longtable}  
 
 
\begin{longtable}{|p{4cm}|p{2.5cm}|p{7cm}|}
\hline
\multicolumn{3}{|p{13.5cm}|}
{\bf void loadVRML(const char *url);}\\
\hline
{Description:}  
           & \multicolumn{2}{|p{9.5cm}|}
	   {Load the specified VRML file into the OpenSceneGraph scene graph 
	   and appends it under objectsRoot} \\
\hline
\multicolumn{1}{|r|}{IN:} & {url} 
                          & {url containing a vrml97 file}\endhead
\hline
\end{longtable} 
  

\begin{longtable}{|p{4cm}|p{10cm}|}
\hline
\multicolumn{2}{|p{9.5cm}|}
{\bf void reloadVRML();}\\
\hline
{Description:}  
           & 
	   {reload the previously loaded VRML file. If your VRML file 
	   is on a remote machine and you changed the model there you can 
	   update it in OpenCOVER by using this method} \endhead
\hline
\end{longtable}
 

\begin{longtable}{|p{4cm}|p{10cm}|}
\hline
\multicolumn{2}{|p{9.5cm}|}
{\bf void removeVRML();}\\
\hline
{Description:}  
           & 
	   {removes the current VRML file from the scene} \endhead
\hline
\end{longtable}


\begin{longtable}{|p{4cm}|p{10cm}|}
\hline
\multicolumn{2}{|p{9.5cm}|}
{\bf osg::Node *loadIcon(const char *filename);}\\
\hline
{Description:}  
           & 
	   {Load an icon file, similar to loadFile, but this method 
	   looks first in covise/icons/\$LookAndFeel, then covise/icons} \\
\hline
{Return value:}  
           & 
           {the root node of the scene graph created for this file , 
	   or \code{NULL}, if the file was not found} \endhead
\hline
\end{longtable} 
 
 
\subsection{Access to the Buttons of the Input Device}
\index{coVRPluginSupport!Access to the Buttons} 

\begin{longtable}{|p{4cm}|p{10cm}|}
\hline
\multicolumn{2}{|p{9.5cm}|}
{\bf coPointerButton *getButton();}\\
\hline
{Description:}  
           & 
	   {returns a pointer to a pointerButton object} \\
\hline
{Return value:}  
    & {class coPointerButton handles the button state} \endhead
\hline
\end{longtable} 
 
\subsection{Interactions}
\index{coVRPluginSupport!Interactions}

A plugin needs to know if the input device is already used for another 
interaction or if it is free for own interaction. If a plugin uses the input 
device it has to announce this to other plugins and to OpenCOVER.

%TODO document coInteraction

  
 
\subsection{Load and Unload Other Plugins and Communicate with them}
\index{coVRPluginSupport!Communicate with other Plugins}
\label{COVER:messages}

\begin{longtable}{|p{4cm}|p{2.5cm}|p{7cm}|}
\hline
\multicolumn{3}{|p{13.5cm}|}
{\bf coVRPlugin *addPlugin(const char *name);}\\
\hline
{Description:}  
           & \multicolumn{2}{|p{9.5cm}|}{Load a plugin} \\
\hline
\multicolumn{1}{|r|}{IN:} & {name} 
                          & \
			  {The name of the plugin. If the name of the shared 
			  object is libXxxPlugin.so, then the name is XxxPlugin}\\
\hline
{Return value:}  
    & \multicolumn{2}{|p{9.5cm}|}
    {Object of type coVRPlugin. Class coVRPlugin represents a plugin.} \endhead
\hline
\end{longtable} 
 
 
\begin{longtable}{|p{4cm}|p{2.5cm}|p{7cm}|}
\hline
\multicolumn{3}{|p{13.5cm}|}
{\bf int removePlugin(const char *name);}\\
\hline
{Description:}  
           & \multicolumn{2}{|p{9.5cm}|}{Unload a plugin} \\
\hline
\multicolumn{1}{|r|}{IN:} & {name} 
                          & \
			  {The name of the plugin to be unloaded.}\\
\hline
{Return value:}  
    & \multicolumn{2}{|p{9.5cm}|}{true, if successfully unloaded} \endhead
\hline
\end{longtable} 
  
 
\begin{longtable}{|p{4cm}|p{2.5cm}|p{7cm}|}
\hline
\multicolumn{3}{|p{13.5cm}|}
{\bf void unload(coVRPlugin *plugin);}\\
\hline
{Description:}  
           & \multicolumn{2}{|p{9.5cm}|}{Unload a plugin} \\
\hline
\hline
\multicolumn{1}{|r|}{IN:} & {plugin} 
                          & {The plugin object}\endhead
\hline
\end{longtable} 
 

\begin{longtable}{|p{4cm}|p{2.5cm}|p{7cm}|}
\hline
\multicolumn{3}{|p{13.5cm}|}
{\bf void sendMessage(coVRPlugin *sender, int toWhom, int type, int len, const
void *buf);}\\
\hline
{Description:}  
           & \multicolumn{2}{|p{9.5cm}|}{send a message to other plugins} \\
\hline
\multicolumn{1}{|r|}{IN:} & {sender} 
                          & \
			  {the name of the plugin which sends the message}\\
\hline
\multicolumn{1}{|r|}{IN:} & {toWhom} 
                          & \
			  {destination types are:\newline
			  TO\_ALL,\newline
                          TO\_ALL\_OTHERS,\newline
                          TO\_SAME,\newline
			  TO\_SAME\_OTHERS,\newline
			  NUM\_TYPES}\\
\hline
\multicolumn{1}{|r|}{IN:} & {type} 
                          & \
			  {a user defined number. 
			  The sender can identify the message by means 
			  of that number. For messages sent to all plugins 
			  this should be unique.}\\
\hline
\multicolumn{1}{|r|}{IN:} & {len} 
                          & \
			  {length of the message}\\
\hline
\multicolumn{1}{|r|}{IN:} & {buf} 
                          & \
			  {the message}\endhead
\hline
\end{longtable}


\begin{longtable}{|p{4cm}|p{2.5cm}|p{7cm}|}
\hline
\multicolumn{3}{|p{13.5cm}|}
{\bf void sendMessage(coVRPlugin *sender, const char *destination, int type, 
int len, const void *buf);}\\
\hline
{Description:}  
           & \multicolumn{2}{|p{9.5cm}|}{send a message to a plugin} \\
\hline
\multicolumn{1}{|r|}{IN:} & {sender} 
                          & \
			  {the name of the plugin which sends the message}\\
\hline
\multicolumn{1}{|r|}{IN:} & {destination} 
                          & \
			  {The name of the destination plugin}\\
\hline
\multicolumn{1}{|r|}{IN:} & {type} 
                          & \
			  {a user defined number. 
			  The sender can identify the message by means 
			  of that number. For messages sent to all plugins 
			  this should be unique.}\\
\hline
\multicolumn{1}{|r|}{IN:} & {len} 
                          & \
			  {length of the message}\\
\hline
\multicolumn{1}{|r|}{IN:} & {buf} 
                          & \
			  {the message}\endhead
\hline
\end{longtable}

 

\begin{longtable}{|p{4cm}|p{2.5cm}|p{7cm}|}
\hline
\multicolumn{3}{|p{13.5cm}|}
{\bf void addedNode(osg::Node *node, coVRPlugin *myPlugin);}\\
\hline
{Description:}  
           & \multicolumn{2}{|p{9.5cm}|}{
	   informs other plugins that this plugin extended the scene graph} \\
\hline
\multicolumn{1}{|r|}{IN:} & {node} 
                          & \
			  {pointer to the OpenSceneGraph node which was added}\\
\hline
\multicolumn{1}{|r|}{IN:} & {myPlugin} 
                          & \
			  {the plugin which added the node}\endhead
\hline
\end{longtable}  
 
\subsection{Append Buttons to the Pinboard}
\index{coVRPluginSupport!Append Buttons to the Pinboard}

The plugin programmer has full access to the Virtual Reality User 
Interface library (OpenVRUI), which is used by OpenCOVER to generate the pinboard 
and its submenus. With this library simple user interface elements
like labels and icons and more complex elements like menu items can
be created. The OpenVRUI header files are located in 
\code{covise/src/OpenCOVER/OpenVRUI}. A small example how to use the
library is available in the Cube plugin source code under
\code{covise/src/OpenCOVER/plugins/examples/Cube}.

Sometimes a plugin needs to create own menu items or submenus
and needs to add them to the main pinboard. The method getMenu
returns a pointer to the pinboard menu. The plugin Cube is a small
example which create an own submenu and add it to the pinboard.
\vspace*{0.5cm} 
\begin{longtable}{|p{4cm}|p{2.5cm}|p{7cm}|}
\hline
\multicolumn{3}{|p{13.5cm}|}
{\bf coMenu *getMenu();}\\
\hline
{Description:}  
           & \multicolumn{2}{|p{9.5cm}|}{
	   Get a pointer to the Pinboard.} \\
\hline
{Return Value:} & \multicolumn{2}{|p{9.5cm}|}
			  {pointer to the pinboard menu}\endhead
\hline
\end{longtable}  



\vspace*{0.5cm}
For compatibility with older COVISE Versions (< 5.0) the following
functions for menu item generation are available:

\begin{longtable}{|p{4cm}|p{3cm}|p{5cm}|}
\hline
\multicolumn{3}{|p{13.5cm}|}
{\bf void addToggleButton(const char *buttonName, const char
 *parentMenuName, int state, void *callback, void *classPtr, void
 *userData);}\\
\hline
{Description:}  
           & \multicolumn{2}{|p{9.5cm}|}{append a switch button to a menu} \\
\hline
\multicolumn{1}{|r|}{IN:} & {buttonName} 
                          & 
			  {the name of the button which is also the text 
			  on the button. Button names have to be unique.}\\
\hline
\multicolumn{1}{|r|}{IN:} & {parentMenuName} 
                          & 
			  {the name of the menu to which this button will 
			  be appended, code{NULL} menas to append it to the main Pinboard}\\
\hline
\multicolumn{1}{|r|}{IN:} & {state} 
                          & 
			  {The state of the switch button,
			  false=off, true=on}\\
\hline
\multicolumn{1}{|r|}{IN:} & {callback} 
                          & 
			  {function which is called on press/release}\\
\hline
\multicolumn{1}{|r|}{IN:} & {classPtr} 
                          & 
			  {ptr to the class which calls the callback}\\
\hline
\multicolumn{1}{|r|}{IN:} & {userData} 
                          & 
			  {ptr to data which are handed over to the callback}\endhead
\hline
\end{longtable}  
 

\begin{longtable}{|p{4cm}|p{3cm}|p{5cm}|}
\hline
\multicolumn{3}{|p{13.5cm}|}
{\bf void addGroupButton(const char *buttonName, const char *parentMenuName, int
 state, void *callback, int groupId, void *classPtr, void *userData);}\\
\hline
{Description:}  
           & \multicolumn{2}{|p{9.5cm}|}{append a switch button to a menu} \\
\hline
\multicolumn{1}{|r|}{IN:} & {buttonName} 
                          & 
			  {the name of the button which is also the text on 
			  the button. Button names have to be unique.}\\
\hline
\multicolumn{1}{|r|}{IN:} & {parentMenuName} 
                          & 
			  {the name of the menu to which this button will 
			  be appended, \code{NULL} means to append it to the main Pinboard}\\
\hline
\multicolumn{1}{|r|}{IN:} & {state} 
                          & 
			  {The state of the switch button,
                                         false=off, true=on}\\
\hline
\multicolumn{1}{|r|}{IN:} & {callback} 
                          & 
			  {function which is called on press/release}\\
\hline
\multicolumn{1}{|r|}{IN:} & {groupId} 
                          & 
			  {if a button is in the same group with others it 
			  is automatically switched off if another button 
			  is pressed. The group id of the navigation functions 
			  in the main Pinboard is zero. To create a 
			  new group id, use the method uniqueButtonGroup}\\
\hline
\multicolumn{1}{|r|}{IN:} & {classPtr} 
                          & 
			  {ptr to the class which calls the callback}\\
\hline
\multicolumn{1}{|r|}{IN:} & {userData} 
                          & 
			  {ptr to data which are handed over to the callback}\endhead
\hline
\end{longtable}  

 
\begin{longtable}{|p{4cm}|p{3cm}|p{5cm}|}
\hline
\multicolumn{3}{|p{13.5cm}|}
{\bf void addSubmenuButton(const char *buttonName, const char *parentMenuName,
 const char *subMenuName, int state, void *callback, int groupId, void
 *classPtr);}\\
\hline
{Description:}  
           & \multicolumn{2}{|p{9.5cm}|}
	   {append a switch button to a menu} \\
\hline
\multicolumn{1}{|r|}{IN:} & {buttonName} 
                          & 
			  {the name of the button which is also the text 
			  on the button. Button names have to be unique.}\\
\hline
\multicolumn{1}{|r|}{IN:} & {parentMenuName} 
                          & 
			  {the name of the menu to which this button will 
			  be appended, \code{NULL} means to append it to the main Pinboard}\\
\hline
\multicolumn{1}{|r|}{IN:} & {subMenuName} 
                          & 
			  {name of the submenu which is also the header text 
			  of the submenu. Name has to be unique?}\\
\hline
\multicolumn{1}{|r|}{IN:} & {state} 
                          & 
			  {The state of the switch button,
			  false=off, true=on}\\
\hline
\multicolumn{1}{|r|}{IN:} & {callback} 
                          & 
			  {function which is called on press/release}\\
\hline
\multicolumn{1}{|r|}{IN:} & {groupId} 
                          & 
			  {if a button is in the same group with others it 
			  is automatically switched off if another button is 
			  pressed. The group id of the navigation functions 
			  in the main Pinboard is zero. To create a new group 
			  id, use the method uniqueButtonGroup}\\
\hline
\multicolumn{1}{|r|}{IN:} & {classPtr} 
                          & 
			  {ptr to the class which calls the callback}\endhead
\hline
\end{longtable}  
 
 

\begin{longtable}{|p{4cm}|p{3cm}|p{5cm}|}
\hline
\multicolumn{3}{|p{13.5cm}|}
{\bf void setButtonState(const char *buttonName, int state);}\\
\hline
{Description:}  
           & \multicolumn{2}{|p{9.5cm}|}
	   {set the state of a switch or submenu button} \\
\hline
\multicolumn{1}{|r|}{IN:} & {buttonName} 
                          & 
			  {the name of the button}\\
\hline
\multicolumn{1}{|r|}{IN:} & {state} 
                          & 
			  {The state of the button\newline
			  switch button:\newline 
			  false=off, true=on\newline
			  submenu: \newline
			  false=closed, true=open}\endhead
\hline
\end{longtable} 


\begin{longtable}{|p{4cm}|p{3cm}|p{5cm}|}
\hline
\multicolumn{3}{|p{13.5cm}|}
{\bf addFunctionButton(const char *buttonName, const char *parentMenuName,
*callback, void *classPtr);}\\
\hline
{Description:}  
           & \multicolumn{2}{|p{9.5cm}|}
	   {add a function button, a function button just calls 
	   the callback when pressed} \\
\hline
\multicolumn{1}{|r|}{IN:} & {buttonName} 
                          & 
			  {the name of the button which is also the text 
			  on the button. Button names have to be unique.}\\
\hline
\multicolumn{1}{|r|}{IN:} & {parentMenuName} 
                          & 
			  {the name of the menu to which this button will 
			  be appended, \code{NULL} means to append it to the main Pinboard}\\
\hline
\multicolumn{1}{|r|}{IN:} & {callback} 
                          & 
			  {function which is called on press/release}\\
\hline
\multicolumn{1}{|r|}{IN:} & {classPtr} 
                          & 
			  {ptr to the class which calls the callback}\endhead
\hline
\end{longtable} 


\begin{longtable}{|p{4cm}|p{3cm}|p{5cm}|}
\hline
\multicolumn{3}{|p{13.5cm}|}
{\bf void addSliderButton(const char *buttonName, const char *parentMenuName,float min, float max, float value, void *callback, 
void *classPtr); }\\
\hline
{Description:}  
           & \multicolumn{2}{|p{9.5cm}|}{add a slider to a menu} \\
\hline
\multicolumn{1}{|r|}{IN:} & {buttonName} 
                          & 
			  {the name of the button which is also the text on the button. 
			  Button names have to be unique.}\\
\hline
\multicolumn{1}{|r|}{IN:} & {parentMenuName} 
                          & 
			  {the name of the menu to which this button will be 
			  appended, \code{NULL} means to append it to the main Pinboard}\\
\hline
\multicolumn{1}{|r|}{IN:} & {min} 
                          & 
			  {minimum slider value}\\
\hline
\multicolumn{1}{|r|}{IN:} & {max} 
                          & 
			  {maximum slider value}\\
\hline
\multicolumn{1}{|r|}{IN:} & {value} 
                          & 
			  {current slider value}\\
\hline
\multicolumn{1}{|r|}{IN:} & {callback} 
                          & 
			  {function which is if slider value changes}\\
\hline
\multicolumn{1}{|r|}{IN:} & {classPtr} 
                          & 
			  {ptr to the class which calls the callback}\endhead
\hline
\end{longtable}
 

\begin{longtable}{|p{4cm}|p{3cm}|p{5cm}|}
\hline
\multicolumn{3}{|p{13.5cm}|}
{\bf void setSliderValue(const char *buttonName, float value);}\\
\hline
{Description:}  
           & \multicolumn{2}{|p{9.5cm}|}{set slider value} \\
\hline
\multicolumn{1}{|r|}{IN:} & {buttonName} 
                          & 
			  {the name of the button}\\
\hline
\multicolumn{1}{|r|}{IN:} & {value} 
                          & 
			  {current slider value}\endhead
\hline
\end{longtable}
 

\begin{longtable}{|p{4cm}|p{3cm}|p{5cm}|}
\hline
\multicolumn{3}{|p{13.5cm}|}
{\bf void removeButton(const char *buttonName, const char *parentMenuName);}\\
\hline
{Description:}  
           & \multicolumn{2}{|p{9.5cm}|}{remove a button\newline
                                      parentMenuName=NULL: remove it from main menu} \\
\hline
\multicolumn{1}{|r|}{IN:} & {buttonName} 
                          & 
			  {the name of the button which is also the text on the button. 
			  Button names have to be unique.}\\
\hline
\multicolumn{1}{|r|}{IN:} & {parentMenuName} 
                          & 
			  {the name of the menu to which this button will be 
			  appended, NULL = append it to the main Pinboard}\endhead
\hline
\end{longtable}





\section{Use OpenCOVER built-in interaction without the Pinboard}


A plugin programmer may want to user OpenCOVER interaction although the pinboard is 
not visible, for example if the plugin implements a custom menu, but wants to use the navigation functions like XFORM, or the plugin wants to set 
the state of a button without using the menu.In this case the corresponding 
function needs to be configured in the section COVERPinboard. The Pinboard can 
be visible or invisible.

OpenCOVER supports the following functionality (see the usersguide for details):

Navigation: 
\begin{itemize}
\item XFORM
\item SCALE
\item VIEW\_ALL
\item FREEZE
\item COLLIDE
\item FLY
\item WALK
\item DRIVE
\item DRIVE\_SPEED
\end{itemize}

Part Manipulation: 
\begin{itemize}
\item SNAP
\item REMOVE
\item UNDO
\item MOVE\_PART
\end{itemize}

View Options: 
\begin{itemize}
\item COORD\_AXIS
\item SPECULAR
\item SPOTLIGHT
\item STEREO\_SEP
\end{itemize}

Collaborative Working: 
\begin{itemize}
\item LOOSE\_COUPLING
\item TIGHT\_COUPLING
\item MASTERSLAVE\_COUPLING
\item SHOW\_AVATAR
\end{itemize}

Scene Graph: 
\begin{itemize}
\item STORE
\item RELOAD\_VRML
\end{itemize}

COVISE: 
\begin{itemize}
\item EXECUTE
\item CUTTINGSURFACE
\item ISOSURFACEP
\item TRACERUSG, STRACER, BLOCKSTRACER, TETRATRACE, MAGTRACER, MAGBLOCKTRACER, 
CELLTRACER
\end{itemize}

Animation: 
\begin{itemize}
\item FORWARD
\item BACKWARD,
\item ANIM\_SPEED
\item STEADYCAM
\end{itemize}



\begin{longtable}{|p{4cm}|p{2.5cm}|p{7cm}|}
\hline
\multicolumn{3}{|p{13.5cm}|}
{\bf int isFunction(const char *functionName);}\\
\hline
{Description:}  
           & \multicolumn{2}{|p{9.5cm}|}
	   {Test if a function is configured and therefore in the pinboard} \\
\hline
\multicolumn{1}{|r|}{IN:} & {functionName} 
                          & \
			  {The name of the function, for example "XFORM"}\\
\hline
{Return value:}  
    & \multicolumn{2}{|p{9.5cm}|}{true, if function is in pinboard} \endhead
\hline
\end{longtable}



\begin{longtable}{|p{4cm}|p{2.5cm}|p{7cm}|}
\hline
\multicolumn{3}{|p{13.5cm}|}
{\bf void setFunctionState(const char *functionName, int state);}\\
\hline
{Description:}  
           & \multicolumn{2}{|p{9.5cm}|}
	   {Set the state of a toggle, group, or submenu button 
	   and call the callback} \\
\hline
\multicolumn{1}{|r|}{IN:} & {functionName} 
                          & \
			  {The name of the function, for example "FREEZE"}\\
\hline
\multicolumn{1}{|r|}{IN:} & {state} 
                          & {true/false}\endhead
\hline
\end{longtable}



\begin{longtable}{|p{4cm}|p{2.5cm}|p{7cm}|}
\hline
\multicolumn{3}{|p{13.5cm}|}
{\bf void setFunctionValue(const char *functionName, float val);}\\
\hline
{Description:}  
           & \multicolumn{2}{|p{9.5cm}|}
	   {Set the value of a slider button and call the callback} \\
\hline
\multicolumn{1}{|r|}{IN:} & {functionName} 
                          & \
			  {The name of the function, for example "DRIVE\_SPEED"}\\
\hline
\multicolumn{1}{|r|}{IN:} & {value} 
                          & {value between min and max}\endhead
\hline
\end{longtable}



\begin{longtable}{|p{4cm}|p{2.5cm}|p{7cm}|}
\hline
\multicolumn{3}{|p{13.5cm}|}
{\bf void doFunctionState(const char *functionName);}\\
\hline
{Description:}  
           & \multicolumn{2}{|p{9.5cm}|}{Call the callback of a function button} \\
\hline
\multicolumn{1}{|r|}{IN:} & {functionName} 
                          & \
			  {The name of the function, for example "XFORM"}\endhead
\hline
\end{longtable}


\begin{longtable}{|p{4cm}|p{2.5cm}|p{7cm}|}
\hline
\multicolumn{3}{|p{13.5cm}|}
{\bf int getFunctionState(const char *functionName, int *state);}\\
\hline
{Description:}  
           & \multicolumn{2}{|p{9.5cm}|}
	   {Get the state of a toggle, group or submenu button} \\
\hline
\multicolumn{1}{|r|}{IN:} & {functionName} 
                          & \
			  {The name of the function, for example "FREEZE"}\\
\hline
\multicolumn{1}{|r|}{IN:} & {state} 
                          & \
			  {provide a pointer for getting the current state}\\
\hline
{Return value:}  
    & \multicolumn{2}{|p{9.5cm}|}
    {true, if the function is in pinboard, false if not configured} \endhead
\hline
\end{longtable}


\begin{longtable}{|p{4cm}|p{2.5cm}|p{7cm}|}
\hline
\multicolumn{3}{|p{13.5cm}|}
{\bf int getFunctionValue(const char *functionName, float *value);}\\
\hline
{Description:}  
           & \multicolumn{2}{|p{9.5cm}|}
	   {Get the state of slider button} \\
\hline
\multicolumn{1}{|r|}{IN:} & {functionName} 
                          & \
			  {The name of the function, for example "DRIVE\_SPEED"}\\
\hline
\multicolumn{1}{|r|}{IN:} & {state} 
                          & \
			  {provide a pointer for getting the current value}\\
\hline
{Return value:}  
    & \multicolumn{2}{|p{9.5cm}|}
    {true, if the function is in pinboard, false if not configured} \endhead
\hline
\end{longtable}



\section{Controlling a COVISE module from within OpenCOVER}
\label{COVER:module}
\latexonly
\index{OpenCOVER!Controlling a COVISE module}
\endlatexonly

Module parameters are usually adjusted through the COVISE Map Editor or its 
Control Panel. If parameters should also be controllable from within OpenCOVER,
\begin{enumerate}
\item the module has to append the parameter information to the data objects and 
\item OpenCOVER has to be extended by a plugin.
\end{enumerate}

\subsection{Class coFeedback}

Feedback information is appended to data objects as attributes with a certain 
keyword. The class coFeedback handles the creation of such feedback-attributes:

In the constructor of coFeedback the name of the plugin has to be provided. When 
OpenCOVER receives this feedback information, it loads the appropriate plugin, if 
not already loaded (note that plugins can be loaded at starting time through the
keyword MODULE in the section CoverConfig in the file
covise.config).

 
\begin{longtable}{|p{4cm}|p{2.5cm}|p{7cm}|}
\hline
\multicolumn{3}{|p{13.5cm}|}
{\bf coFeedback::coFeedback(const char *pluginName)}\\
\hline
{Description:}  
           & \multicolumn{2}{|p{9.5cm}|}{Create a feedback object} \\
\hline
\multicolumn{1}{|r|}{IN:} & {PluginName} 
                          & \
			  {The name of the plugin which should be loaded by OpenCOVER}\endhead
\hline
\end{longtable}
 
 Parameters can now be added to such a feedback object with:

 
\begin{longtable}{|p{4cm}|p{2.5cm}|p{7cm}|}
\hline
\multicolumn{3}{|p{13.5cm}|}
{\bf void coFeedback::addPara(coUifPara *parameter);}\\
\hline
{Description:}  
           & \multicolumn{2}{|p{9.5cm}|}{add a parameter} \\
\hline
\multicolumn{1}{|r|}{IN:} & {Parameter} 
                          & \
			  {The parameter which should be steerable from 
			  within the plugin}\endhead
\hline
\end{longtable}
 
 
 In case that the module programmer needs to provide other information than 
 parameters to the plugin he can use:

 
\begin{longtable}{|p{4cm}|p{2.5cm}|p{7cm}|}
\hline
\multicolumn{3}{|p{13.5cm}|}
{\bf void coFeedback::addString(const char *userString);}\\
\hline
{Description:}  
           & \multicolumn{2}{|p{9.5cm}|}{add a string} \\
\hline
\multicolumn{1}{|r|}{IN:} & {UserString} 
                          & \
			  {Any string which is needed by the plugin}\endhead
\hline
\end{longtable}  
 
 
The feedback object has to be applied to a COVISE data object. This data object 
can be one which is created by the module anyway or you can create a dummy 
output object, for example a single point and append it to this object.


 
\begin{longtable}{|p{4cm}|p{2.5cm}|p{7cm}|}
\hline
\multicolumn{3}{|p{13.5cm}|}
{\bf void coFeedback:: apply(coDistributedObject *obj}\\
\hline
{Description:}  
           & \multicolumn{2}{|p{9.5cm}|}{apply feedback object} \\
\hline
\multicolumn{1}{|r|}{IN:} & {Obj} 
                          & \
			  {The COVISE data object to which the feedback is appended}\endhead
\hline
\end{longtable}
 
 The feedback object can be deleted after it is appended:

 
\begin{longtable}{|p{4cm}|p{10cm}|}
\hline
\multicolumn{2}{|p{9.5cm}|}
{\bf coFeedback::~coFeedback(const char *pluginName)}\\
\hline
{Description:}  
           & {Delete a feedback object} \endhead
\hline
\end{longtable} 


\subsection{Other functions}

If an OpenCOVER plugin has the functions \code{addObject} and \code{removeObject} 
implemented, it is informed if OpenCOVER receives "ADD OBJECT" or "REMOVE OBJECT". 
If it has the function \code{newInteractor} implemented it is also informed, when 
the COVISE object has feedback attributes attached.

With the coFeedback::getXXX methods, the plugin can access the values
of the module parameters, here one example for retreiving the values
of a slider parameter: 
 
\begin{longtable}{|p{4cm}|p{2.5cm}|p{7cm}|}
\hline
\multicolumn{3}{|p{13.5cm}|}
{\bf void coInteractor::getFloatSliderParam(int paraNo, float \and min, float \and max, float \and val);}\\
\hline
{Description:}  
           & \multicolumn{2}{|p{9.5cm}|}{Get the values of a slider parameter} \\
\hline
\multicolumn{1}{|r|}{IN:} & {paraNo} 
                          & \
			  {The index of the parameter. The parameters are attached to
			  the covise object with coFeedback::addPara(). paraNo is the
			  index of the parameter in the list of the parameters which
			  are added as feedback. it is not the parameter index in
			  the module info window or the control panel. }\\
			  \hline
\multicolumn{1}{|r|}{IN:} & {min} 
                          & \
			  {The minimum value of the slider parameter }\\
\multicolumn{1}{|r|}{IN:} & {max} 
                          & \
			  {The maximum value of the slider parameter }\\
\multicolumn{1}{|r|}{IN:} & {min} 
                          & \
			  {The current value of the slider parameter }\endhead
\hline
\end{longtable}  
  
 
The plugin can set module parameters with the coInteractor::setXXX
methods: 


\begin{longtable}{|p{4cm}|p{2.5cm}|p{7cm}|}
\hline
\multicolumn{3}{|p{13.5cm}|}
{\bf void coInteractor::setSliderParam(const char *name,float min,float max, float value);}\\
\hline
{Description:}  
           & \multicolumn{2}{|p{9.5cm}|}{Set the values of a slider parameter} \\
\hline
\multicolumn{1}{|r|}{IN:} & {name} 
                          & \
			  {The parameter name. It needs to be exactly the name
			  of the parameter of the module}\\
			  \hline
\multicolumn{1}{|r|}{IN:} & {min} 
                          & \
			  {The minimum value of the slider parameter }\\
\multicolumn{1}{|r|}{IN:} & {max} 
                          & \
			  {The maximum value of the slider parameter }\\
\multicolumn{1}{|r|}{IN:} & {min} 
                          & \
			  {The current value of the slider parameter }\endhead
\hline
\end{longtable}  

Please look at \code{covise/src/OpenCOVER/cover/coInteractor.h} for 
the methods to receive and set also boolean, scalar, vector, string and choice 
parameters.

With \code{newInteractor}, \code{addObject} and \code{removeObject} a plugin
is informed about any object which is added to OpenCOVER or removed from
OpenCOVER. There are a few functions which can be used to find out from
which module the coInteractor comes from or for which plugin it is
intended:

\begin{longtable}{|p{4cm}|p{2.5cm}|p{7cm}|}
\hline
\multicolumn{3}{|p{13.5cm}|}
{\bf const char *coInteractor::getPluginName();}\\
\hline
{Description:}  
           & \multicolumn{2}{|p{9.5cm}|}{Get the name of the plugin for which
			  this coInteractor is intended} \\
\hline
{Return value:} & \multicolumn{2}{|p{9.5cm}|}
			  {The name of the plugin.}\endhead
\hline
\end{longtable}  
   

\begin{longtable}{|p{4cm}|p{2.5cm}|p{7cm}|}
\hline
\multicolumn{3}{|p{13.5cm}|}
{\bf const char *coInteractor::getModuleName();}\\
\hline
{Description:}  
           & \multicolumn{2}{|p{9.5cm}|}{Get the name of the module from which
			  this coInteractor is generated} \\
\hline
{Return value:} & \multicolumn{2}{|p{9.5cm}|}
			  {The name of the module.}\endhead
\hline
\end{longtable}  

\begin{longtable}{|p{4cm}|p{2.5cm}|p{7cm}|}
\hline
\multicolumn{3}{|p{13.5cm}|}
{\bf int coInteractor::getModuleInstance();}\\
\hline
{Description:}  
           & \multicolumn{2}{|p{9.5cm}|}{Get the instance of the module from which
			  this coInteractor is generated} \\
\hline
{Return value:} & \multicolumn{2}{|p{9.5cm}|}
			  {The instance of the module.}\endhead
\hline
\end{longtable}     

\subsection{A Plugin Programming Example}

Have a look at the programming example in the directory
\code{covise/src/OpenCOVER/plugins/examples/Cube}.

The plugin \code{Cube} implements interaction with the COVISE example 
module \code{Cube}. The \code{Cube} module generates a solid cube. The
size and center of the cube can be manipulated through the module parameters.

The Cube plugin main class \code{CubePlugin} implements constructor and destructor
and the virtual methods \code{init}, \code{newInteractor}, \code{removeObject} and
\code{preFrame}.
In \code{init}, OpenSceneGraph nodes are prepared to be added to the scene graph later. 
Through \code{newInteractor} the plugin can access the current value of the
module parameters and the current cube COVISE object. In \code{removeObject}
the plugin is informed, when an object is deleted. In \code{preFrame}
direct manipulation of a wireframe cube is implemented. We need this
additional representation of the cube for direct manipulation because we 
cannot manipulate the
parameters of the Cube module in realtime (the communication between
OpenCOVER and the module is not fast enough). As soon as
the manipulation is finished, the current size and center of this wireframe
cube is sent back to the module.
 
\subsection{Module Example} 

(from \code{covise/src/application/examples/Cube})

\begin{verbatim}
#include "coFeedback.h"

// the ports and parameters of the module
coOutputPort *p_polyOut;
coFloatVectorParam *p_center;
coFloatParam *p_cusize;
 
// in the compute callback we create an output object
polygonObj = new DO_Polygons(polygonObjName, 8, xCoords, yCoords, zCoords, 24, 
vertexList, 6, polygonList);
 
// interaction info for OpenCOVER
coFeedback feedback("CubePlugin");
feedback.addPara(p_center);
feedback.addPara(p_cusize);
feedback.addString("Test the user string as well");
feedback.apply(polygonObj);
 
// apply the object to the port
p_polyOut->setObj(polygonObj);
\end{verbatim}





\section{Utility Classes}

\subsection{The classes \code{coIntersection} and \code{coAction}}

The classes coIntersection and coAction provide intersection testing of a node 
in your scene graph with the pointer ray.

 
\begin{longtable}{|p{4cm}|p{2.5cm}|p{7cm}|}
\hline
\multicolumn{3}{|p{13.5cm}|}
{\bf virtual int coAction::hit(osg::Vec3 \&hitPoint, osgUtil::Hit *hit)}\\
\hline
{Description:}  
           & \multicolumn{2}{|p{9.5cm}|}
	   {hit is called whenever the node or any node with this action is 
	   intersected}  \\
\hline
\multicolumn{1}{|r|}{IN:} & {hitPoint} 
                          & \
			  {intersection point in world coordinates }\\
\hline
\multicolumn{1}{|r|}{IN:} & {hit} 
                          & \
			  {see man osgUtil::Hit}\\
\hline
{Return value:}  
    & \multicolumn{2}{|p{9.5cm}|}
    {return ACTION\_CALL\_ON\_MISS if you want miss to be called, 
    otherwise return ACTION\_DONE } \endhead
\hline
\end{longtable}


 
\begin{longtable}{|p{4cm}|p{10cm}|}
\hline
\multicolumn{2}{|p{9.5cm}|}
{\bf virtual void coAction::miss()}\\
\hline
{Description:}  
           & 
	   {miss is called once after a hit if the node is 
	   not intersected any more} \\
\hline
{Return value:}  
    & 
    {return ACTION\_CALL\_ON\_MISS if you want miss to be called, 
    otherwise return ACTION\_DONE } \endhead
\hline
\end{longtable}


 
 
Append the node which you want to have tested for intersection to the global 
intersection list:

 
\begin{longtable}{|p{4cm}|p{2.5cm}|p{7cm}|}
\hline
\multicolumn{3}{|p{13.5cm}|}
{\bf void coIntersection::add(osg::Node *node, coAction *action)}\\
\hline
{Description:}  
           & \multicolumn{2}{|p{9.5cm}|}
	   {you can add a node to the global intersection 
	   list {\bf intersector}} \\
\hline
\multicolumn{1}{|r|}{IN:} & {node} 
                          & \
			  {the node you want to have tested 
			  for intersection}\\
\hline
\multicolumn{1}{|r|}{IN:} & {action} 
                          & \
			  {your class which is derived from coAction and 
			  which contains the functions hit and miss}\endhead
\hline
\end{longtable}

 
 
\begin{longtable}{|p{4cm}|p{2.5cm}|p{7cm}|}
\hline
\multicolumn{3}{|p{13.5cm}|}
{\bf void coIntersection::remove(osg::Node *node)}\\
\hline
{Description:}  
           & \multicolumn{2}{|p{9.5cm}|}
	   {remove a node from the global intersection 
	   list {\bf intersector}} \\
\hline
\multicolumn{1}{|r|}{IN:} & {node} 
                          & \
			  {the node you want to have tested for intersection}\endhead
\hline
\end{longtable}



\vspace*{1cm}
{\Large Example:}
\vspace*{0.5cm}

\begin{verbatim}
#include <coInteraction.h>
 
class myClass: public coAction
{
private:
   osg::MatrixTransform *transform;
 
public:
   myClass()
{
       //...
       intersector.add(transform, this);
       //...
};
   ~myClass()
{
       //...
       interscetor.remove(transform);
       //...
};
   virtual int hit(osg::Vec3 &hitPoint, osgUtil::Hit *hit)
{
       //...
       fprintf(stderr, "my transform was hit\n");
       return ACTION_CALL_ON_MISS;
};
 
   virtual void miss()
{
       //...
       fprintf(stderr, "my transform was missed\n");
};
}
\end{verbatim}


\subsection{OpenCOVER in a clustered environment}

\input{coVRPluginSupport/classopencover_1_1co_v_r_plugin_support.tex}
\input{coVRPluginSupport/classopencover_1_1co_pointer_button.tex}
\input{coVRPluginSupport/structopencover_1_1_isect.tex}

% Objektname Avatar.*: auschliessen von der BB-Berechnung
 




