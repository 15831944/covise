\begin{htmlonly}\begin{htmlonly}

\usepackage{html, htmllist}
\usepackage{longtable}

\bodytext{bgcolor="#ffffff" link="#0033cc" vlink="#0033cc"}

%%%==================================================	
%%%==================================================	

% #1  mark defined by \label
% #2  a linktext 
% #3  a html link 
\newcommand{covlink}[3]{\htmladdnormallink{#2}{#3} \latex{(\ref{#1})} }


\newenvironment{covimg}[4]%
{
 \begin{figure}[htp]
  \begin{center}
   \latexonly
      \includegraphics[scale=#4]{#1/pict/#2}
   \endlatexonly  
   \html{\htmladdimg[align="center"]{pict/#2.png}}
   \caption{#3}
  \end{center}
 \end{figure} 
}{} 

\newenvironment{covimg2}[3]%
{ 
 \begin{figure}[htp]
  \begin{center}
     \latexonly
       \includegraphics[scale=#3]{#1/pict/#2}   
     \endlatexonly
     \html{\htmladdimg[align="center"]{pict/#2.png}}
  \end{center}
 \end{figure} 
}{}

\definecolor{output}{rgb}{0.,0.,1.}
\definecolor{depend}{rgb}{1.,0.65,0.}
\definecolor{required}{rgb}{0.58,0.,0.83}
\definecolor{optional}{rgb}{0.,0.39,0.}

\newcommand{\addimage}[1] {\html{\htmladdimg{pict/#1.png}}}

\newcommand{\addpict}[4] {\latexonly
	     \begin{figure}[!htbp]
			  \begin{center}
   	 		  \includegraphics[scale=#1]{#2}
   	 		  \caption{#3}
		 		  \label{#4}
			  \end{center}
	 		\end{figure}
	     \endlatexonly}



\end{htmlonly}



%=============================================================
\startdocument
\chapter{Simulation Library}
\label{SimulationLibrary}
%=============================================================
\index{Simulation Library}

\section{Usage models}

{\Large Interactive simulation}
\vspace*{0.5cm}

The primary focus of the Simulation Library is to allow an easy implementation of 
interactive simulations. An interactive simulation is a simulation code which is started 
directly from COVISE, immediately visualizes its results and allows parameter changes 
during ongoing simulation. It is assumed, that the simulation code is not directly 
integrated as subroutine into a COVISE module but a client-server coupling to a remote 
machine is used.

\vspace*{1cm}
{\Large Control Visualization}
\vspace*{0.5cm}

Another application of simulation coupling is checking simulations with long execution 
times. In this case, the simulation start-up is not done by COVISE, but COVISE can 
connect to the running simulation and display the intermediate results. This allows 
an engineer to check them and terminate erroneous simulations early, avoiding unnecessary 
computing costs. Control visualization commands with repeated attachment to and 
detachment from the simulation is not yet implemented.


\section{Basic structure}

The simulation library is split into two parts:
\begin{itemize}
\item The simulation server is the COVISE-sided part of the simulation coupling. It is 
integrated into a module and takes care of the flow control, receives the parameters 
from the Map-Editor and puts the data from the simulation into the COVISE administered 
shared memory.
\item The simulation client is the simulation-sided part of the coupling. It connects 
the COVISE module to the simulation code, exchanges messages with the module to request 
parameter values and sends results to the server for visualization. 
\end{itemize}


\section{Language bindings}
\latexonly
\index{Simulation!Language bindings}
\endlatexonly

The simulation client is written in ANSI C and calls only functions of the standard C 
library, which is automatically linked to every code on UNIX operating systems. This is 
important, because some simulation codes do not allow additional libraries to be added 
to the linker line but only object files.

The implementation language of COVISE is C++, which also applies to the COVISE-sided 
part of CoSimLib used by the simulation server.  Thus it is possible to use all features 
of the development system.

The simulation library is accessible from multiple languages via appropriate language 
bindings. At the moment, 4 languages are supported:

\begin{enumerate} 
\item C as the internally-used `natural' language binding for the client
\item FORTRAN77 as the most common language used for simulations
\item Fortran 90 by using backward-compatibility with FORTRAN77
\item C++ by using header-definitions to map C++ language calls to C language calls
\end{enumerate} 

Additional language bindings will be supplied if necessary.

 
\section{Simulation Server Commands}
\latexonly
\index{Simulation!Server Commands}
\endlatexonly

The Simulation Server is the COVISE-sided part of the simulation coupling. When using 
interactive simulations, the module has to start the simulation code, for control 
visualization the module has to connect and authenticate to a running simulation. In 
both cases a network connection has to be created for control and data flow between 
the simulation and COVISE.

\vspace*{1cm}
{\Large Simulation start-up}
\index{Simulation!start-up}
\vspace*{0.5cm}

The simulation start-up is highly dependent on the circumstances in which the simulation 
takes place. To ensure sufficient flexibility for the module programmer as well as for 
the module user, the start-up procedure can be configured in the covise.config file. 
The following lines show an excerpt from covise.config.

\begin{verbatim}
MiniSim
{
   STARTUP SGI    CO_SIMLIB_CONN=%e ; export CO_SIMLIB_CONN; cd %0; ./miniSim
   STARTUP MANUAL echo "set CO_SIMLIB_CONN=%e and start MiniSim"
   PORTS    31000-31999
   SERVER   Module
   TIMEOUT  90
   VERBOSE  1
}
\end{verbatim}

For each simulation coupling module, a section with the name of the module executable 
is searched. This section must include at least one STARTUP line, since no default 
start-up can be defined. The first word after STARTUP defines a label, the rest is 
executed by a shell in the active directory where COVISE was started in. If the module 
is started on a remote host, it is executed in the directory the user enters on login, 
usually the home directory. It has to define an (exported) environment variable 
CO\_SIMLIB\_CONN in the shell and then start the simulation job itself. The following
variables can be used in the STARTUP string:

\begin{verbatim}
%%        the % character itself
%e        the value CO_SIMLIB_CONN has to be set to
%h        The host IP of the simulation host in "1.2.3.4" fashion (user-defined)
%0 ... %9 User-definable strings
\end{verbatim}

Every SimLib module automatically creates a choice parameter for selecting the startup 
line from the possibilities given in the config file.

Examples:

\begin{verbatim}
STARTUP LOCAL  CO_SIMLIB_CONN=%e ; export CO_SIMLIB_CONN; cd %0; ./miniSim
\end{verbatim}

If the user chooses "LOCAL" on the startup method choice, the environment variable is 
set, then changes to the module-defined directory (possibly from a string parameter of 
the module) and starts the program `miniSim'.

\begin{verbatim}
STARTUP REMOTE xterm -geometry 135x35 -e rsh -l %0 %h 'cd %1;
                        export CO_SIMLIB_CONN=%s; echo y | star' &
\end{verbatim}
This has to be written in one line. Here the user creates a new terminal window, starts 
a remote shell on a given host with a user-supplied login, changes to a user-supplied 
directory, sets the environment variable, then changes to the module-defined directory 
(possibly from a string parameter of the module) and then starts the program 'star', 
piping a 'y' answer to a 'are you sure' question.

All other parameters in the file are optional:
\begin{verbatim}
   PORTS    <minPort>-<maxPort>
\end{verbatim}
Sets the range of allowed port numbers for the network connection. For a single port, 
minimum and maximum has to be set to the same value. If no port range is given, the range
31000 ... 31999 is used. The explicit setting of port numbers is only required if 
firewalls are between the simulation host and the COVISE module host. Consult your 
Network administrators for help with firewalls and port numbers.
\begin{verbatim}
   SERVER   Module|Client
\end{verbatim}
Defines, which side of the client/server pair should open the TCP connection. As a 
default, the server is opened by the module. Most firewalls require the server 'outside' 
of the secure area. Consult your Network administrators for help with TCP socket 
opening directions.
\begin{verbatim}
   TIMEOUT  90
\end{verbatim}
Time in seconds to wait for a connection before giving up. Default is 90 sec. Increase 
the value if the network is extremely slow or if the simulation start-up time before 
starting the network client is long.
\begin{verbatim}
   VERBOSE  1
\end{verbatim}
Verbose level: when set to values \latexonly $>$ \endlatexonly \begin{htmlonly}
> \end{htmlonly} 0, CoSimLib commands create debug output to stderr. 
Currently, values 1...3 are supported.
\begin{verbatim}
   LOCAL  <local machine name>
\end{verbatim}
The machine name for the simulation to connect to. By default this is set to the result
of a nameserver lookup performed by a gethostname system call. This is typically the 
default network interface. The local machine name can be changed either to use a 
faster network connection or to connect through firewalls via IP-masquerading. Consult 
your network administrator for help with firewall connections and IP addresses. The 
IP number of the local host can also be set by the program. 

The start-up is triggered by the {\tt startupSim()} command:


\begin{longtable}{|p{4cm}|p{10cm}|}
\hline
\multicolumn{2}{|p{13.5cm}|}
{\bf int coSimLib::startupSim()}\\
\hline
{Description:}  
 & {1. Analyses the parameters given in \newline
                            the covise.config\newline
                            2. Sets verbose level\newline
                            3. Sets \% arguments in STARTUP line\newline
                            4. Calls shell with STARTUP command\newline
                            5. Tries to connect with simulation} \\
\hline
{COVISE states:}  & {all} \\
\hline
{Return value:}  & {-1 on error, 0 on success} \endhead
\hline
\end{longtable}
\index{coSimLib!startupSim}

If the user arguments (\%0 ... \%9) are used in the covise.config START line, they 
have to be set before the startupSim() command is called.

\vspace*{1cm}
{\Large Status request}
\index{Simulation!Status request}
\vspace*{0.5cm}

All coSimLib commands check for errors, which are usually reported by returning 
(-1) as result. For further checking, the internal status variable can be accessed:

\latexonly
\begin{longtable}{|p{4cm}|p{10cm}|}
\hline
\multicolumn{2}{|p{13.5cm}|}
{\bf int coSimLib::status()}\\
\hline
{Description:}  
           & {Return the error code of the last failed operation} \\
\hline
{COVISE states:}  & {all} \\
\hline
{Return value:}  
  & {= 0  no error,\newline 
                              $>$ 0:  UNIX errno variable value,\newline 
                              $<$ 0:  unknown Err} \endhead
\hline
\end{longtable}
\endlatexonly
\index{coSimLib!status}

\begin{htmlonly}
\begin{longtable}{|p{4cm}|p{10cm}|}
\hline
\multicolumn{2}{|p{13.5cm}|}
{\bf int coSimLib::status()}\\
\hline
{Description:}  
           & {Return the error code of the last failed operation} \\
\hline
{COVISE states:}  & {all} \\
\hline
{Return value:}  
  & {= 0  no error,\newline 
                               > 0: UNIX errno variable value,\newline 
                               < 0:  unknown Err} \endhead
\hline
\end{longtable}
\end{htmlonly}

\vspace*{1cm}
{\Large Setting start-up parameters}
\index{Simulation!start-up parameters}
\vspace*{0.5cm}

Both the local and target host IP addresses can be set explicitly. As a default, 
the local interface to be used is set to the value of the LOCAL variable if
given, otherwise the host's default interface is used. 

Both functions call the nameserver if the host is not given in dot notation and check 
whether the requested host name exists. If the name is given in dot notation, no 
nameserver lookup or host existence testing is performed.


\begin{longtable}{|p{4cm}|p{2.5cm}|p{7cm}|}
\hline
\multicolumn{3}{|p{13.5cm}|}
{\bf int coSimLib::setTargetHost(const char *hostname)}\\
\hline
{Description:}  
           & \multicolumn{2}{|p{9.5cm}|}{} \\
\hline
{COVISE states:}  
& \multicolumn{2}{|p{9.5cm}|}{all, only before coSimLib::startupSim()} \\
\hline
\multicolumn{1}{|r|}{IN:} & {hostname} 
                          & {name of target host}\\
\hline
{Return value:}  
& \multicolumn{2}{|p{9.5cm}|}{0 = no error, -1 = host unknown} \endhead
\hline
\end{longtable}
\index{coSimLib!setTargetHost}

\begin{longtable}{|p{4cm}|p{2.5cm}|p{7cm}|}
\hline
\multicolumn{3}{|p{13.5cm}|}
{\bf int coSimLib::setLocalHost(const char *hostname)}\\
\hline
{Description:}  
           & \multicolumn{2}{|p{9.5cm}|}{Set the local hostname} \\
\hline
{COVISE states:}  
& \multicolumn{2}{|p{9.5cm}|}{all, only before coSimLib::startupSim()} \\
\hline
\multicolumn{1}{|r|}{IN:} & {hostname} 
                          & {name of local host}\\
\hline
{Return value:}  
& \multicolumn{2}{|p{9.5cm}|}{0 = no error, -1 = host unknown} \endhead
\hline
\end{longtable}
\index{coSimLib!setLocalHost}

The user arguments \%0 ... \%9 must be set explicitly before starting the simulation. 
This allows simulation specific parameters, e.g. starting directory or startup parameters,
to be set.

\begin{longtable}{|p{4cm}|p{2.5cm}|p{7cm}|}
\hline
\multicolumn{3}{|p{13.5cm}|}
{\bf int coSimLib::setUsrArg (int num, const char *argVal)}\\
\hline
{Description:}  
           & \multicolumn{2}{|p{9.5cm}|}{Set one of the user arguments} \\
\hline
{COVISE states:}  
   & \multicolumn{2}{|p{9.5cm}|}{all, only before coSimLib::startupSim()} \\
\hline
\multicolumn{1}{|r|}{IN:} & {num} 
                     & {which argument to set: 0...9}\\
\hline
\multicolumn{1}{|r|}{IN:} & {argVal} 
                          & {value of the argument}\\
\hline
{Return value:}  
   & \multicolumn{2}{|p{9.5cm}|}{0 = no error, -1 = unknown} \endhead
\hline
\end{longtable}
\index{coSimLib!setUsrArg}

\vspace*{1cm}
{\Large Starting the Server mode}
\index{Simulation!Starting the Server mode}
\vspace*{0.5cm}

Whenever this command is called, the module waits for the simulation to send client commands 
until a `finish' command is sent. This may only be done from the compute callback.


\begin{longtable}{|p{4cm}|p{10cm}|}
\hline
\multicolumn{2}{|p{13.5cm}|}
{\bf int coSimLib::serverMode()}\\
\hline
{Description:}  
           & {set server mode: wait for client commands and perform them until a finish command is received} \\
\hline
{COVISE states:}  
   & {Compute} \\
\hline
{Return value:}  
   & {0 = returned by finish command,\newline-1 = returned because of error} \endhead
\hline
\end{longtable}
\index{coSimLib!serverMode}


Parameter requests are also handled outside of the server mode, but data creation is only allowed 
when the simulation server module is in the compute() callback and in server mode.


\vspace*{1cm}
{\Large Requesting and setting verbose mode}
\vspace*{0.5cm}

The verbose level can be set and requested by the user. It is recommended to request the verbose 
mode when creating own debug output in simulation coupling modules.


\begin{longtable}{|p{4cm}|p{2.5cm}|p{7cm}|}
\hline
\multicolumn{3}{|p{13.5cm}|}
{\bf void coSimLib::setVerbose(int level)}\\
\hline
{Description:}  
           & \multicolumn{2}{|p{9.5cm}|}{set verbose level} \\
\hline
\multicolumn{1}{|r|}{IN:} & {level} 
                          & {0=no, 4=max. verbose level}\endhead
\hline
\end{longtable}
\index{coSimLib!setVerbose}

\begin{longtable}{|p{4cm}|p{10cm}|}
\hline
\multicolumn{2}{|p{13.5cm}|}
{\bf int coSimLib::getVerboseLevel()}\\
\hline
{Description:}  
           & {get verbose level} \\
\hline
{Return value:}  
         & {verbose level: 0=no, 4=max. verbose level} \endhead
\hline
\end{longtable}
\index{coSimLib!getVerboseLevel}

\section{Simulation Client Commands}
\latexonly
\index{Simulation!Client Commands}
\endlatexonly

{\Large Start-up the connection with COVISE}
\vspace*{0.5cm}

This command checks the environment for the variable CO\_SIMLIB\_CONN, analyses it and tries to 
open a connection with the COVISE module. Depending on the setting in the covise.config file, 
this can be either a client or a server connection on any port in the specified direction. 
The command tries to open the connection for the period specified in the configuration file and 
returns with an error message if the connection was not established.


\begin{longtable}{|p{4cm}|p{10cm}|}
\hline
{\bf C:}  
           & {\bf coInitConnect()} \\
\hline
{\bf FORTRAN:}  
           & {\bf COVINI()} \\
\hline
\hline
{Description:}  
           & {Start connection with COVISE} \\
\hline
{Return value:}  
         & {0 = successfully connected, -1 = not connected} \endhead
\hline
\end{longtable}
\index{SimClient!coInitConnect}
\index{SimClient!COVINI}

\vspace*{1cm}
{\Large Check whether COVISE is connected}
\index{Simulation!Check connection}
\vspace*{0.5cm}

This call checks whether the connection to COVISE is still active. It does not check, whether 
the module is able to handle data now or whether it is busy.


\begin{longtable}{|p{4cm}|p{10cm}|}
\hline
{\bf C:}  
           & {\bf coNotConnected()} \\
\hline
{\bf FORTRAN:}  
           & {\bf CONOCO()} \\
\hline
\hline
{Description:}  
           & {Check connection with COVISE} \\
\hline
{Return value:}  
   & {0 = connected, -1 = not connected} \endhead
\hline
\end{longtable}
\index{SimClient!coNotConnected}
\index{SimClient!CONOCO}

\vspace*{1cm}
{\Large Retrieving parameter values}
\vspace*{0.5cm}

All parameter calls can be submitted both in the compute and main-loop state of the COVISE 
coupling module. One should be aware that only immediate mode parameters are updated at the 
module immediately while non-immediate parameters are only updated before entering the compute()
callback. 

All parameter requests have the parameter name as input and the requested parameter(s) as output. 
Returned is an error code: 0 for proper function, -1, if the parameter is not yet available, 
-2 for an unknown parameter name.

Not all parameter types are currently implemented in the simulation library.

\begin{longtable}{|p{4cm}|p{2.5cm}|p{7cm}|}
\hline
\multicolumn{3}{|p{13.5cm}|}
{\bf int coGetParaSlider(const char *name,float *min, float *max, float *value)}  \\
\hline
\multicolumn{3}{|p{13.5cm}|}{\bf COGPSL(name,min,max,val)}  \\
\hline
\hline
{Description:}  
           & \multicolumn{2}{|p{9.5cm}|}{Gets values from a COVISE slider parameter: 
	                  minimum, maximum and value of the slider parameter.} \\
\hline
\multicolumn{1}{|r|}{IN:} & {name} 
                          & {name of parameter}\\
\hline
\multicolumn{1}{|r|}{OUT:} & {min, max, value} 
                           & {Slider parameters)}\\
\hline
{Return value:}  
  & \multicolumn{2}{|p{9.5cm}|}{0 = ok, -1 = not yet available, -2 = unknown} \endhead
\hline
\end{longtable}
\index{SimClient!coGetParaSlider}
\index{SimClient!COGPSL}

\begin{longtable}{|p{4cm}|p{2.5cm}|p{7cm}|}
\hline
\multicolumn{3}{|p{13.5cm}|}{\bf int coGetParaFloat(const char *name, float *value)} \\
\hline
\multicolumn{3}{|p{13.5cm}|}{\bf COGPFL(name,value)}  \\
\hline
\hline
{Description:}  
           & \multicolumn{2}{|p{9.5cm}|}{Gets values from a COVISE float parameter} \\
\hline
\multicolumn{1}{|r|}{IN:} & {name} 
                          & {name of parameter}\\
\hline
\multicolumn{1}{|r|}{OUT:} & {value} 
                           & {parameter value (output)}\\
\hline
{Return value:}  
  & \multicolumn{2}{|p{9.5cm}|}{0 = ok, -1 = not yet available, -2 = unknown} \endhead
\hline
\end{longtable}
\index{SimClient!coGetParaFloat}
\index{SimClient!COGPFL}

\begin{longtable}{|p{4cm}|p{2.5cm}|p{7cm}|}
\hline
\multicolumn{3}{|p{13.5cm}|}{\bf int coGetParaInt(const char *name, int *value)}  \\
\hline
\multicolumn{3}{|p{13.5cm}|}{\bf COGPIN(name, value)}  \\
\hline
\hline
{Description:}  
           & \multicolumn{2}{|p{9.5cm}|}{Gets values from a COVISE int parameter} \\
\hline
\multicolumn{1}{|r|}{IN:} & {name} 
                          & {name of parameter}\\
\hline
\multicolumn{1}{|r|}{OUT:} & {value} 
                           & {parameter value (output)}\\
\hline
{Return value:}  
  & \multicolumn{2}{|p{9.5cm}|}{0 = ok, -1 = not yet available, -2 = unknown} \endhead
\hline
\end{longtable}
\index{SimClient!coGetParaInt}
\index{SimClient!COGPIN}

\begin{longtable}{|p{4cm}|p{2.5cm}|p{7cm}|}
\hline
\multicolumn{3}{|p{13.5cm}|}{\bf int coGetParaChoice(const char *name, int *value)}\\
\hline
\multicolumn{3}{|p{13.5cm}|}{\bf COGPCH(name, ivalue)}  \\
\hline
\hline
{Description:}  
           & \multicolumn{2}{|p{9.5cm}|}{Gets values from a COVISE choice parameter} \\
\hline
\multicolumn{1}{|r|}{IN:} & {name} 
                          & {name of parameter}\\
\hline
\multicolumn{1}{|r|}{OUT:} & {value} 
                           & {parameter value (output)}\\
\hline
{Return value:}  
  & \multicolumn{2}{|p{9.5cm}|}{0 = ok, -1 = not yet available, -2 = unknown} \endhead
\hline
\end{longtable}
\index{SimClient!coGetParaChoice}
\index{SimClient!COGPCH}

\begin{longtable}{|p{4cm}|p{2.5cm}|p{7cm}|}
\hline
\multicolumn{3}{|p{13.5cm}|}{\bf int coGetParaBool(const char *name, int *value)}  \\
\hline
\multicolumn{3}{|p{13.5cm}|}{\bf COGPBO(name, ivalue)} \\
\hline
\hline
{Description:}  
           & \multicolumn{2}{|p{9.5cm}|}{Gets values from a COVISE choice parameter} \\
\hline
\multicolumn{1}{|r|}{IN:} & {name} 
                          & {name of parameter}\\
\hline
\multicolumn{1}{|r|}{OUT:} & {value} 
                           & {parameter value (output)}\\
\hline
{Return value:}  
  & \multicolumn{2}{|p{9.5cm}|}{0 = ok, -1 = not yet available, -2 = unknown} \endhead
\hline
\end{longtable}
\index{SimClient!coGetParaBool}
\index{SimClient!COGPBO}

\begin{longtable}{|p{4cm}|p{2.5cm}|p{7cm}|}
\hline
\multicolumn{3}{|p{13.5cm}|}{\bf int coGetParaFile(const char *name, char *filepath)}  \\
\hline
\multicolumn{3}{|p{13.5cm}|}{\bf COGPFI(name, path)} \\
\hline
\hline
{Description:}  
           & \multicolumn{2}{|p{9.5cm}|}{Gets values from a COVISE file-browser parameter} \\
\hline
\multicolumn{1}{|r|}{IN:} & {name} 
                          & {name of parameter}\\
\hline
\multicolumn{1}{|r|}{OUT:} & {filepath} 
                           & {selected file (output)}\\
\hline
{Return value:}  
  & \multicolumn{2}{|p{9.5cm}|}{0 = ok, -1 = not yet available, -2 = unknown} \endhead
\hline
\end{longtable}
\index{SimClient!coGetParaFile}
\index{SimClient!COGPFI}

\begin{longtable}{|p{4cm}|p{2.5cm}|p{7cm}|}
\hline
\multicolumn{3}{|p{13.5cm}|}{\bf int coGetParaText(const char *name, char *text)}  \\
\hline
\multicolumn{3}{|p{13.5cm}|}{\bf COGPTX(name, text)}  \\
\hline
\hline
{Description:}  
           & \multicolumn{2}{|p{9.5cm}|}{Gets values from a COVISE file-browser parameter} \\
\hline
\multicolumn{1}{|r|}{IN:} & {name} 
                          & {name of parameter}\\
\hline
\multicolumn{1}{|r|}{OUT:} & {value} 
                           & {parameter value (output)}\\
\hline
{Return value:}  
  & \multicolumn{2}{|p{9.5cm}|}{0 = ok, -1 = not yet available, -2 = unknown} \endhead
\hline
\end{longtable}
\index{SimClient!coGetParaText}
\index{SimClient!COGPTX}

\vspace*{1cm}
{\Large Creating data objects}
\index{Simulation!Creating data objects}
\vspace*{0.5cm}

Currently only the creation of unstructured scalar and vector data fields  
({\tt coDoFloat} and {\tt coDoFloat}) is supported.


\begin{longtable}{|p{4cm}|p{2.5cm}|p{7cm}|}
\hline
\multicolumn{3}{|p{13.5cm}|}{\bf int coSend1Data(const char *name, int num, float *data)}  \\
\hline
\multicolumn{3}{|p{13.5cm}|}{\bf COSU1D(name, num, data)}  \\
\hline
\hline
{Description:}  
           & \multicolumn{2}{|p{9.5cm}|}{Create a 1D unstructured data object 
	                              at a port and copy the data to 
				      the shared memory} \\
\hline
{State:}  & \multicolumn{2}{|p{9.5cm}|}{Compute } \\
\hline
\multicolumn{1}{|r|}{IN:} & {name} 
                          & {name of output port}\\
\hline
\multicolumn{1}{|r|}{IN:} & {num} 
                          & {number of data elements}\\
\hline
\multicolumn{1}{|r|}{IN:} & {data} 
                          & {Pointer to array containing data}\\
\hline
{Return value:}  
  & \multicolumn{2}{|p{9.5cm}|}{=0: ok , otherwise error} \endhead
\hline
\end{longtable}
\index{SimClient!coSend1Data}
\index{SimClient!COSU1D}

\begin{longtable}{|p{4cm}|p{2.5cm}|p{7cm}|}
\hline
\multicolumn{3}{|p{13.5cm}|}
  {\bf int coSend3Data(const char *name, int num, float *x, float *y, float *z)}  \\
\hline
\multicolumn{3}{|p{13.5cm}|}{\bf COSU3D(name, num, x,y,z)}  \\
\hline
\hline
{Description:}  
           & \multicolumn{2}{|p{9.5cm}|}{Create a 3D unstructured data object 
	                    at a port and copy the data to the shared memory} \\
\hline
{State:}  & \multicolumn{2}{|p{9.5cm}|}{Compute} \\
\hline
\multicolumn{1}{|r|}{IN:} & {name} 
                          & {name of output port}\\
\hline
\multicolumn{1}{|r|}{IN:} & {num} 
                          & {number of data elements}\\
\hline
\multicolumn{1}{|r|}{IN:} & {x,y,z} 
                          & {Pointer to arrays containing 
			                               data components}\\
\hline
{Return value:}  
  & \multicolumn{2}{|p{9.5cm}|}{=0: ok , otherwise error} \endhead
\hline
\end{longtable}
\index{SimClient!coSend3Data}
\index{SimClient!COSU3D}

Other types will be implemented in the future, especially for sending unstructured grids and 
structured data sets.




\section{Handling data on parallel machines}
\latexonly
\index{Simulation!data on parallel machines}
\endlatexonly

Parallelization of simulations is usually done by domain decomposition: parts of the grid are decomposed to several processors. For these kinds of
simulations, only one node is connected to COVISE and has to perform the data and parameter propagation.

\vspace*{1cm}
{\Large Initialisation}
\vspace*{0.5cm}

The simulation library supports distributed data by supplying a set of additional calls, which prepare and perform the gathering operations. Therefore,
two different index mappings can be defined for both cell indices and vertex indices. To set these tables, all nodes send their local-to-global mapping
field (array size of local data size, containing global data indices).


\begin{longtable}{|p{4cm}|p{2.5cm}|p{7cm}|}
\hline
\multicolumn{3}{|p{13.5cm}|}{\bf int coParallelInit(int numParts, int numPorts)}  \\
\hline
\multicolumn{3}{|p{13.5cm}|}{\bf COPAIN(numprocs,numports)}  \\
\hline
\hline
{Description:}  
           & \multicolumn{2}{|p{9.5cm}|}{initialise parallel data sending} \\
\hline
{State:}  & \multicolumn{2}{|p{9.5cm}|}{All } \\
\hline
\multicolumn{1}{|r|}{IN:} & {numParts} 
                          & {number of domains}\\
\hline
\multicolumn{1}{|r|}{IN:} 
       & {numPorts} 
       & {number of ports creating parallel data}\\
\hline
{Return value:}  
  & \multicolumn{2}{|p{9.5cm}|}{=0: ok, otherwise error} \endhead
\hline
\end{longtable}
\index{SimClient!coParallelInit}
\index{SimClient!COPAIN}

\begin{longtable}{|p{4cm}|p{2.5cm}|p{7cm}|}
\hline
\multicolumn{3}{|p{13.5cm}|}
  {\bf int coParallelPort(const char *portname, int isCellData);}  \\
\hline
\multicolumn{3}{|p{13.5cm}|}{\bf COPAPO(portname, isCellData)}  \\
\hline
\hline
{Description:}  
           & \multicolumn{2}{|p{9.5cm}|}{Declare the named port as parallel data port} \\
\hline
{State:}  & \multicolumn{2}{|p{9.5cm}|}{All } \\
\hline
\multicolumn{1}{|r|}{IN:} & {Portname} 
                          & {name of the output data port}\\
\hline
\multicolumn{1}{|r|}{IN:} & {IsCellData} 
                          & {whether this is\newline 
			                               (1)cell- or\newline
						       (0)vertex-based data}\\
\hline
{Return value:}  
  & \multicolumn{2}{|p{9.5cm}|}{=0: ok, otherwise error} \endhead
\hline
\end{longtable}
\index{SimClient!coParallelPort}
\index{SimClient!COPAPO}

There can be two different tables for mapping local fields into global fields: a cell map and 
a vertex map. For every node, the cell map gives for each local cell the cell numbers in the 
global field. Accordingly, the vertex table gives the global coordinate index for every local 
coordinate.


\begin{longtable}{|p{4cm}|p{2.5cm}|p{7cm}|}
\hline
\multicolumn{3}{|p{13.5cm}|}
{\bf int coParallelCellMap(int node, int numCells, int *locToGlob);}  \\
\hline
\multicolumn{3}{|p{13.5cm}|}
{\bf int coParallelVertexMap(int node, int numVert, int *locToGlob);}  \\
\hline
\multicolumn{3}{|p{13.5cm}|}{\bf COPACM(node, numCells, locToGlob)}  \\
\hline
\multicolumn{3}{|p{13.5cm}|}{\bf COPAVM(node, numVert, locToGlob)}  \\
\hline
\hline
{Description:}  
      & \multicolumn{2}{|p{9.5cm}|}{Set index mapping for a node's cell/vertex data} \\
\hline
{State:}  & \multicolumn{2}{|p{9.5cm}|}{All } \\
\hline
\multicolumn{1}{|r|}{IN:} 
           & {Node} 
           & {node ID (starting with node 0)}\\
\hline
\multicolumn{1}{|r|}{IN:} 
           & {NumCells}              
	   & {number of cells on this node}\\
\hline
\multicolumn{1}{|r|}{IN:} 
           & {NumVert} 
           & {number of vertices on this node}\\
\hline
\multicolumn{1}{|r|}{IN:} 
           & {LocToGlob} 
           & {global index for each local index}\\
\hline
{Return value:}  
  & \multicolumn{2}{|p{9.5cm}|}{=0: ok, otherwise error} \endhead
\hline
\end{longtable}
\index{SimClient!coParallelCellMap}
\index{SimClient!COPACM}
\index{SimClient!coParallelVertexMap}
\index{SimClient!COPAVM}

\vspace*{1cm}
{\Large Data creation}
\vspace*{0.5cm}

Data is created by telling the COVISE server, which node sends, and then using the standard data 
object creation calls, which automatically gather the data according to the mapping set during 
the initialization.


\begin{longtable}{|p{4cm}|p{2.5cm}|p{7cm}|}
\hline
\multicolumn{3}{|p{13.5cm}|}{\bf int coParallelNode(int node);}  \\
\hline
\multicolumn{3}{|p{13.5cm}|}{\bf COPANO(node)}  \\
\hline
\hline
{Description:}  
           & \multicolumn{2}{|p{9.5cm}|}{set current processor number} \\
\hline
{State:}  & \multicolumn{2}{|p{9.5cm}|}{All } \\
\hline
\multicolumn{1}{|r|}{IN:} & {node} 
                          & {node ID (starting with node 0)}\\
\hline
{Return value:}  
  & \multicolumn{2}{|p{9.5cm}|}{=0: ok, otherwise error} \endhead
\hline
\end{longtable}
\index{SimClient!coParallelNode}
\index{SimClient!COPANO}





















































