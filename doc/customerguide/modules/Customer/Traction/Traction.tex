\begin{htmlonly}\begin{htmlonly}
\documentclass{covise}

\usepackage{html, htmllist}
\usepackage{color}
\usepackage{graphicx}
\usepackage{longtable}
\usepackage{palatino}
\usepackage{picins}
\usepackage[colorlinks,dvips]{hyperref}

	  
\bodytext{BGCOLOR=FFFFFF LINK=#0033cc VLINK=#0033cc}


% #1  mark defined by \label
% #2  a linktext 
% #3  a html link 
\newcommand{\covlink}[3]%
{\html{\htmladdnormallink{#1}{#3}}\latex{\hyperref[#1]{#2} (\ref{#1})}}


\newenvironment{covimg}[4]%
{ \html{\htmladdimg[ALIGN=CENTER]{#2.gif}}
 
 \latexonly
 \begin{figure}[htp]
  \begin{center}
   \includegraphics[scale=#4]{#1/#2}
   \caption{#3}
  \end{center}
 \end{figure}
 \endlatexonly
}

\newenvironment{covimg2}[3]%
{ \html{\htmladdimg[ALIGN=CENTER]{#2.gif}}
 
 \latexonly
 \begin{figure}[htp]
  \begin{center}
   \includegraphics[scale=#3]{#1/#2}
  \end{center}
 \end{figure}
 \endlatexonly
}

\definecolor{output}{rgb}{0.,0.,1.}
\definecolor{depend}{rgb}{1.,0.65,0.}
\definecolor{required}{rgb}{0.58,0.,0.83}
\definecolor{optional}{rgb}{0.,0.39,0.}

\end{htmlonly}

%=============================================================
%=============================================================


%=============================================================
\startdocument
\subsection{Traction}
\label{Traction}
%=============================================================


%
% short description what the module does
%
Traction starts an ANSYS traction simulation. The ANSYS script
file name is zug\_driver.log. This script applies on a basic
cell a force directed along the X axis. The applied force is
such that the force per unit length has the value 1 in the
system of units employed. The value itself is not very
relevant, because we are looking for the traction resistance
loss. This number appears in percent in the info message window.

%
% input of a module icon for example
% #1	path for eps
% #2  picture name
% #3  scale factor
\begin{covimg2}{modules/Customer/Traction}{Traction}{0.7}\end{covimg2}



%
% short information about versions 
%
%Sample is available since COVISE snap-2000-10 on all supported platforms.

%
%=============================================================
%\subsubsection{Parameters}
%=============================================================
%

%\covlink{Colors}{Colors}{../../Color/Colors/Colors.html}

%This module has no parameters.


%
%=============================================================
\subsubsection{Input Ports}
%=============================================================
%


\begin{longtable}{|p{3.5cm}|p{4cm}|p{7cm}|}
\hline
   \bf{Name} & \bf{Type} & \bf{Description} \endhead
\hline\hline
	\textcolor{required}{zugParam} & DO\_Text & Message with
                 values for simulation parameters. These values
                 are in fact only used as a trigger of the simulation.
                 If the message has no contents, i.e. if it has 0 length, 
                 then no simulation is started. If this object has
                 the READ\_ANSYS attribute, this attribute is replicated
                 on output for an automatic adjustment of the parameters
                 of a ReadANSYS module, which is usually connected with
                 the output of Traction.\\
\hline
	\textcolor{required}{mssgFromEmbossing} & DO\_Text
				       & This is a message from Embossing
                    which signals whether a traction simulation may be started
                    or not. In the first case the string {\sl Go} is expected.
                    A {\sl Do not go} message is sent to this module if 
                    a traction simulation should not be started. \\
%\covlink{Transform}{Transform}{../../Tools/Transform/Transform.html}.\\
                     
                    
														
%	....
%	....

\hline
\end{longtable}
%=============================================================



%
%=============================================================
\subsubsection{Output Ports}
%=============================================================
%

 
\begin{longtable}{|p{3.5cm}|p{4cm}|p{7cm}|}
\hline
   \bf{Name} & \bf{Type} & \bf{Description} \endhead
\hline\hline
	\textcolor{required}{file\_name} & DO\_Text& Text with the path of the results
                          file. This object may also have a READ\_ANSYS attribute for
                          the automatic adjustment of parameters in ReadANSYS. \\
%	....
%	....

\hline
\end{longtable}
%=============================================================


%%=============================================================
%\subsubsection{Examples}
%%=============================================================
%%
%
%% examples for using this module
%
%%\paragraph{First example}
%%
%\begin{covimg}{modules/Tools/ImageToTexture}%
%		{ImageToTextureMap1}{covise/net/examples/ImageToTexture.net}{0.6}\end{covimg}
%
%In the first example we show a dynamic geometry. As the geometry moves the image 
%moves with it. In order to achieve this effect, we use the displacement information
%at the second port.
%
%\begin{covimg}{modules/Tools/ImageToTexture}%
%		{ImageToTextureRend1}{The image is dragged by the material motion.}{0.6}\end{covimg}
%
%\begin{covimg}{modules/Tools/ImageToTexture}%
%		{ImageToTextureMap2}{covise/net/examples/ImageToTexture2.net}{0.6}\end{covimg}
%
%In the second example we want to illustrate the effect of the parameter {\sl GroupGeometry}
%and of size adjustment. The geometry is a set with 4 DO\_Polygon object. When
%the value of {\sl GroupGeometry} is true (default), the image is mapped once onto the
%whole geometry. This effect is seen in the first renderer image. If the value
%of this parameter is false, then we get the second image. Note that here the geometry
%is used separately for each DO\_Polygon object. The third image has the default
%value for this parameter, i.e. true. If we are seeing here many eyes, it is because
%we are no longer fitting the image size to that of the geometry. In this case, we
%have manually set the image size to an inferior value, that is why we have to replicate
%the image in order to create a texture for the whole geometry.
%
%\begin{covimg}{modules/Tools/ImageToTexture}%
%		{ImageToTextureRend2_1}{GroupGeometry is true, the image size is that of the geometry.}{0.6}\end{covimg}
%
%\begin{covimg}{modules/Tools/ImageToTexture}%
%		{ImageToTextureRend2_2}{GroupGeometry is false.}{0.6}\end{covimg}
%
%\begin{covimg}{modules/Tools/ImageToTexture}%
%		{ImageToTextureRend2_3}{GroupGeometry is true, but the image size has been manually adjusted to an inferior value.}{0.6}\end{covimg}
%
%%
%%
%%The dimension of the sampled grid was 30x30x30 and the fill value of Sample
%%was set to 0.0.
%%
%%The module \covlink 
%%{CuttingSurface}{CuttingSurface}{../../Filter/CuttingSurface/CuttingSurface.html}
%% computes a cuttingsurface on the uniform grid and the module 
%%\covlink {Colors}{Colors}{../../Color/Colors/Colors.html} maps the
%%data on the surface to colors.
%%
%%The module
%%\covlink{ShowGrid}{ShowGrid}{../../Tools/ShowGrid/ShowGrid.html}
%% displays the uniform grid (in this case 3 sides of the outer surface).
%%
%%\begin{covimg2}{modules/Tools/Sample}{SampleRenderer}{0.7}\end{covimg2}
%%
%%\paragraph{Second example}
