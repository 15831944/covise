\begin{htmlonly}\begin{htmlonly}
\documentclass{covise}

\usepackage{html, htmllist}
\usepackage{color}
\usepackage{graphicx}
\usepackage{longtable}
\usepackage{palatino}
\usepackage{picins}
\usepackage[colorlinks,dvips]{hyperref}

	  
\bodytext{BGCOLOR=FFFFFF LINK=#0033cc VLINK=#0033cc}


% #1  mark defined by \label
% #2  a linktext 
% #3  a html link 
\newcommand{\covlink}[3]%
{\html{\htmladdnormallink{#1}{#3}}\latex{\hyperref[#1]{#2} (\ref{#1})}}


\newenvironment{covimg}[4]%
{ \html{\htmladdimg[ALIGN=CENTER]{#2.gif}}
 
 \latexonly
 \begin{figure}[htp]
  \begin{center}
   \includegraphics[scale=#4]{#1/#2}
   \caption{#3}
  \end{center}
 \end{figure}
 \endlatexonly
}

\newenvironment{covimg2}[3]%
{ \html{\htmladdimg[ALIGN=CENTER]{#2.gif}}
 
 \latexonly
 \begin{figure}[htp]
  \begin{center}
   \includegraphics[scale=#3]{#1/#2}
  \end{center}
 \end{figure}
 \endlatexonly
}

\definecolor{output}{rgb}{0.,0.,1.}
\definecolor{depend}{rgb}{1.,0.65,0.}
\definecolor{required}{rgb}{0.58,0.,0.83}
\definecolor{optional}{rgb}{0.,0.39,0.}

\end{htmlonly}

%=============================================================
%=============================================================


%=============================================================
\startdocument
\subsection{MergeAndNormals}
\label{MergeAndNormals}
%=============================================================


%
% short description what the module does
%

MergeAndNormals corrects a field of normal vectors
calculated by GenNormals. GenNormals does not correctly work out
the normals when the connectivity of the polygon object
is not seamless. This is unfortunately the case at issue.
Here the seams appear between neighbouring basic cells.
FixUsg might correct this problem, but this solution
would not be acceptable because the computation of FixUsg
is very time-consuming. MergeAndNormals may be much faster
than FixUsg because it directly looks for the places where seams
are present and instead of eliminating them, the {\sl wrong}
normal values are corrected.


%
% input of a module icon for example
% #1	path for eps
% #2  picture name
% #3  scale factor
\begin{covimg2}{modules/Customer/MergeAndNormals}{MergeAndNormals}{0.7}\end{covimg2}



%
% short information about versions 
%
%Sample is available since COVISE snap-2000-10 on all supported platforms.

%
%=============================================================
%\subsubsection{Parameters}
%=============================================================
%

%\covlink{Colors}{Colors}{../../Color/Colors/Colors.html}

%\begin{longtable}{|p{4.5cm}|p{2cm}|p{8.5cm}|}
%\hline
%   \bf{Name} & \bf{Type} & \bf{Description} \endhead
%\hline\hline
%	ndivMet & Scalar & Controls the number of divisions for the metal part.\\
%\hline
%	ndivPap & Scalar & Controls the number of divisions for the paper part.\\
%\hline
%\end{longtable}

%The model for the LS-DYNA simulation has three parts: metal, paper and rubber.
%The paper sheet is sandwiched between metal and rubber.


%
%=============================================================
\subsubsection{Input Ports}
%=============================================================
%


\begin{longtable}{|p{2.5cm}|p{4.5cm}|p{7cm}|}
\hline
   \bf{Name} & \bf{Type} & \bf{Description} \endhead
\hline\hline
	\textcolor{required}{InGeometry} & DO\_Polygons DO\_Lines & 
                 This object is simply reused at output.\\
\hline
	\textcolor{required}{InNormals} & DO\_Unstructured\_V3D\_Data & 
                    The vector field generated by GenNormals,
                    which is to be corrected.\\
\hline
	\textcolor{required}{Text} & DO\_Text & 
                    A message from ControlSCA with the information
                    of all design parameters. The relevant
                    parameters for this module are the basic
                    cell dimensions.\\
%\covlink{Transform}{Transform}{../../Tools/Transform/Transform.html}.\\
                     
                    
														
%	....
%	....

\hline
\end{longtable}
%=============================================================



%
%=============================================================
\subsubsection{Output Ports}
%=============================================================
%

 
\begin{longtable}{|p{2.5cm}|p{4.5cm}|p{7cm}|}
\hline
   \bf{Name} & \bf{Type} & \bf{Description} \endhead
\hline\hline
	\textcolor{required}{OutGeometry} & DO\_Polygons  DO\_Lines& 
                    Output polygons. This object is reused from
                    the input.\\
\hline
	\textcolor{required}{OutNormals} & DO\_Unstructured\_V3D\_Data& 
                    The corrected normal field.\\
%	....
%	....

\hline
\end{longtable}
%=============================================================


%%=============================================================
%\subsubsection{Examples}
%%=============================================================
%%
%
%% examples for using this module
%
%%\paragraph{First example}
%%
%\begin{covimg}{modules/Tools/ImageToTexture}%
%		{ImageToTextureMap1}{covise/net/examples/ImageToTexture.net}{0.6}\end{covimg}
%
%In the first example we show a dynamic geometry. As the geometry moves the image 
%moves with it. In order to achieve this effect, we use the displacement information
%at the second port.
%
%\begin{covimg}{modules/Tools/ImageToTexture}%
%		{ImageToTextureRend1}{The image is dragged by the material motion.}{0.6}\end{covimg}
%
%\begin{covimg}{modules/Tools/ImageToTexture}%
%		{ImageToTextureMap2}{covise/net/examples/ImageToTexture2.net}{0.6}\end{covimg}
%
%In the second example we want to illustrate the effect of the parameter {\sl GroupGeometry}
%and of size adjustment. The geometry is a set with 4 DO\_Polygon object. When
%the value of {\sl GroupGeometry} is true (default), the image is mapped once onto the
%whole geometry. This effect is seen in the first renderer image. If the value
%of this parameter is false, then we get the second image. Note that here the geometry
%is used separately for each DO\_Polygon object. The third image has the default
%value for this parameter, i.e. true. If we are seeing here many eyes, it is because
%we are no longer fitting the image size to that of the geometry. In this case, we
%have manually set the image size to an inferior value, that is why we have to replicate
%the image in order to create a texture for the whole geometry.
%
%\begin{covimg}{modules/Tools/ImageToTexture}%
%		{ImageToTextureRend2_1}{GroupGeometry is true, the image size is that of the geometry.}{0.6}\end{covimg}
%
%\begin{covimg}{modules/Tools/ImageToTexture}%
%		{ImageToTextureRend2_2}{GroupGeometry is false.}{0.6}\end{covimg}
%
%\begin{covimg}{modules/Tools/ImageToTexture}%
%		{ImageToTextureRend2_3}{GroupGeometry is true, but the image size has been manually adjusted to an inferior value.}{0.6}\end{covimg}
%
%%
%%
%%The dimension of the sampled grid was 30x30x30 and the fill value of Sample
%%was set to 0.0.
%%
%%The module \covlink 
%%{CuttingSurface}{CuttingSurface}{../../Filter/CuttingSurface/CuttingSurface.html}
%% computes a cuttingsurface on the uniform grid and the module 
%%\covlink {Colors}{Colors}{../../Color/Colors/Colors.html} maps the
%%data on the surface to colors.
%%
%%The module
%%\covlink{ShowGrid}{ShowGrid}{../../Tools/ShowGrid/ShowGrid.html}
%% displays the uniform grid (in this case 3 sides of the outer surface).
%%
%%\begin{covimg2}{modules/Tools/Sample}{SampleRenderer}{0.7}\end{covimg2}
%%
%%\paragraph{Second example}
